%%%%%%%%%%%%%%%%%%%%%%%%%%%%%%%%%%%%%%%%%%%%%%%%%%%%%%%%%%%%%%%%%%%%%%%%%%%
%%
%%  ch-magicsim.tex
%%
%%  Created: Fri Oct 10 14:24:37 1997
%%  Author.: Jose Carlos Gonzalez
%%  Notes..:
%%          
%%-------------------------------------------------------------------------
%% Filename: $RCSfile$
%% Revision: $Revision$
%% Date:     $Date$
%%%%%%%%%%%%%%%%%%%%%%%%%%%%%%%%%%%%%%%%%%%%%%%%%%%%%%%%%%%%%%%%%%%%%%%%%%%


\chapter{Simulaci'on del detector: \MAGIC}
\label{chapter:simmagic}

A program for the simulation of the detector is necessary in order to
study the response of our system (the \Cerenkov telescope) to the
input data (\Cerenkov light from the incoming electromagnetic and
hadronic showers). Several small simulation codes have been written.
In this chapter we present the philosophy of the main programs used
for this topic: the simulation codes \reflector and \camera.

The general flow of the simulation process is outlined in Fig.
\ref{fig:simprocess}. As it was already shown, the simulation of the
detector is splitted in three steps:
%
\begin{enumerate}[i.]
  
\item Generation of the simulated atmospheric showers, with the
  \CORSIKA code
  
\item Passage of the generated \Cherenkov light through a simulation
  of the atmosphere, and reflection of this light in the multi-mirror
  of \MAGIC
  
\item Pixelization process, simulation of trigger logic and image
  analysis

\end{enumerate}
%
The process of the generation of the atmospheric showers was already
explained in chapter \ref{chapter:simshowers}. Here I will talk about
the last two steps.

%------------------------------------------------------------
\section{Reflexi'on y recolecci'on de la luz en el plano focal}
\label{sec:reflcoll}

If we look at the output of CORSIKA, we will see some data files with
information about particles and \Cherenkov photons in each of the user
selected observation levels. We use only one main observation level
2\,200\u{m} a.s.l. In our investigations we are interested in the
\Cherenkov light. Therefore, the programs developed for the simulation
concentrate in the analysis of this light produced in the atmosphere.

The information stored in disk for each \Cherenkov photon was shown in
Table \fullref{tab:cphstruct}, and consists for each \Cherenkov photon
of the 7-tuple $\{\lambda,x,y,u,v,t,h\}$, where:
%
{\centering
\begin{tabular}{cl}
$\lambda$ & Wavelength of the emitted \Cherenkov photon \\
$x,y$     & Position of the \Cherenkov photon in the horizontal
            plane of the observation level \\
$u,v$     & Director cosines in X and Y directions of the incoming
            photon\\
$t$       & Time elapsed since the first interaction of the primary until the 
            photon was emitted\\
$h$       & Height of production of the photon\\
\end{tabular}
}

The quantities $x,y,u$ and $v$ are measured in the coordinate system
of the observation level (I will call this system the \emph{observer's
  coordinate system}). However, we use two more systems in our
simulation. In total we use these three coordinate systems (see
Fig.\ref{fig:coordsys}):
%
\begin{Ventry}{$S" \equiv O"X"Y"Z"$:}
\item[$S \equiv OXYZ$] The observer's coordinate system, where all
  the variables from the output of CORSIKA are measured
  
\item[$S' \equiv O'X'Y'Z'$] The system of the global frame of the
  telescope (we have that $O\equiv O'$)
  
\item[$S" \equiv O"X"Y"Z"$] The local system of the small mirror
  where each incoming photon is actually hitting
\end{Ventry}

\simprocessfig
%
In the simulation of the reflexion, we obey the following algorithm:
%
\begin{enumerate}[1.]
\item Read parameters for the simulation, allocate memory and
  initialize different variables, and open the files.

\item For each \Cherenkov photon in every shower to be analyzed,
  perform the following steps:
  
  \begin{enumerate}[\theenumi.1.]
  \item Calculate, for this photon, the transmitance of the
    atmosphere, and test whether it passed the absorption.
    
  \item Apply the reflectivity of the mirrors, possible dead zones and
    simulation of mirror imperfections.

  \item If the photon survives to all these effects, its way to the
    camera plane is traced and the photon is saved to the output file.
  \end{enumerate}

\item Go back to 2.1 while there are photons/showers left.
\end{enumerate}

Let's explain briefly the most important steps.

\subsection{Simulation of the Atmospheric Absorption}

In order to simulate the atmospheric attenuation suffered by the
Cherenkov light emitted in the development of an atmospheric shower,
an appropriate estimation of the quantities involved must be done.
These quantities are the \emph{true vertical height}, defined as the
vertical height below a point in the atmosphere, and the \emph{air
  mass}, defined as the density of air traversed by the photons. Since
the classical CORSIKA simply gives us the height of emission of a
photon in the vertical of the observer, we are forced to calculate the
values of these variables in our simulation programs. They are of
special importance in the case of High Zenith Angle (HZA) studies.

%\coordsysfig
\Systemsfig %% in landscape

\subsubsection{Calculation of the true vertical height}

For the simple simulation of HZA using our plane-parallel-atmosphere
version of CORSIKA, as well as for the calculation of the true
vertical height of a photon (at its emission point) and the airmass
(AM) it traverses, a very simple estimation can be done.

We will assume that our observer is located at point A (see
Fig.\ref{fig:geomview}), at a heigh \ho above sea level (a.s.l.).  Our
observer is looking at a source (which emits gamma rays, hopefully),
under a Zenith Angle $\theta$. One of these gamma rays just arrived at
the top of Earth's atmosphere, and developed an gamma ray induced
atmospheric shower. (For simplicity, we will assume that the
trajectory of the photon will cross the observation level at point A.)

I will use the following notation (from Fig.\ref{fig:geomview}):

\begin{tabular}{ll}
A & Point where the observer is located \\
O & Point at sea level, in observer's vertical \\
P & Point of emission of the photon \\
\ho & Height of the observation level \\
\hc & Height given by CORSIKA, in the vertical of the observer\\
\hv & Height in the vertical of the photon\\
$R \equiv R_\oplus$ & Earth's radius\\
$\theta$ & Zenith Angle in observer's level\\
$\omega$ & Zenith Angle at sea level, in observer's vertical\\
$\alpha$ & Angle from Earth's center between A and P\\
\end{tabular}

%\geomviewfig

If $\ho \neq 0$ nor negligible with respect to \hc, and $\theta \simeq
\omega$, the expression that gives us the value of the \emph{true
  vertical height} at P in the vertical, \hv, is:
%
\hveq
%
In the case of $\ho \simeq 0$ or negligible with respect to \hc, this
becomes:
%
\hvapproxeq

%------------------------------------------------------------
\subsubsection{Pure geometrical estimation of the air mass}

In the simplest approximation, the air mass factor (AM, the relation
between the optical travel path at a given zenith angle and the
optical path at the vertical) can be estimated from geometrical
consideratins. Using this simple approach, the air mass factor will
be:
%
\mgeomeq

%------------------------------------------------------------
\subsubsection{Calculation of the air mass using an 
exponential density model}

The above expression in Eq.\eqref{eq:mgeom} uses just geometrical 
arguments to estimate the ratio represented by m. We would like to go
one step beyond, by introducing an exponential density model. In our
simulations we are using an atmospheric model represented by 4
exponentials and a linear function in the top of the
atmosphere. Therefore, although the expression that we want to get is
not fully correct, will be closer to the reality.

We will use then the following density model:
%
\denseq
%
where $h$ is the height a.s.l., and \Hs is the \emph{scale height} of
the atmosphere. In the Earth this results to be approximately $\Hs =
7.4\u{km}$. With this model, the optical path (ignoring refraction)
can be written (with the approximation that $h/R \simeq 0$) as:
%
\optpathredeq
%
Note that we already wrote explicitly that we want to calculate the
travel path for a Zenith Angle $\theta$ for vertical heights between
\ho and \hv. Our \emph{air mass} $\mathcal{AM}$ is defined
as\footnote{The \emph{Airmass} in Optical Astronomy is usually defined
  as the relation $\mathit{Airmass} \equiv
  {I(\theta;0,\infty)}/{I(0\deg;0,\infty)} $}:
%
\AMdefeq
%
Therefore, integrating we have:
%
\AMfulleq
%
where $\mathrm{erfc}$ is the \emph{complementary error function}. An
approximation for low zenith angles, or more technically, for
moderately large $X\equiv\sqrt{{R\,\cos^2\theta}/{2\,\Hs}}$ is
%
\AMapproxeq

\paragraph{Correction for refraction.} We have defined $R \equiv
R_\oplus$, the radius of the Earth. A simple correction for refraction
can be obtained by taking instead
%
\refracapproxeq

%------------------------------------------------------------
\subsubsection{Comparison between different approximations}

First, we would like to compare the true expression for the
calculation of the \hv with the one obtained by assuming that \ho is
small enough. For this purpose we calculate the vertical height for
$\ho=0\u{km}$ and $\ho=2.2\u{km}$ (height of the future location of
MAGIC), for different Zenith Angles, and plot the difference in
percentage (see Fig.\ref{fig:hvdiffs}). We can see that the difference
increases with the zenith angle, as expected, although still this
difference remains far below 5\% up to 85\deg.

\hvdiffsfig

The second thing we want to compare is the geometrical expressions for
the air mass, namely $m$ and $m_{\mathrm{simple}}$ (the simple
approximation for $ho=0\u{km}$), with the more elaborated calculation
using the exponential density model, $mathcal{AM}$. We will include in
the last expression the simple correction for refraction given in
Eq.\eqref{eq:refrac}. This comparison, for three value of \ho
(0\u{km}, 4\u{km} and 2.2\u{km}), is shown in Fig.\ref{fig:AMcomp}. We
have used in this case two values for the vertical height,
$\hv=5\u{km}$ and $\hv=100\u{km}$.

\AMcompfig

%------------------------------------------------------------
\subsubsection{Calculation of the atmospheric transmission}

Three effects have been included in the final calculation of the
transmission of the atmosphere for the Cherenkov photons:

\paragraph{Rayleigh scattering.} The molecules in the atmosphere
produce scattering of the photons. This effect is strongly correlated
with the wavelength $\lambda$ of the light, being more important for
small wavelengths. The transmission coefficient due to Rayleigh
scattering is:
%
\Rayleigheq
%
where $\tau_{i=1,2} = \tau_0 \, \exp(-h_i/H_{\mathrm{S}}) \sec\theta$
is the \emph{slanted thickness} above a height $h_i$, in the case of
an exponential density profile (actually in our case we calculatec
these values with the CORSIKA density model),
$\tau_{\mathrm{R}}(\lambda=400\u{nm}) = 2970\u{g/cm^2}$ is the mean
free path of the Rayleigh scattering in terms of thickness, $\tau_0 =
0.00129\u{g/cm^2}$, and $H_{\mathrm{S}}$ is the mentioned scale-height
of the atmosphere. We pre-calculated the exponents for different
wavelengths.
  
\paragraph{Mie scattering.} This effect is due to the aerosol (dust)
particles suspended inthe air, being its effect bigger at lower
heights. It depends also on the size of these particles. For an
exponential density profile, the transmission coefficient is
%
\Mieeq
%
where $l_{\mathrm{M}}(400\u{nm}) \simeq 14\u{km}$ is the mean free
path for the Mie scattering, and $h_{\mathrm{M}} \simeq 1.2\u{km}$ is
the scale-height for the aerosol distribution. We have pre-calculated
the exponents for different wavelengths, although this dependence is
very small.
  
\paragraph{Ozone absorption.} This effect is very important in the
range 280--340\u{nm} and for low energy showers, in which most of the
Cherenkov light emission is located where the density of ozone is high
(between 20--30\u{km}). For this we have first estimated the
extinction in magnitudes per unit of air mass:
%
\amagneq
%
where $k(\lambda)$ is the absorption coefficient in \u{cm^{-1}}, and
$T(\hv)$ is the total ozone concentration for a typical Tropical
region (we coose this because was very similar to the actual profile
in La Palma), above the vertical in each position, in \u{atm cm}. The
coefficient 1.11 was chosen to correspond to Elterman's optical
thickness for ozone at 320\u{nm}\cite{Elterman}.

With this value, we can estimate the extinction in magnitudes per air
mass between two height, and using the Pogson law we can determine the
transmission coefficient, which results (for more details, please
refer to \cite{Aitor:tesina}):
%
\Ozoneeq

Of course, the total transmission coefficient will be:
%
\TotalTransmissioneq

%------------------------------------------------------------
\section{Conclusions}

I hope that with this small document, the way we calculate the
parameters for the simulation of the atmospheric attenuation is a bit
more clear. Note that the more appropriate procedure to calculate the
air mass for a given Zenith angle, between two vertical height, is, at
the time of writing these note, \emph{not} yet implemented in the
simulation code.

The right equations should be used, when available and when the cost
in computing time is inexistent with respect to any other more simple
approach. However, from our studies it becomes clear that, under some
conditions, simple approximations can give results that are accurate
enough for most of the studies.



\randpoifig

\section{Detecci'on en la c'amara}

\subsection{Simulaci'on de la cadena de proceso de la se�al}

\subsection{L'ogica de disparo}

\section{Resultados}

\subsection{Eficiencias de disparo}

\subsection{'Areas de colecci'on efectivas}

\subsection{Ritmos de detecci'on}

\subsubsection{Ritmos diferenciales de detecci'on}

\subsubsection{Ritmos integrales de detecci'on}

\subsubsection{Fondos hadr'onico, electr'onico y mu'onico}

\endinput
%
%% Local Variables:
%% mode:latex
%% End:

%%EOF
