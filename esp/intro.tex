%%%%%%%%%%%%%%%%%%%%%%%%%%%%%%%%%%%%%%%%%%%%%%%%%%%%%%%%%%%%%%%%%%%%%%%%%%%
%%
%%  intro.tex
%%
%%  Created: Fri Oct 10 14:24:37 1997
%%  Author.: Jose Carlos Gonzales
%%  Notes..:
%%          
%%-------------------------------------------------------------------------
%% Tesis  :: RCS controlled system
%% Filename: $RCSfile$
%% Revision: $Revision$
%% Date:     $Date$
%% $Id$
%%%%%%%%%%%%%%%%%%%%%%%%%%%%%%%%%%%%%%%%%%%%%%%%%%%%%%%%%%%%%%%%%%%%%%%%%%%
%

\chapter*{Introducci'on}
\label{chapter:intro}
\markboth{Introducci'on}{Introducci'on}
\renewcommand{\headname}{Introducci'on}
\addcontentsline{toc}{chapter}{\numberline{}Introducci'on}
%
En los 'ultimos a~nos la {\bfseries \I{Astrof'isica} de \II{Alta
Energ'ia}{Astrof'isica}}, disciplina que se ocupa del estudio de la
\I{radiaci'on c'osmica} de muy alta y ultra alta energ'ia, ha
sufrido un impulso considerable. Desde que a principios de
siglo los pioneros en el estudio de rayos c'osmicos hicieran
sus primeras medidas en globos, este campo de la F'isica, a
caballo entre la Astrof'isica y la F'isica de Part'iculas,
no ha dejado de aportar novedades que sirvieran para ahondar
en el conocimiento de la radiaci'on cosmica. Pero ha sido en
los 'ultimos diez a~nos cuando la observaci'on directa de
fuentes de rayos gammas por detectores en tierra y a bordo
de sat'elites ha producido un efecto de {\itshape bola de
nieve}. Las observaciones de fuentes astrof�sicas de rayos
gamma, conocidas en otras longitudes de onda, y la
detecci'on de nuevas fuentes desconocidas se suceden. Esta
expansi'on de la Astrof'isica de Alta Energ'ia queda patente
cuando nos damos cuenta de que las observaciones no se
limitan ya a la detecci'on de se~nales provinientes de
fuentes en el cielo, sino que se ha pasado al intento de
interpretaci'on astrof'isica de los resultados
observacionales, a la construcci'on de modelos f'isicos que
expliquen la generaci'on y el transporte de la radiaci'on
c'osmica observada.

A'un existen dentro de este vasto campo de investigaci'on
muchos misterios por desvelar. Estos misterios se refieren
fundamentalmente a la naturaleza de diferentes objetos
astrof'isicos. Entre todos ellos podemos citar las fuentes
de rayos $\gamma$, los \Iw{n'ucleos de galaxias
activas}{AGNs} (\I{AGNs}), los objetos \I{BL-Lacert\ae}, las
\I{novas}, las \I{supernovas}, las \I{estrellas de neutrones}, los
\I{p'ulsares}, los \I{agujeros negros}, los \I{GRBs} ({\sl \Iw{Gamma Ray
Bursts}{GRBs}}), o los \Iw{cu'asares}{QSOs} (\I{QSOs}).

Las observaciones dentro de la Astrof'isica de Alta Energ'ia
se han dividido tradicionalmente en dos tipos. En primer
lugar se encuentra el estudio de los {\itshape rayos c'osmicos},
part'iculas de alta energ'ia que viajan por el espacio desde
el momento en que son generadas por alg'un objeto
astrof'isico. Compartiendo recursos y conocimientos con el
campo de los rayos c'osmicos podemos destacar tambi'en el
estudio de las fuentes astrof'isicas de rayos $\gamma$.

En el campo de los rayos c'osmicos se han realizado avances
destacables como puede ser la medici'on del espectro de muy
alta y ultra-alta energ'ia. Existen medidas directas
realizadas por sat'elites y globos hasta energ'ias del orden
de los $\u{PeV}$ ($1\u{PeV}=10\pow{15}\u{eV}$), as'i como medidas
indirectas realizadas por detectores a nivel del suelo hasta
los $10\u{EeV}$ ($1\u{EeV}=10\pow{18}\u{eV}$)\footnote{Sobre
unidades y ordenes de magnitud, ver apendice
\ref{ap:units}.}. Tambi'en se han realizado importantes
estudios en el 'ambito de la determinaci'on de la
composici'on qu'imica de los rayos c'osmicos.

Todos estos avances en la F'isica de Rayos C'osmicos son de
vital importancia para la Astrof'isica de Altas
Energ'ias. Por medio del estudio de las caracter'isticas
tanto cualitativas como cuantitativas del flujo de rayos
c'osmicos, que impregna todo el espacio, se puede
discriminar qu'e modelos para la creaci'on, emisi'on y
transporte de estas part'iculas resultan v'alidos. Tambi'en
permite dilucidar los modelos que puedan dar cuenta de cada
uno de los misteriosos y ex'oticos objetos que pueblan
nuestro Universo. Y por 'ultimo, pueden permitir el
descubrimiento de nuevas propiedades, de nuevos escenarios
f'isicos en el cosmos, y servir de filtro para aquellas {\sl
ideas locas} que de forma tan apasionante y prol'ifica salen
a la luz dentro de la Astrof'isica.

Si la F'isica de Rayos C'osmicos puede parecer de vital
inter'es para la Astrof'isica en general, e incluso para la
F'isica de Part'iculas y la F'isica Te'orica, como
herramienta que aporta tests para la verificaci'on
experimental de modelos te'oricos, a'un m'as importante es
la detecci'on y el estudio de fuentes astrof'isicas de
radiaci'on c'osmica. Es en el cap'itulo \ref{ch:radcosmica}
donde damos cuenta de manera breve de los diferentes tipos
de fuentes de radiaci'on c'osmica. Tambi'en realizamos un
somero repaso a la f'isica de la radiaci'on c'osmica, en
cuanto a generaci'on, composici'on y transporte, hasta su
llegada a la Tierra.

A continuaci'on nos dedicamos a una descripci'on de los
m'etodos de simulaci'on de las cascadas atmosf'ericas, y de
los observables que se pretenden medir en todo experimento
de rayos c'osmicos y radiaci'on gamma (cap'itulo
\ref{ch:simulacion}). Nos centraremos en el c'odigo \CORSIKA
(secci'on \ref{sec:CORSIKA}), y en las modificaciones
realizadas en 'el. Finaliza este cap'itulo con una
descripci'on breve de otros tipos de simulaci'on empleados
(secci'on \ref{sec:simotros}).

En el siguiente cap'itulo abordamos la descripci'on del
detector \MAGIC, en fase de proyecto, el cual consta de un
espejo octogonal teselado de unos 17\u{m} de diametro. Se
exponen de manera concisa las diferentes caracter'isticas de
los detectores, del reflector, as'i como del modo
operacional previsto. Brevemente, como comparaci'on y debido
a la naturaleza de este trabajo, se hablar'a tambi'en del
telescopio CT1 de la colaboraci'on HEGRA ({\itshape High
Energy Gamma Ray Astronomy}).

En los siguientes cap'itulos (n'umeros \ref{ch:magic}, 
\ref{ch:simdetector} y \ref{ch:simdetectorCT1}) 
describimos la manera en que hemos realizado la simulaci'on
del ambos detectores, \MAGIC y CT1. Se muestran los
resultados obtenidos en cuanto al c'alculo de diferentes
par'ametros f'isicos como la {\itshape eficiencia de
detecci'on}, {\itshape 'areas efectivas de colecci'on}, o
los {\itshape ritmos diferenciales e integrales de
detecci'on}.

En el cap'itulo \ref{ch:analysis} hemos realizado un 
an'alisis de las imagenes obtenidas, describiendo y
utilizando primero diferentes tipos de algoritmos para la
limpieza de las im'agenes, y calculando despu'es diferentes
par'ametros de las mismas. Estos par'ametros nos permitiran
discernir, en mayor o menor medida, el tipo de primario
causante de la cascada atmosf'erica detectada, y, por tanto,
ser'an susceptibles de aportar criterios para la separaci'on
de la se~nal de una posible fuente, del fondo is'otropo de
radiaci'on c'osmica.

Con la simulaci'on realizada de cascadas atmosf'ericas y de
los detectores \MAGIC y CT1, somos capaces de calcular
par'ametros de los mismos, como pueden ser la {\itshape
resoluci'on angular} o la {\itshape incertidumbre en
energ'ia}. Estos c'alculos se exponen en el cap'itulo
\ref{ch:resolucion}.

Basados en los par'ametros estudiados en los cap'itulos
anteriores, se estudiar'a la capacidad de separaci'on de la
se~nal del fondo ({\itshape separaci'on gamma-hadr'on}),
fundamental para la busqueda de se~nales provinientes de
fuentes a'un no observadas o incluso desconocidas. A ello
dedicaremos el cap'itulo \ref{ch:sepgh}. Se ha estudian y
desarrollan tanto los m'etodos tradicionales, aplicados a un
nuevo detector como lo es \MAGIC, as'i como nuevos m'etodos
para la separaci'on gamma-hadr'on.

Para finalizar se realiza una exposici'on de las
conclusiones m'as importantes del trabajo aqu'i presentado.

Con este trabajo se ha pretendido realizar un an'alisis en
concreto de unos datos de simulaci'on determinados, y
esbozar nuevos criterios para la separaci'on {\itshape
gamma-hadr'on}. Pero tambi'en se ha intentado dar una idea
sobre la importancia del estudio de los rayos c'osmicos y de
las fuentes astrof'isicas de rayos $\gamma$, campos de
investigaci'on ambos, que, sin ninguna duda, ampliar'an muy
pronto nuestro horizonte de conocimientos y nos desvelar'an
nuevas maravillas que a'un nos esperan aqu'i, en nuestro
Universo.

\endinput
%
%% Local Variables:
%% mode:latex
%% End:

%%EOF
