%%%%%%%%%%%%%%%%%%%%%%%%%%%%%%%%%%%%%%%%%%%%%%%%%%%%%%%%%%%%%%%%%%%%%%%%
% ATR - Notes about the calculation of the ATR due to LONS
%
% Filename: $RCSfile$
% Author:   $Author$
% Revision: $Revision$
% Date:     $Date$
%%%%%%%%%%%%%%%%%%%%%%%%%%%%%%%%%%%%%%%%%%%%%%%%%%%%%%%%%%%%%%%%%%%%%%%%

\documentclass{article}

%\usepackage{emulateapj}
\usepackage{a4wide}
\usepackage{amstext}
\usepackage{amsmath}
\usepackage{epsfig}
%\usepackage[spanish,activeacute]{babel}
\usepackage{xspace}
%\usepackage{wrapfig}
%\usepackage{floatflt}
%\usepackage{charter}

\def\Cherenkov{Cherenkov\xspace}

\def\cph{Ch.\ensuremath{\gamma}\xspace}
\def\cphs{Ch.\ensuremath{\gamma}s\xspace}

\def\phe{\u{ph.e^{-}}\xspace}
\def\phes{\u{ph.e^{-}s}\xspace} 

\def\ATR{\ensuremath{\mathrm{ATR}}\xspace}
\def\Ncomb{\ensuremath{\mathcal{N}_{\mathrm{comb}}}\xspace}
\def\Npix{\ensuremath{N_{\mathrm{pix}}}\xspace}
\def\Rate{\ensuremath{R}\xspace}
\def\Time{\ensuremath{\Delta T}\xspace}

\renewcommand{\thefootnote}{\fnsymbol{footnote}}

\renewcommand{\u}[1]{\ensuremath{\,\mathrm{#1}}}  %% units

\setlength{\parindent}{0pt}
%\setlength{\parskip}{2pt}

\pagestyle{empty}

%%% BEGIN %%%%%%%%%%%%%%%%%%%%%%%%%%%%%%%%%%%%%%%%%%%%%%%%%%%%%%%%%%%%

\begin{document}

\begin{center}
  {\large \scshape Notes on the estimation of the Accidental Trigger
        Rate due to the Light of Night Sky\\}
  J. C. Gonz\'alez\\
  {\small Max-Planck-Institut f\"ur Physik, 
  F\"ohringer Ring 6, D-80805 M\"unchen, Deutschland\\
  E.mail: \texttt{gonzalez@mppmu.mpg.de}\\}
  January, 2000
\end{center}
\vskip 1cm


%------------------------------------------------------------

\begin{abstract}
The Artificial Trigger Rate (\ATR) due to the Light of Night Sky
(LONS) must be known in order to better understand the working
conditions of our \Cherenkov telescope. In this small note I try to
estimate such \ATR for different trigger conditions.
\end{abstract}
%------------------------------------------------------------

\section*{Light of the Night Sky}

The amount of light from the night sky (LONS) has been measured, and
is around 
\begin{equation*}
\langle\mathrm{LONS}\rangle = 2\cdot 10^{12} \u{ph}/\u{m}^2\u{s}\u{sr}
\end{equation*}

\subsection*{Estimation of the LONS per pixel for MAGIC}

For this calculation, the following parameters have to be taken into
account:

\begin{center}
\begin{tabular}{ll}
$S_{\text{mirror}} = 230 \u{m}^2 $ &
$\text{Reflectivity} = 80\% $ \\
$\epsilon_{\text{l.guides}} = 90\% $ &
$\epsilon_{\text{plexiglas}} = 95\% $ \\
$\epsilon_{1^{\mathrm{st}}\text{dyn.coll.}} = 90\%$ &
$\theta_{\text{1pixel}} = 0.1^\circ$ \\
$\theta_{\text{h.pixel}} = 0.05^\circ$ &
$\mathrm{QE}_{\mathrm{LONS}} \sim 13\% $ \\
$\Delta\Omega = 2\pi(1-\cos\theta_{\text{h.pixel}}) 
= 2.39\cdot 10^{-6} \u{sr}$ \\
\end{tabular}
\end{center}

Then, the mean number of photons arriving at the entrance of the pixel
in $1\u{ns}$ is:

\begin{eqnarray*}
\mathcal{N}_{\mathrm{in}} 
&=& \langle\mathrm{LONS}\rangle 
\cdot t 
\cdot S_{\text{mirror}}
\cdot \epsilon_{\text{l.guides}}
\cdot \epsilon_{\text{plexiglas}}
\cdot \Delta\Omega \\
&=& (2\cdot 10^{12} \u{ph}/\u{m}^2\u{s}\u{sr})
\cdot (10^{-9} \u{s}/\u{ns}) 
\cdot (230 \u{m}^2) 
\cdot (0.90) 
\cdot (0.95) 
\cdot (2.39\cdot 10^{-6} \u{sr})  \\
&=& 0.94 \u{ph}/\u{ns} 
\end{eqnarray*}

Since our mean QE for the LONS is $\mathrm{QE}_{\mathrm{LONS}} \sim
13\%$, this means:

\begin{equation*}
\mathcal{N}_{\mathrm{in}}' = \mathcal{N}_{\mathrm{in}} 
\cdot \mathrm{QE}_{\mathrm{LONS}} 
\cdot \epsilon_{1^{\mathrm{st}}\text{dyn.coll.}}
= 0.11 \phe/\u{ns}
\end{equation*}

If we use then a gate of $\Delta T=10\u{ns}$, we arrive at a mean
contribution of LONS per pixel per gate of:

\begin{equation*}
\langle\text{LONS}\rangle_{\text{1 pixel}} =
\mathcal{N}_{\mathrm{in}}' \cdot \Delta T = 1.10 \phe/\u{gate}
\end{equation*}

\section*{Estimation of the Number of Combinations}

For the estimation of the Artificial Trigger Rate (\ATR), we have to
take into account several parameters:

\begin{enumerate}

\item The trigger pattern used for the trigger: it can be \emph{simple
  multiplicity} (SM) of $n$ pixels above a single pixel threshold, or a
  \emph{next-neighbor} (NN) trigger of $n$ pixels above the threshold,
  or any other pattern.

\item The single pixel threshold, in photoelectrons, milivolts, 
  \ldots

\item The size of the camera, or more specifically, the size of the
  trigger region in the camera, and its topology.

\end{enumerate}

In addition, of course, we have to include the rate of the LONS.

For the calculation of the \ATR, then, we will use the following
formula:

\begin{equation*}
\ATR = \Ncomb \Rate^{n} \Time^{(n-1)}
\end{equation*}

where \Rate is the \emph{singles} rate, that is, the trigger rate of a
given individual pixel due to the LONS, $n$ is the number of pixels
involved in the trigger, \Time is the time window fixed in our trigger
logic, and \Ncomb is a geometric term which gives the number of
possible combinations in the camera for the user-defined trigger
pattern.

Let's assume we have $m$ pixels in the trigger region (it can be the
whole camera). In the case of \emph{simple multiplicity} trigger
conditions for $n$ pixels above threshold (I will call this
SM$_n$-trigger), the geometrical term is simply:

\begin{equation*}
\Ncomb = \binom{m}{n} = \frac{n!}{m!\,(m-n)!}
\end{equation*}

In the case of \emph{next-neighbor} trigger conditions for $n$ pixels
above threshold (which I will call NN$_n$-trigger) the situation is
much more complex. For a hexagonal camera or trigger region with
hexagonal pixels (as is usually the case) with $r$ rings of pixels,
plus the one in the center, the number of pixels is:

\begin{equation*}
\Npix(r) \equiv \Npix^r = 3 r (r+1) + 1
\end{equation*}

with this, we have (assuming $r>3$, in order to avoid problems) the
results shown in Table \ref{table:Ncomb}.

\begin{table}[b]
\begin{center}
  \begin{tabular}{crl}
    Trigger Condition & \Ncomb \\
    \hline
    NN$_2$  & 
    $3\left(2n+2\overset{2n-1}{\underset{i=n}\sum} i\right) =$& $9n^2+3n$ \\
%    $[6 \cdot 3 + 6(r-1) \cdot 4 + \Npix^{(r-1)}\cdot 6] / 2$ \\
    NN$_3$  & 
    $2\left(\overset{2n}{\underset{i=n+1}\sum} i + 
            \overset{2n-1}{\underset{i=n}\sum} i\right) =$& $6n^2$ \\
%    $[6 \cdot 2 + 6(r-1) \cdot 3 + \Npix^{(r-1)}\cdot 6] / 3$ \\
    NN$_4$  & 
    $3\left(2n+2\overset{2n-1}{\underset{i=n+1}\sum} i\right) =$& $9n^2-3n$ \\
%    $[6 \cdot 3 + 6(r-1) \cdot 5 + 6 \cdot (6+4) + 6(r-2)\cdot(6+5) + 
%    \Npix^{(r-2)}\cdot (6+6)] / 4$ \\
  \end{tabular}
  \caption[Number of combinations for NN-triggers]{Number of 
     combinations (combinatorial factor) for NN-triggers}
  \label{table:Ncomb}
\end{center}
\end{table}

For instance, let's assume we have 217 pixels (number of rings is
$r=8$). Then, for different configurations of trigger, we will have
the combinatorial factors shown in Table \ref{table:Ncomb_example}.

\begin{table}[t]
\begin{center}
  \begin{tabular}{cr}
    Trigger Condition & \Ncomb \\
    \hline
    SM$_2$  & 23\,436 \\
    SM$_3$  & 1\,679\,580 \\
    SM$_4$  & 89\,857\,530 \\
    NN$_2$  & 600 \\
    NN$_3$  & 384 \\
    NN$_4$  & 552 \\
  \end{tabular}
  \caption[Number of combinations for different trigger
     schemes]{Number of combinations for different trigger schemes, 
     for the case of $r=8$, or 217 pixels.}
  \label{table:Ncomb_example}
\end{center}
\end{table}

\section*{Estimation of the Artificial Trigger Rate}

Let's assume we have a trigger window of $\Time = 10\u{ns} =
10^{-8}\u{s}$. In this case, we saw that for 1 pixel

\begin{equation*}
\langle\text{LONS}\rangle = 1.10 \phe/\u{gate}
\end{equation*}

We then fix a value for the single pixel threshold, $q_0$. Let's
assume we take a value of $q_0 =
7\,\phe/\mathrm{pixel}/\mathrm{gate}$.  In our trigger gate, \Time,
the mean of the Poisson distribution describing the incoming LONS
photons is $\lambda = \langle\text{LONS}\rangle = 1.10
\phe/\u{gate}$.  The probability of getting a charge $q$ in one
pixel due to the LONS, such that $q \geq q_0 \equiv
7\phe/\mathrm{pixel}/\mathrm{gate}$ is:

\begin{equation*}
\mathcal{P}(q \geq q_0) = 
\overset{\infty}{\underset{k=q_0}\sum} \mathcal{P}(q = k) =
\overset{\infty}{\underset{k=q_0}\sum}
\frac{\lambda^k}{k!}e^{-\lambda} =
1-\overset{q_0-1}{\underset{k=0}\sum} 
\frac{\lambda^k}{k!}e^{-\lambda}
\end{equation*}

If we want to use the exact values, this last expression is the best
to be used. However, if $q_0 > \lambda$ (as is the case here) we can
use an expansion of the first expression, using the property of the
Poisson distribution $\mathcal{P}(r+1) = [\lambda/(r+1)]\mathcal{P}(r)$

\begin{eqnarray*} 
\mathcal{P}(q \geq q_0) &=& 
\overset{\infty}{\underset{k=q_0}\sum} \mathcal{P}(q = k) \\
&=& \mathcal{P}(q = q_0) + \mathcal{P}(q = q_0+1) +  
\mathcal{P}(q = q_0+2) + \ldots \\
&=& \mathcal{P}(q = q_0) + 
\frac{\lambda}{(q_0+1)} \mathcal{P}(q = q_0) +
\frac{\lambda^2}{(q_0+1)(q_0+2)} \mathcal{P}(q = q_0) + \ldots \\
&=& \left[
1 + \frac{\lambda}{(q_0+1)} + \frac{\lambda^2}{(q_0+1)(q_0+2)} + \ldots
\right] \left( \frac{\lambda^{q_0}}{q_0!}e^{-\lambda} \right)
\end{eqnarray*}

We can then simply take the first two or three terms of this
expansion, since the rest are going to be very small.

Let's take for our example only the first two terms. In this case:

\begin{equation*}
\mathcal{P}(q \geq 7\phe) \approx 
\left[ 1 + \frac{1.10}{(7+1)} \right] 
\left( \frac{1.10^{7}}{7!}e^{-1.10} \right) = 1.46\cdot 10^{-4}
\end{equation*}

With tihs probability, we can calculate the rate \Rate of a single
pixel triggering

\begin{equation*}
\Rate = \mathcal{P}(q \geq 7\phe) / \Time = 14640 \u{Hz} 
\approx 15 \u{kHz}
\end{equation*}

This is the \emph{singles-rate}. Using the formula given at the
beginning, we would have, for instance in the case of NN$_4$ trigger:

\begin{equation*}
\ATR(\mathrm{NN}_4) = 
\Ncomb(\mathrm{NN}_4) \Rate^4 \Time^3 =
552 \cdot 14640^4 \cdot (10^{-8})^3 = 2.5\cdot 10^{-5} \u{Hz}
\end{equation*}

In the Table \ref{table:results1} I show the results of these
calculations for different trigger patterns and thresholds.

\begin{table}[t]
\begin{center}
  \begin{tabular}{cr}
  \end{tabular}
  \caption[Artificial Trigger Rates]{Artificial Trigger Rates
    calculated for different trigger configurations and single pixel
    thresholds. In all cases a trigger region of $r=8$ rings, or 217
    pixels, has been used.}
  \label{table:results1}
\end{center}
\end{table}

\end{document}

%%% END

%%%%%%%%%%%%%%%%%%%%%%%%%%%%%%%%%%%%%%%%%%%%%%%%%%%%%%%%%%%%%% 
% Last log: $Log$
% Last log: Revision 1.2  2000/01/31 07:35:43  gonzalez
% Last log: *** empty log message ***
% Last log:
%%EOF

