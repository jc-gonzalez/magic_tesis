\input texinfo @c -*-texinfo-*-
@c %%%%%%%%%%%%%%%%%%%%%%%%%%%%%%%%%%%%%%%%%%%%%%%%%%%%%%%%%%%%
@c Tesis  :: RCS controlled system                           
@c Filename: $RCSfile$                          
@c Revision: $Revision$                            
@c Date:     $Date$                    
@c $Id$
@c %%%%%%%%%%%%%%%%%%%%%%%%%%%%%%%%%%%%%%%%%%%%%%%%%%%%%%%%%%%%
@setfilename tesis-sch.info
@settitle Esquema de Tesis

@c %% Definiciones %%%%%%%%%%%%%%%%%%%%%%%%%%%%%%%%%%%%%%%%%%%%%%%%%%
@c ----  A C T U A L I Z A R   L A   F E C H A  ----
@set lastupdate 6 de Octubre de 1998
@set YO J C Gonz@'alez
@c %%%%%%%%%%%%%%%%%%%%%%%%%%%%%%%%%%%%%%%%%%%%%%%%%%%%%%%%%%%%%%%%%%

@c \@\(item)
@c %**end of header 

@ifinfo
@format
START-INFO-DIR-ENTRY
* Tesis: (standards).           Esquema de Tesis
END-INFO-DIR-ENTRY
@end format
@end ifinfo

@c @setchapternewpage odd
@setchapternewpage off

@c This is used by a cross ref in make-stds.texi
@set CODESTD  1
@iftex
@set CHAPTER chapter*
@end iftex
@ifinfo
@set CHAPTER node
@end ifinfo

@ifinfo
Este es el esquema de mi tesis ... y ol@'e!
@ignore
... y ol@'e!
@end ignore
@end ifinfo

@titlepage
@title Esquema de Tesis
@subtitle Simulaci@'on de cascadas de Muy Alta Energia
@subtitle para el proyecto "The MAGIC Telescope"
@author @value{YO}
@author Max-Planck-Institut fuer Physik, Muenchen
@author @value{lastupdate}

@page

@vskip 0pt plus 1filll
Copyright @copyright{} 1998, J C Gonzalez
@end titlepage

@ifinfo
@node Top
@top Esquema de Tesis

Last updated @value{lastupdate}.
@end ifinfo

@menu
* Esquema de Tesis::            Este es el esquema de mi tesis
@end menu

@node Esquema de Tesis
@unnumbered Esquema de la tesis
@unnumberedsec Contenidos

@enumerate

@item
Introducci@'on
@c--------------------------------------------------
@item
Los Rayos C@'osmicos
@enumerate a 
@item
Origen de los rayos c@'osmicos
@itemize @bullet
@item
Fuentes puntuales
@itemize @minus
@item
Supernovas
@item
P@'ulsares
@item
N@'ucleos Activos de Galaxias (AGNs)
@item
Cu@'asares
@item
@emph{Gamma-Ray Bursts}
@end itemize
@item
Fuentes extensas
@itemize @minus
@item
Halos
@item
@emph{WIMPs}
@item
Cuerdas c@'osmicas
@end itemize
@end itemize
@item
La F@'isica de los Rayos C@'osmicos
@itemize @bullet
@item
Formaci@'on de los rayos c@'osmicos
@itemize @minus
@item
Rayos c@'osmicos cargados
@item
Rayos gamma
@item
Neutrinos
@item
Espectro de la radiaci@'on c@'osmica
@item
Composicon qu@'imica de los rayos c@'osmicos
@end itemize
@item
Propagaci@'on a trav@'es de los medios 
              interestelares e intergal@'actico
@itemize @minus
@item
Difusi@'on en las regiones de formaci@'on
@item
Absorci@'on en el medio intergal@'actico
@item
Absorci@'on en el medio interestelar
@end itemize
@item
Los rayos c@'osmicos a su entrada en la atm@'osfera
@itemize @minus
@item
Generaci@'on de las cascadas
@item
Frente de part@'iculas
@item
Frente de luz Cherenkov
@item
Observables de las cascadas atmosf@'ericas
@item
@dots{}
@end itemize
@end itemize
@end enumerate

@item
Simulaci@'on de Cascadas Atmosf@'ericas
@enumerate a
@item
Simulaci@'on por metodos de Monte Carlo
@item
El C@'odigo CORSIKA
@item
Generaci@'on de luz Cherenkov
@item
Generaci@'on a alto @'angulo cenital
@item
Informaci@'on obtenida
@item
Estructura temporal del pulso Cherenkov
@item
Otros m@'etodos de simulaci@'on
@end enumerate

@item
El Telescopio @emph{MAGIC}
@enumerate a
@item
Objetivos f@'isicos
@itemize @bullet
@item
@dots{}
@end itemize 
@item
Caracter@'isticas del reflector
@item
La c@'amara de @emph{MAGIC}
@itemize @bullet
@item
Descripci@'on general
@item
Dispositivos de deteccion
@itemize @minus
@item
Dispositivos cl@'asicos: fotomultiplicadores (PMTs)
@item
Dispositivos avanzados: IPCs
@item
El futuro: dispositivos h@'ibridos y fotodiodos de avalancha
@end itemize
@end itemize
@item
Adquisici@'on de datos
@item
Programas observacionales
@item
El telescopio CT1 de la colaboraci@'on HEGRA
@end enumerate

@item
Simulacion del detector: @emph{MAGIC}
@enumerate a
@item
Eficiencias de Detecci@'on
@item
Colecci@'on en el plano focal
@item
Detecci@'on en la c@'amara
@itemize @bullet
@item
Simulaci@'on de la electr@'onica
@item
Descripci@'on de la l@'ogica de @emph{trigger}
@item
Eficiencias de trigger
@item
Areas de colecci@'on efectivas
@item
Ritmos de detecci@'on
@itemize @minus
@item
Ritmos diferenciales de detecci@'on
@item
Ritmos integrales de detecci@'on
@item
Fondos hadr@'onico, electr@'onico y mu@'onico
@end itemize
@end itemize
@end enumerate

@item
Simulacion del detector: @emph{CT1}
@enumerate a
@item
Eficiencias de Detecci@'on
@item
Colecci@'on en el plano focal
@item
Detecci@'on en la c@'amara
@itemize @bullet
@item
Simulaci@'on de la electr@'onica
@item
Descripci@'on de la l@'ogica de @emph{trigger}
@item
Eficiencias de trigger
@item
Areas de colecci@'on efectivas
@item
Ritmos de detecci@'on
@itemize @minus
@item
Ritmos diferenciales de detecci@'on
@item
Ritmos integrales de detecci@'on
@item
Fondos hadr@'onico, electr@'onico y mu@'onico
@end itemize
@end itemize
@end enumerate

@item
Analisis de Imagen
@enumerate a
@item
Limpieza de las im@'agenes
@item
Parametros de Imagen
@item
Analisis estad@'istico de las imagenes
@end enumerate

@item
Resoluciones Angular y Energ@'etica de MAGIC
@enumerate a
@item
Estimaci@'on de la resoluci@'on angular
@item
C@'alculo de la resoluci@'on energetica
@item
Otros procedimientos
@end enumerate

@item
Separaci@'on Gamma-Hadr@'on en MAGIC
@enumerate a
@item
An@'alisis de momentos 
@item
Factores de calidad
@itemize @bullet
@item
C@'alculo de cortes @'optimos
@item
Metodo de las distribuciones acumuladas
@item
Metodo de la entrop@'ia
@item 
Utilizaci@'on de metodos de L@'ogica Difusa (@i{fuzzy logic}) para
la separaci@'on gamma-hadr@'on
@end itemize
@item
Otros metodos
@end enumerate

@item
Conclusiones 

@end enumerate

@enumerate A
@item
Descripci@'on del C@'odigo CORSIKA
@item
Obtenci@'on de los par@'ametros de imagen
@item
Analisis estad@'istico de correlaciones
@item
Descripcion tecnica de los programas de an@'alisis
@end enumerate

@bye


