%%%%%%%%%%%%%%%%%%%%%%%%%%%%%%%%%%%%%%%%%%%%%%%%%%%%%%%%%%%%%%%%%%%%%%%%%%%
%%
%%  ch-cosmicrad.tex
%%
%%  Created: Fri Oct 10 14:24:37 1997
%%  Author.: Jose Carlos Gonzalez
%%  Notes..:
%%          
%%-------------------------------------------------------------------------
%% Filename: $RCSfile$
%% Revision: $Revision$
%% Date:     $Date$
%%%%%%%%%%%%%%%%%%%%%%%%%%%%%%%%%%%%%%%%%%%%%%%%%%%%%%%%%%%%%%%%%%%%%%%%%%%


\chapter{The Cosmic Radiation}
\label{chapter:cosmicrad}
\let\rightmark\leftmark

%\section{The Cosmic Radiation}
\label{sec:cosmicradiation}

The \emph{\I{cosmic radiation}} (\I{CRAD})\footnote{In this work we
  talk about \emph{\I{cosmic radiation}}, and use the abbreviation
  CRAD, when we refer to both particles and light quanta (photons). We
  will use the terms \emph{\I{cosmic rays}}, abbreviated CR, and
  \emph{\I{gamma rays}} or \emph{\I{gamma radiation}}, abbreviated GR
  and GRAD (or simply $\gamma$'s) when we refer exclusively to
  particles (matter) or photons.}  is a bath or background of
energetic ions and photons flying all over the Universe, from their
creation in high energy astrophysical scenarios until the interaction
with other bodies, with other cosmic particles or their arrival to the
Earth.

In the \emph{\I{cosmic rays}} (CR), for energies around $1\u{TeV}$,
about 70\% (in flux) of these particles are protons, 25\% $\alpha$-particles
(He nuclei), and the rest heavier nuclei.  In energy density contained
in the particles (namely in energy per nucleon), however, about 90\% of
the total energy is in the protons and 9\% in the $\alpha$-particles.
Nevertheless, the \Isee{chemical}{composition} \I{composition} of the
CR varies with the energy of the particles.  The \I{rate of arrival}
to the Earth is approximately 1\,000 particles per square meter per
second, and the \I{energy spectrum} of the \emph{\I{cosmic rays}}
extends till about\footnote{About units and orders of magnitude, see
  appendix \fullref{appendix:units}.}  $100\u{EeV}$
($10\pow{20}\u{eV}$).  So huge energies can only be reached in
astrophysical scenarios comprising very intense gravitational and
electromagnetic fields. We divide the different sources of CR into two
mayor groups, namely the point-like sources and the extended sources.

%\section{Origin and Sources of Cosmic and Gamma Rays}
%\label{sec:sources}

%\section{Study of High-Energy Gamma-ray Astronomy}
%\label{sec:hegra}

%Until the early 1950s, where the first high-energy accelerators
%appeared, experiments in high energy where conducted with cosmic rays.
%The muon, pion and kaon where all discovered in cosmic-ray
%experiments.

Historically, one of the main motivations for the study of cosmic and
gamma rays has been the tremendous energy they can reach in
astrophysical scenarios. This allows scientists to use them to study
particle interactions at energies above those available at terrestrial
accelerators.

These detected high energies led to the first efforts in $\gamma$-ray
astronomy. The observed non-thermal photon spectra allows us to test
possible acceleration mechanisms in exotic sources such us active
galactic nuclei and supernov{\ae}. The $\gamma$-rays themselves show us
where to look for Fermi acceleration (see below) of charged particles,
and tell us about the intervening intergalactic medium, giving us
clues about the early Universe.


\section{Acceleration mechanisms}
\label{sec:accel}

In the late 1940s, Enrico Fermi postulated that plasma clouds in the
interstellar medium might act as magnetic mirrors and reflect charged
cosmic rays \cite{XXX}.  The cosmic ray will lose or gain energy,
depending on whether the cloud is going away from or moving towards the
particle, respectively (there will be more collisions particle-cloud
if they are moving toward each other).  This is known as
\emph{second-order Fermi acceleration}.  This is a very slow mechanism
though, and due to the presence of multiple energy-loss processes, it
can not account for the high-energy cosmic rays observed.

This idea was applied to a \emph{shock front}.  Shocks are believed to
be very common in the Universe.  In the rest frame of the shock,
material both upstream and downstream of the shock moves toward the
shock.  Therefore, a charged particle trapped between the upstream and
downstream regions will gain energy every time it crosses the shock
front, as it is always reflected by an approaching wall.  This
mechanism is the so called \emph{first-order Fermi acceleration}
\cite{XXX}.  The maximum attainable energy is determined by the
lifetime of the shock.
 
Electromagnetic mechanisms may also be responsible for the
acceleration of charged cosmic rays.  A rotating magnetised neutron
star, or accretion disks (with matter mainly in the form of highly
conductive plasma) can give rise to very strong electric fields, that
can accelerate these particles up to high energies\footnote{Neutron
  stars, pulsars, the mentioned shock fronts and even accretion disks
  are all objects related to the last episodes in the life of a star
  or a cluster of stars.}.  In
%  or a cluster of stars (see Fig.\ref{fig:bhstarevol}).}.  In
particular, the conductive matter in accretion disks have a
\emph{differential rotation}, and this produces the effect of an
electric generator --- the accretion disk around a neutron star could
rise the energy of charged particles up to $10^{17}\u{eV}$ \cite{XXX}.

Energetic protons and electrons in the vicinity of astrophysical
objects produce high-energy photons.  The electrons generate low-energy
photons through synchrotron radiation in magnetic fields, which can be
boosted to \u{TeV} energies by inverse Compton scattering with other
high-energy electrons.  On the other hand, protons and nuclei produce
high-energy pions through strong interactions with the surrounding
matter. The neutral pions will decay into photons, which can escape
from the source and give us hints about the original acceleration
process.

\section{Sources of cosmic and gamma rays}
\label{sec:sources}

In 1910, Victor Hess discovered, in a series of balloon flights, that
the rate of ions produced in the atmosphere per unit volume increased
with altitude \cite{XXX}. This was the first experimental evidence of
the cosmic rays. In 1938, Pierre Auger discovered extensive air
showers: he found that the radiation reaching the ground was
correlated over large distances (up to 300\u{m}) and short time scales
(around $1\,\mu\text{s}$) \cite{XXX}.  Nowadays, several experiments all
around the world measure the spectrum, abundances, flux and age of
cosmic rays.

Where do they come from? For charged particles, we cannot determine
the source of individual particles, since the magnetic field of our
own galaxy bends their trajectories, and therefore they lose all
directional information before reaching the Earth.  However,
$\gamma$-rays have no charge, and hence do not bend in the magnetic
fields. Therefore, we can actually make observations on direct or
indirect sources of $\gamma$-rays.

Two types of objects are the main detected sources of $\gamma$-rays:
pulsars and active galactic nuclei. Both types contain compact
objects, neutron stars and black holes, respectively. X-ray binary
systems were thought to be also the origin of $\u{TeV}$ and $\u{PeV}$
$\gamma$-rays, but no observation up to now has confirm these claims.
Finally, there are also some speculative objects, such as primordial
black holes and cosmic strings, that might produce $\u{TeV}$ photons.

\subsubsection{Active Galactic Nuclei}
\label{sec:agn}

After several decades of casuistry studies and proliferation of
morphological classes and subclasses of active galactic nuclei (AGN)
--- quasars, radio-quiet and radio-loud galaxies, blazars, etc. (see
Fig.  \ref{fig:AGNclassification}), a well accepted unified theory of
the various classes emerged, the \emph{unified AGN model}. The basic
elements of this model are (see Fig. \ref{agnmodel:fig}):

\AGNclassificationfig

\begin{itemize}
\item A central engine, a \emph{supermassive black hole} with mass
  $M\sim 10^{7} - 10^{10} M_\odot$, and corresponding Schwarzschild
  radius of $r_{\mathrm{S}} \sim 10^{-6} - 10^{-3}\u{pc} \sim 44 -
  44000 R_\odot$,

\item a \emph{thin accretion disk} around  the black hole, surrounded by

\item a \emph{thick torus} standing on the equatorial plane;
  
\item collimated \emph{jets} perpendicular to the accretion disk
  appear in the radio-loud AGN;
  
\item finally, there would be \emph{clouds} around the black hole,
  responsible for the observed emission lines.
\end{itemize}

\agnartisticviewfig

The distinct parts of the AGN emit radiation in different ranges of
the electromagnetic spectrum, from radio waves (jets), through
infrared and X-rays (accretion disk and torus), to TeV $\gamma$-rays
(also from the jets).  In this model, there are two main effects of
the viewing angle with respect to the rotation axis: shadowing effects
of the torus, and Doppler boosting of a relativistic jet observed
under a small angle.  Relativistic flows under small angles produce
observation of \emph{superluminal motion} (motion with apparent
velocity greater than the speed of light).  The relativistic beaming
also causes the apparent luminosity to be enhanced by the so called
\emph{Doppler factor}:
%
\dopplerfactoreq
%
where $\beta = v/c$, $\Gamma=1/\sqrt{\left(1-\beta^2\right)}$, c is
the speed of light and $\theta$ the viewing angle with respect to the
jet axis. The ratio between the observed and the intrinsic luminosity
is:
%
\ratiolumeq
%
with $\alpha$ the differential spectral index of the source (see
Eq.\eqref{eq:specindex}). Figure \ref{fig:dopplerboost} shows how
intense is this boosting effect for a jet-like source, for different
viewing angles.  This angle must be smaller than $\sim 10\deg$ in
order to observe \u{TeV} $\gamma$-rays.  This angle means a solid
angle of $\sim 1/132$ of the entire solid angle of the sphere. 

\agninterpretationfig
%
This model leads to an interpretation of the different observational
properties of the AGN (wavelength range of emission, presence/absence
and type of emission lines, time scale of variability, etc.) and,
consequently, the morphological classification in terms of the viewing
angle $\theta$, as summarised in Fig.  \ref{fig:agninterpretation}.
For radio-loud AGN, the angle is defined from the jet axis, whereas
for radio-quiet, with no jets, there is still a preferred direction
defined by the rotation axis.

There are two mechanisms that seem to be responsible for the observed
high energy emission, the inverse Compton process and the
proton-initiated cascade (PIC). The first mechanism is used in the
synchrotron self-Compton (SSC), external Compton (EC) and inhomogeneous
models. In all of them the electrons are accelerated in the jets, and
upscatter photons to high energies. In the SSC model, the electrons
are themselves the source of photons (via synchrotron radiation); in
the EC model the photons are located outside the jet and do not come
from synchrotron processes; in the inhomogeneous model, different
energies originates in different regions of the jet. In these models
the maximum $\gamma$-ray energy is limited to $\sim 10\u{TeV}$.
Finally, the PIC assume shock-accelerations of protons to ultra high
energies, $\sim 10^{10}\u{GeV}$, which interact with ambient photons
producing neutral pions that decay and initiate electromagnetic
cascades.

\dopplerboostfig

\subsubsection{Pulsars}
\label{sec:pulsars}

Pulsars are rapidly rotating neutron stars with strong magnetic fields
($\sim 10^{12}\u{G}$). There seem to be two different groups of pulsars,
namely those with period of a few milliseconds, and those with periods
of the order of seconds.  Neutron stars can have a mass up to $\sim 3
M_\odot$ within a radius of around $10\u{km}$, being the densest stable
form of matter known.

\pulsarschfig
%
The satellite \I{SAS II} (\emph{small astronomical satellite}),
launched on November 15$^{\mathrm{th}}$, 1\,972, registered a great
amount of data on the galactic gamma emission. Among the mission's
results, there were found two peaks in flux intensity (in gamma rays)
in the emission coming from two well known pulsars: \I{Vela}
(\I{PSR\,0833-45}) and the \I{Crab pulsar} (\I{PSR\,0531+21}). The
identification of these peaks was confirmed when it was found a
pulsated emission with similar periods to those already detected in
radio.

In the last times, another pulsar has been confirmed as a gamma rays
source: \I{PSR\,1706-44}. Some authors believe that a supernova
remnant is associated to this pulsar, but this is still questionable.
What is well known is that this pulsar has associated a X-rays nebula.

Vela and the Crab pulsar are two very different pulsars. The Crab
pulsar has light curves that are very similar in all the
electromagnetic spectrum, from radio to some GeV. One can also see, in
these light curves, the characteristic \I{main pulse}, with a less
important interpulse. On the contrary, the light curve of Vela
corresponding to radio emission has no \I{interpulse}. Also, the phase
difference $\Phi(\mathrm{pulse})$--$\Phi(\mathrm{interpulse})$ are different
in gamma rays and in the visible. This is a consequence of the
different mechanisms involved in each pulsar.

Up to now, there have been detection of \u{TeV} $\gamma$-rays only from a
class of pulsars known as \emph{plerions} --- a supernova remnant with
a filled morphology ---, and only unpulsed \u{TeV} emission has been
observed.

Several models try to explain the generation of gamma rays in pulsars.
In the \emph{\I{polar cap model}} the $\gamma$ photons are produced
when charged particles are accelerated near the poles. This model
predict a maximal energy for the $\gamma$-rays of $E=6-20\u{GeV}$.
When the $\gamma$ photons are produced in the border of the so-called
\emph{light cylinder}\footnote{The \emph{light cylinder} is defined to
  be the region near the pulsar where the velocity of the lines of the
  magnetic field in corrotation with the pulsar reach c, the speed of
  light in vacuum}, where there can be discontinuities in the magnetic
field, the energy can go up to $100\u{GeV}$ (as predicted by the
\emph{\I{outer cap model}}). Other models use the \emph{\I{inverse
    Compton effect}}. X-ray photons could be injected into the gamma
rays domain by means of collisions with energetic electrons. In any
case, it's very likely that the emissions in different energy ranges
come from distinct scenarios.

Everything said above is concerned to pulsars with known emission in
radio (\emph{radio-pulsars}). There exist also pulsars without this
radio-emission, the so-called \emph{radio-quiet pulsars}. In this
group, only one has been detected to be a gamma-rays source. This is
the case of \I{Geminga}. This discrete point-like source of $\gamma$-rays
has shown since the beginning a strange behaviour, for there was no
detected counterpart in any other energetic range.  Nowadays, however,
it is almost clear that Geminga is a binary system formed by a neutron
star in the orbit of another bigger neutron star in fast rotation.

%\afterpage
%\clearpage

\subsubsection{X-ray binaries}
\label{sec:xbinaries}

In these systems, a neutron star and a main-sequence star orbit around
each other.  The distance between the two stars is small enough for
the more massive star to accrete matter from the companion. This
matter falls inward in a spiral orbit, until the magnetic forces
dominate: then the matter flows along the field lines down to the
magnetic poles of the neutron star, where two \emph{hot-spots} (with
temperatures of the order of $T\simeq 10-15\u{keV}$) are formed. Some of
these systems show jet-like structures. This fact leads to the
hypothesis that one could observe high-energy $\gamma$-rays. Actually, in
the 1980s there were some reports claiming the detection of these
signals. As a matter of fact, there is no confirmation of these
detections from the modern instruments.

\subsubsection{Primordial black holes}
\label{sec:primbh}

Black holes are objects so collapsed that their escape velocity exceeds
the speed of light, $c$. The surface surrounding the black hole where
the escape velocity equals $c$ is called the \emph{event horizon}.
Actually, black holes are not really black: pairs particle-antiparticle
are formed in the surface of their event horizon. This leads to an
observable flux of particles which seem to come out from the event
horizon, with a black-body spectrum. This is the \emph{Hawking
radiation}, which depends inversely on the black hole's mass: the more
massive the black hole, the slower is the evaporation.  Primordial black
holes are mini black holes, weighing only $\sim 10^{11}\u{kg}$ and with
radii of $\sim 10^{-10}\u{m}$, presumably formed in the highly turbulent
conditions existing after the Big Bang. Their extremely small size makes
them evaporate in the lifetime of the Universe (black holes with the
mentioned mass would be evaporating now), with a final burst of
$\gamma$-rays and microwaves that should be detectable, but has not yet
been found. 

\crspectrumfig

\subsubsection{Gamma-ray bursts}
\label{sec:grbs}

Gamma-ray bursts are short duration ($1\u{ms}$ to tens of seconds),
intense flashes of hard X-rays and $\gamma$-rays, that come from
apparently random locations in the sky.  They were discovered by US Air
Force satellites in 1967 (which were looking for nuclear tests over the
Soviet Union), but not de-classified until 1973. There appear sharp
temporal features in their time profile, which eases its study. EGRET,
the high energy gamma-ray detector on board the CGRO, and recently
Beppo-Sax, a new GRBs detector, have observed many GRBs. This later
satellite, with its wide-field camera, allows to localise the burst
position. This has led to the discovery of a soft X-ray afterglow that
lasts for days after the initial burst, as well as to a search for
X-ray, optical and radio counterparts to these events. Several of these
counterparts have been found, all consistent with the model of a
cosmological, relativistic expanding \emph{fireball} (perhaps the
aftermath of the merger of a binary neutron star system). 

\section{Spectrum and Chemical Composition of the Cosmic Radiation}
\label{sec:crspectrum}

The energy spectrum is determined by the creation and acceleration
mechanisms in the source. Very high energy particles require
non-thermal scenarios, and this forces a power-law energy spectrum:
%
\powerlaweq
%
For cosmic rays there exist direct (from balloons and satellites) and
indirect measurements (from detectors on ground).  Figure
\ref{fig:crspectrum} shows the all-particle spectrum as measured from
different experiments.  Some of these experiments can discriminate the
type of particle, and a different spectral index and abundance can be
set for different nuclei, as shown in Table \ref{table:crindex}. 

\CRfluxindextable

We already mentioned that for energies around $1\u{TeV}$, around 70\%
of the cosmic rays are protons, 25\% $\alpha$-particles (He nuclei),
and the rest heavier nuclei. This abundances have been measured by
several experiments, and compared to the nuclear abundances in the
Solar System (see Fig.\ref{fig:abund}). There exist a general
agreement, specially in the even-odd structure, but there exist a
deficit in H and He, and a clear over-abundance in lithium, beryllium
and boron (most probably produced by nuclear collisions of heavier
elements with the interstellar matter).

\abundfig

\section{Propagation through the interstellar and
  intergalactic media}

The cosmic rays with electric charge, when flying from their sources
across the interstellar and intergalactic media, feel the presence of
the electric and magnetic fields, resulting in a bending of their
trajectories.  Due to this, it is impossible to determine
observationally whether an astrophysical object is a source of cosmic
rays or not.

For $\gamma$-rays, the different backgrounds of photons in the
Universe interact with them, via the channel $\gamma\gamma \to
e^+e^-$.  This reaction has a maximum of probability slightly above
the reaction threshold, at about $E_{\gamma_1}E_{\gamma_2} = 2 m_e^2
\simeq 0.5\E{12}\u{eV}^2$, and thus it sets limits in the distance
from which the $\gamma$-rays can reach us.  For diffuse
infrared/optical photons (generated in the stars and in the heating of
clouds od dust) and the microwave background radiation, this reaction
means an effective absorption of $\gamma$-rays at energies of the
order of \u{TeV}s and \u{PeV}s, respectively. In
Fig.\ref{fig:meanfreepath} one can see the pair-production mean free
path as a function of the $\gamma$-ray energy. We see that the optical
depth for $200\u{TeV}$ is about $1\u{Mpc}$, thus only $\gamma$-rays
from our galaxy or its vicinity can be observed. For protons, the
corresponding \emph{cut-off} (for the reaction $p\gamma \to N^*$)
occurs at much higher energies.

The determination of the precise curves of absorption of the cosmic
and $\gamma$-rays due to these background of photons can provide
information about many topics in Astrophysics, like the age of
formation of stars and its correlation with the era of galaxy
formation, the discrimination between different possible models of
dark matter, and the dynamics of the whole Universe.


%\bhstarevolfig

%\elemespectrofig

\endinput

%
%% Local Variables:
%% mode:latex
%% TeX-master: t
%% End:

%%EOF
