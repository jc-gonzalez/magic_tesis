%%%%%%%%%%%%%%%%%%%%%%%%%%%%%%%%%%%%%%%%%%%%%%%%%%%%%%%%%%%%%%%%%%%%%%%%%%%
%%
%%  ch-cosmicrad.tex
%%
%%  Created: Fri Oct 10 14:24:37 1997
%%  Author.: Jose Carlos Gonzalez
%%  Notes..:
%%          
%%-------------------------------------------------------------------------
%% Filename: $RCSfile$
%% Revision: $Revision$
%% Date:     $Date$
%%%%%%%%%%%%%%%%%%%%%%%%%%%%%%%%%%%%%%%%%%%%%%%%%%%%%%%%%%%%%%%%%%%%%%%%%%%


\chapter{The Cosmic Radiation}
\label{chapter:cosmicrad}
\let\rightmark\leftmark

\section{The Cosmic Radiation}

The \emph{\I{cosmic radiation}} (\I{CRAD})\footnote{In this work we
  talk about \emph{\I{cosmic radiation}}, and use the abbreviation
  CRAD, when we refer to both particles and light quanta (photons). We
  will use the terms \emph{\I{cosmic rays}}, abbreviated CR, and
  \emph{\I{gamma rays}} or \emph{\I{gamma radiation}}, abbreviated GR
  and GRAD (or simply $\gamma$'s) when we refer exclusively to
  particles (matter) or photons.}  is a bath or background of
energetic ions and photons flying all over the Universe, from their
creation in high energy astrophysical scenarios until the interaction
with other bodies, with other cosmic particles or their arrival to the
Earth.

In the \emph{\I{cosmic rays}} (CR), about 90\% of these particles are
protons, 9\% $\alpha$-particles (He nuclei), and the rest heavier
nuclei. Nevertheless, the \Isee{chemical}{composition} \I{composition}
of the CR may vary on the energy of the particles. The \I{rate of
  arrival} to the Earth is approximately 1\,000\u{s^{-1}m^2} and the
\I{energy spectrum} of the \emph{\I{cosmic rays}} extends till about
100\u{EeV} (10\pow{20}\u{eV})\footnote{About units and orders of
  magnitude, see appendix \fullref{appendix:units}.}. So huge energies
can only be reached in astrophysical scenarios comprising
gravitational and electromagnetic fields very intense. We divide the
different sources of CR into two mayor groups, namely the point-like
sources and the extended sources.


\MORE%%%%%%%%%%%%%%%%%%%%%%%%%%%%%%%%%%%%%%%%%%%%%%%%%%%%%%%%%%%%

\section{Physics of the Cosmic Rays}

\subsection{Production of Cosmic Rays}

\subsubsection{Charged Cosmic Rays}

\subsubsection{Gamma Rays}

\subsubsection{Neutrinos}

\subsubsection{Cosmic Rays' Spectrum}

\subsubsection{Chemical Composition of the Cosmic Radiation}

\subsection{Cosmic Rays Sources}

Most of the efforts spent in the field of the \emph{\I{Gamma Ray
Astronomy}} has been done in the search and study of new sources of
CRAD. Some sources have been already identified, but the number of
sources decrease when we increase the energy of the observed
particles. Let's make a brief review of the main gamma and cosmic rays
sources.

\subsubsection{\I{Pulsars}}

The satellite \I{SAS II} (\emph{small astronomical satellite}),
launched on November 15$^{\mathrm{th}}$, 1\,972, registered a great
amount of data on the galactic gamma emission. Among the mission's
results, there were found two peaks in flux intensity (in gamma rays)
in the emission coming from two well known pulsars: \I{Vela}
(\I{PSR\,0833-45}) and the \I{Crab pulsar} (\I{PSR\,0531+21}). The
identification of these peaks was confirmed when it was found a
pulsated emission with similar periods to those already detected in
radio.

In the last time, another pulsar has been confirmed as a gamma rays
source: \I{PSR\,1706-44}. Some authors believe that a supernova
remnant is asociated to this pulsar, but this is still
questionable. What is well known is that this pulsar has associated a
X-rays nebula.

Vela and the Crab pulsar are two very different pulsars. The Crab
pulsar has light curves that are very similar in all the
electromagnetic specrtum, from radio to some GeV. One can also see, in
these light curves, the caracteristic \I{main pulse}, with a less
important interpulse. On the contrary, the light curve of Vela
corresponding to radio emission has no \I{interpulse}. Also, the phase
difference $\Phi(\mathrm{pulse})$--$\Phi(\mathrm{interpulse})$ are
different in gamma rays and in the visible. This is a consequence of
the different mechanisms involved in each pulsar.

\pulsarschfig

Several models try to explain the generation of gamma rays in pulsars.
In the \emph{\I{polar cap model}} the $\gamma$ photons are produced
when charged particles are accelerated near the poles. This model
predict a maximal energy for the $\gamma$-rays of
$E=6-20\u{GeV}$. When the $\gamma$ photons are produced in the border
of the so-called \emph{light cylinder}\footnote{The \emph{light
cylinder} is defined to be the region near the pulsar where the
velocity of the lines of the magnetic field in corrotation with the
pulsar reach c, the speed of the light in vacuum}, where there can be
discontinuities in the magnetic field, the energy can go up to
$100\u{GeV}$ (as predicted by the \emph{\I{outer cap model}}). Other
models use the \emph{\I{inverse Compton effect}}. X-ray photons could
be inyected into the gamma rays domain by means of collisions with
energetic electrons. In any case, it's very likely that the emissions
in different energy ranges come from distinct scenarios.

All we have said up to now is concerned to pulsars with known emission
in radio (\emph{radio-pulsars}). There exist also pulsars without this
radio-emission, the so-called \emph{radio-quiet pulsars}. In this
group, only one has been detected to be a gamma-rays source. This is
the case of \I{Geminga}. This discrete point-like source of
$\gamma$-rays has shown since the beginning a strange behaviour, for
there was no detected counterpart in any other energetic range.
Nowadays, however, it is almost clear that Geminga is a binary system
formed by a neutron star in the orbit of another bigger neutron star
in fast rotation.

\subsubsection{Supernov\ae}

After big efforts, only three supernova remnants appear to be
$\gamma$-ray sources: the \I{Crab nebula}, \I{Vela} and \I{SN\,1006},
who was discovered in 1,997. Although a big amount of observation time
is devoted to supernova remnants, still the results are not enough to
explain most of the questions related to the nature of the cosmic
radiation.

\subsubsection{Active Galactic Nuclei (AGNs)}

\subsubsection{Quasars}

\subsubsection{Gamma Ray Bursts}

\subsubsection{Extended Sources}

\subsection{Propagation through the interestelar and
intergalactic media}

\gsourcesfig

\gsnumbersfig

\bhstarevolfig

\subsubsection{Diffusion in the formation regions}

\subsubsection{Absorption in the intergalactic medium}

\subsubsection{Absorption in the interestelar medium}

\subsection{The Cosmic Radiation and its entrance in the
Earth atmosphere}

\elemespectrofig

\subsubsection{Production of atmospheric showers}

\subsubsection{Front of Particles}

\subsubsection{Front of \Cherenkov light}

\subsubsection{Observables in atmospheric showers}

\endinput


%
%% Local Variables:
%% mode:latex
%% End:

%%EOF
