%%%%%%%%%%%%%%%%%%%%%%%%%%%%%%%%%%%%%%%%%%%%%%%%%%%%%%%%%%%%%%%%%%%%%%%%%%%
%%
%%  ch-simshowers.tex
%%
%%  Created: Fri Oct 10 14:24:37 1997
%%  Author.: Jose Carlos Gonzalez
%%  Notes..:
%%          
%%-------------------------------------------------------------------------
%% Filename: $RCSfile$
%% Revision: $Revision$
%% Date:     $Date$
%%%%%%%%%%%%%%%%%%%%%%%%%%%%%%%%%%%%%%%%%%%%%%%%%%%%%%%%%%%%%%%%%%%%%%%%%%%

\chapter{Simulation of atmospheric showers}
\label{chapter:simshowers}
%
Usually, the \MC simulations of atmospheric showers follow a well
defined scheme. They start the simulation at the top of the
atmosphere, with a given primary particle. From there they simulate
the fly and interactions of this particle with its environment. At a
certain moment, this particle produces some secondary particles, which
are written to a stack, being processed each of them afterwards. Each
secondary particle have a certain probability of interact or decay,
and in virtue of this probability new (tertiary) particles are
produced and written into a new stack. The process follows until a
detector (or ground) is reached. See Fig. \ref{fig:MCsimGaisser} as an
example.

\mcsimsample

Several cuts (in energy, for example) or simplifications are normally
taken into account in the simulation programs in order to minimize the
computation time. The simulation code will reject, during the
simulation, those particles which fall into a ``dead zone'' of the
phase space of the simulation parameters, or a ``blind region'' of the
detector.  For instance, if we are interested in detecting electrons,
and our instrument do only detect those above a sharp energy cut-off
$E_{\mathrm{threshold}}$, we are interested quite likely in removing
from the simulation process any electron as soon as his energy falls
down beyond this energy.

%%==========================================================
\section{The \CORSIKA code}
\label{sec:CORSIKA}
%
One of the most spreadly used simulation codes in the field of cosmic
and $\gamma$-rays is the \CORSIKA code \cite{CORSIKA:manual}. The last
official version released at the time of writing this work is
\lastCORSIKA.  \CORSIKA (COsmic Ray SImulations for KAscade) is a
detailed Monte Carlo program to study the evolution and properties of
extensive air showers in the atmosphere, and was developed to perform
simulations for the KASKADE experiment, in Karlsruhe (Germany)
\cite{KASKADE} (this experiment aims to measure the elemental
composition of the primary cosmic radiation in the energy range
$3\E{14}$--$5\E{16}$\u{eV}). The code is mostly written in Fortran 77
(some pieces of code in new versions are written in C for portability)
and hence is easily portable to almost any computer.

\CORSIKA simulates interactions and decays of nuclei and subatomic
particles (hadrons, muons, electrons, photons, \ldots) up to energies
of 10\pow{17}\u{eV}. All secondary particles are tracked along their
trajectories and their parameters when reaching the user-defined
detector levels (energy, position, direction and arrival time) are
saved into disk.

\CORSIKA was originally developed on the basis of three well
established program systems. The first treats the hadronic part of
proton induced showers using the \emph{fire ball} and \emph{isobar},
and was written in the 70's by P.K.F.Grieder \cite{Grieder:1979}.
Nowadays, its structure serves as the general frame where \CORSIKA
handles its input and output, and the interaction routines used in
this program can be used as hadronic interactions at low energies. The
second program was developed by J.N.Capdevielle
\cite{Capdevielle:1989} following the \emph{Dual Parton Model}
(DPM)\cite{Capella:1980}, describing the interactions of protons at
high energies in good agreement with the measured collider data. The
third part, used for the simulation of the electromagnetic part of the
air shower, is based in the EGS4 code \cite{EGS4}. Whenever possible,
experimentally accesible data have been used.

For the simulation of the strong interactions several models can be
used (they are chosed by the user). The current available models are:

\begin{itemize}
  
\item Strong interactions at low energies:

\begin{description}
\item[Isobar \& Fire-ball] These model routines where developed
  \cite{Grieder:1979} for low energy interactions. They work at center
  of mass energies between 0.3\u{GeV} and 10\u{GeV}. In this model the
  hadron-nucleus collisions are approximated by hadron-nucleon
  reactions. The non-interacting nucleons of the target are neglected.
  The hadron-nucleon reactions are assumed to take place via several
  intermediate states which are decaying immediately into up to 5
  secondaries.
  
\item[GHEISHA] It's a more sophisticated replacement of the Isobar and
  Fire-ball routines. The GHEISHA routines are designed for laboratory
  energies up to some hundred \u{GeV}. \CORSIKA uses them only at
  laboratory energies below 80\u{GeV} to treat the interactions of
  hadronic projectiles with the nuclei of the atmosphere.
\end{description}

\item Strong interactions at high energies:

\begin{description}
\item[VENUS] The VENUS code (Very Energetic NUclear Scattering) is
  mainly designed to treat nucleon-nucleon, nucleon-nucleus, and
  nucleus-nucleus scattering at ultrarelativistic energies. It
  simulates the inelastic ultra-relativistic heavy ion collisions with
  detailed simulation of creation, interaction and fragmentation of
  colour strings. It's the alternative to the original DPM.
  
\item[QGSJET] The Quark Gluon String model with JETs is an
  extension of the QGS model developed in the 80's, which describes
  hadronic interactions on the basis of exchanging supercritical
  Pomerons, and includes minijets to describe the hard interactions
  which are important at the highest energies.
  
\item[DPMJET] The Dual Parton Model with JETs (Version II.4) is based
  on the two component Dual Parton Model\cite{Capella:1980}, and
  contains multiple soft chains as well as multiple minijets. It
  relies on the same theory that VENUS and QGSJET do.
  
\item[SIBYLL] SIBYLL (Version 1.6) is a minijet model essentially
  designed for usage in EAS Monte Carlo programs. For hadronic soft
  collisions both projectile and target particles fragment into a
  quark-diquark respect to the quark-antiquark system, that forms a
  triplet and anti-triplet of colour.
  
\item[HDPM] As an alternative model of the interactions between
  hadrons and nuclei at high energies, a phenomenological generator
  called HDPM may be used. For historical reasons it is the default in
  \CORSIKA.  It is developed by Capdevielle \cite{Capdevielle:1989}
  and inspired by the Dual Parton Model\cite{Capella:1980}.
\end{description}

\end{itemize}

The photoproduction of muon pairs and hadrons is added to the
electromagnetic part to allow the calculation of the muon content of
photon induced showers. A second way of treating the electromagnetic
component is to use the corrected and adapted form of the analytic NKG
formula for each electron or photon produced in the cascade.

It is also possible to include the generation of \Cherenkov light in
the atmosphere, to handle electronic and muonic neutrinos and
anti-neutrinos, and to simulate showers under nearly horizontal
incident directions (without the possibility of \Cherenkov light
generation, though).

We used a modified version of \CORSIKA 4.50, suitable for our purposes.
We modified the detector geometry in order to allow the simulation of
collection of light by \Cherenkov telescopes. Several minor
modifications were also introduced in order to make easier the
handling of the output data.

\subsection{Input and output from \CORSIKA}
%
The input for a \CORSIKA run is a plain ASCII parameters file, with
several entries in the form ``\texttt{KEYWORD <par1> <par2>} \ldots'',
where \texttt{<parN>} means the N-th parameter for that keyword
\texttt{KEYWORD}. An example of a parameters file is shown in Fig.
\ref{fig:corinput}.

\CORSIKAinputfig

\afterpage{\clearpage}

\CORSIKAGeanttable

Among other parameters, the user can set the type of primary for the
showers (\texttt{PRMPAR}), the run number (\texttt{RUNNR}), the number
of event for this run (\texttt{NSHOW}), the energy range
(\texttt{ERANGE}), the simulated slope of the spectrum in this energy
range (\texttt{ESLOPE}), the range of $(\theta,\phi)$ angles
(\texttt{THETAP}, \texttt{PHIP}), the observation levels
(\texttt{OBSLEV}), the cuts in energy (\texttt{ECUTS}) which allow us
to reduce the computation time by rejecting those particles below a
certain threshold, and the geometry of the detector (\texttt{CERARY}
in the original version, \texttt{ERANGE} and \texttt{ERANGE} in our
modified version).

\CORSIKA is able to treat the particles that are listed in Table
\ref{table:CORSIKAGeantcodes}

The output of \CORSIKA comes in two different ways. First, we have an
ASCII output, with statistics and sumaries of the simulation of the
showers and even with debugging information if needed.  The real
output of \CORSIKA is given in binary format. For each particle hitting
the detector area in our pre-defined observation levels we get the
type of the particle, the position in the plane of that observation
level, the direction, the energy, the height of production and the
arrival time. If we selected the CERENKOV version of \CORSIKA, we'll be
able to get information about the \Cherenkov photons as well. Due to
the huge number of \Cherenkov photons produces at high energies, we
can generate and follow them in groups (\emph{bunches}).  However, for
our purposes we traced each single \Cherenkov photon. For these
photons we stored into disk the wavelength, the position, the
direction, height of production and arrival time.

The output or the currently used \CORSIKA code is structured as shown
in Fig. \ref{fig:corsikastruc}. For each run with $N$ showers we will
end up with $3 N+2$ binary data files, with information about the
run and each individual shower. These files are (in the following list
\texttt{XXXXXX} means a 6-digit number):

\CORSIKAstructfig

\begin{itemize}
  
\item A \emph{Run Header file}, \texttt{runXXXXXX}, with information
  about the whole run (see Table \ref{tab:rh}). It's the first data
  file generated.
  
\item A \emph{Run End file}, \texttt{endXXXXXX}, with information
  about the whole run and some statistics about the run just generated
  (see Table \ref{tab:re}). It's the last file generated. It's sole
  presence indicates that the run has been finished.
  
\item One \emph{Particle data file} per shower, \texttt{datXXXXXX},
  with information about the particles (not Cherenkov photons)
  generated in the shower and which arrived at the ground (actually,
  at any user defined obs. level), in the user-defined area (see Table
  \ref{tab:part}).
  
\item One \emph{Cherenkov data file} per shower, \texttt{cerXXXXXX},
  with information about the Cherenkov photons generated in the shower
  and which arrived at the ground (actually, at any user defined obs.
  level), in the user-defined area (see Table \ref{tab:cher}).
  
\item One \emph{Statistics data file} per shower, \texttt{staXXXXXX},
  with information longitudinal profiles for different showers and
  other parameters of the shower (see Table \ref{tab:sta}).

\end{itemize}

As you can see in Fig.\ref{fig:corsikastruc}, each Particle and
Cherenkov file has an \emph{Event Header}, a list of \emph{Particle}
or \emph{Cherenkov photon} data blocks, and an \emph{Event End}. The
information stored in the \emph{Event Header} and \emph{Event End}
data blocks are shown in Tables \ref{tab:eh1}-\ref{tab:eh2} and
\ref{tab:ee}.

The structure of each \emph{Particle} or \emph{Cherenkov photon}
sub-blocks is shown in Tables \ref{tab:part}-\ref{tab:cher}.

\CORSIKAtableRH

\CORSIKAtableRH

\CORSIKAtableEHone

\CORSIKAtableEHtwo

\CORSIKAtableEE

\CORSIKAtableRE

\CORSIKAtablePART

\CORSIKAtableCHER

\CORSIKAtableSTA


\MORE%%%%%%%%%%%%%%%%%%%%%%%%%%%%%%%%%%%%%%%%%%%%%%%%%%%%%%%%%%%%

%%==========================================================
\section{Simulation of showers at High Zenith Angle}

%%==========================================================
\section{Information available from the showers}

%%==========================================================
\section{Time Structure of the \Cherenkov pulse from an atmospheric shower}

%%%%%%%%%%%%%%%%%%%%%%%%%%%%%%%%%%%%%%%%%%%%%%%%%%%%%%%%%%
\section{Other simulation methods}
\label{sec:simothers}

\endinput
%
%% Local Variables:
%% mode:latex
%% End:
%%EOF
