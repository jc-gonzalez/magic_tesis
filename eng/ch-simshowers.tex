%%%%%%%%%%%%%%%%%%%%%%%%%%%%%%%%%%%%%%%%%%%%%%%%%%%%%%%%%%%%%%%%%%%%%%%%%%%
%%
%%  ch-simshowers.tex
%%
%%  Created: Fri Oct 10 14:24:37 1997
%%  Author.: Jose Carlos Gonzalez
%%  Notes..:
%%          
%%-------------------------------------------------------------------------
%% Filename: $RCSfile$
%% Revision: $Revision$
%% Date:     $Date$
%%%%%%%%%%%%%%%%%%%%%%%%%%%%%%%%%%%%%%%%%%%%%%%%%%%%%%%%%%%%%%%%%%%%%%%%%%%

\chapter{Simulation of atmospheric showers}
\label{chapter:simshowers}

Usually, the \MC simulations of atmospheric showers follow a well
defined scheme. They start the simulation at the top of the
atmosphere, with a given primary particle. Then they simulate the fly
and interaction of this particle. At a certain moment, this particle
produces some secondary particles, which are written to a stack, being
processed each of them afterwards. Each secondary particle have a
certain probability of interact or decay, and in virtue of this
probability new (tertiary) particles are produced and written into a
new stack. The process follows until a detector (or ground) is
reached. See Fig. \ref{fig:MCsimGaisser} as an example.

\mcsimsample

Several cuts (in energy, for example) or simplifications are normally
taken into account in the simulation programs in order to minimize the
computation time by rejecting, during the simulation, those particles
which fall into a ``dead zone'' of the whole process or our detector.
For instance, if we are interested in detecting electrons, and our
instrument do only detect those above a sharp cut-off energy
$E_{\mathrm{threshold}}$, perhaps we are interested in removing from
the simulation process any electron as soon as he reaches this energy.

%%==========================================================
\section{The \CORSIKA code}
\label{sec:CORSIKA}
%
One of the most spreadly used simulation codes in the field of cosmic
and $\gamma$-rays is the CORSIKA code \cite{CORSIKA:manual}. The last
official version released at the time of writing this work is 5.60.
CORSIKA (COsmic Ray SImulations for KAscade) is a detailed Monte Carlo
program to study the evolution and properties of extensive air showers
in the atmosphere, and was developed to perform simulations for the
KASKADE experiment, in Karlsruhe (Germany) \cite{KASKADE} (this
experiment aims to measure the elemental composition of the primary
cosmic radiation in the energy range $3\E{14}$--$5\E{16}$\u{eV}). The
code is mostly written in Fortran 77 (some pieces of code in new
versions are written in C for portability) and hence is easily portable
to almost any computer.

CORSIKA simulates interactions and decays of nuclei and subatomic
particles (hadrons, muons, electrons, photons, \ldots) up to energies of
10\pow{17}\u{eV}. All secondary particles are tracked along their
trajectories and their parameters when reaching the user-defined
detector levels (energy, position, direction and arrival time) are saved
into disk.

CORSIKA was originally developed on the basis of three well established
program systems. The first treats the hadronic part of proton induced
showers using the \emph{fire ball} and \emph{isobar}, and was written in
the 70s by P.K.F. Grieder \cite{Grieder:1979}. Nowadays, its structure
serves as the general frame where CORSIKA handles its input and output,
and the interaction routines used in this program can be used as
hadronic interactions at low energies. The second program was developed
by J.N. Capdevielle \cite{Capdevielle:1989} following the \emph{Dual
  Parton Model} (DPM)\cite{Capella:1980}, describing the interactions of
protons at high energies in good agreement with the measured collider
data. The third part, used for the simulation of the electromagnetic
part of the air shower, is based in the EGS4 code \cite{EGS4}. Whenever
possible, experimentally accesible data have been used.

For the simulation of the hadronic interactions several models can be
used (they are chosed by the user). The current available models are:

\begin{description}
  
\item[The DPM] These routines are based on the \emph{Dual Parton Model}
  and can be used for high energies to reproduce the kinematical
  distributions being measured or predicted by theory.
  
\item[VENUS] It simulates the inelastion ultra-relativistic heavy ion
  collisions with detailed simulation of creation, interaction and
  fragmentation of colour strings. It's the alternative to the DPM.
  
\item[Isobar and Fire-ball model routines] For low energy interactions
  
\item[GHEISHA] It's a more sophisticated replacement of the isobar and
  fire-ball routines. Suitable for energies up to several hundred
  \u{GeV}.

\item[Q??]      ??

\item[SYBILL] ??

\end{description}

The photoproduction of muon pairs and hadrons is added to the
electromagnetic part to allow the calculation of the muon content of
photon induced showers. A second way of treating the electromagnetic
component is to use the corrected and adapted form of the analytic NKG
formula for each electron or photon produced in the cascade.

It is also possible to include the generation of \Cherenkov light in
the atmosphere, to handle electronic and muonic neutrinos and
anti-neutrinos, and to simulate showers under nearly horizontal
incident directions (without the possibility of \Cherenkov light
generation, though).

We used a modified version of CORSIKA 4.50, suitable for our purposes.
We modified the detector geometry in order to allow the simulation of
collection of light by \Cherenkov telescopes. Several minor
modifications were also introduced in order to make easier the
handling of the output data.

\subsection{Input and output from CORSIKA}
%
The input for a CORSIKA run is a plain ASCII parameters file, with
several entries in the form ``\texttt{KEYWORD <par1> <par2>} \ldots'',
where \texttt{<parN>} means the N-th parameter for that keyword
\texttt{KEYWORD}. An example of a parameters file is shown in Fig.
\ref{fig:CORSIKAparamfile}.

\CORSIKAsampleInputfig

Among other parameters, the user can set the type of primary for the
showers (\texttt{PRMPAR}), the run number (\texttt{RUNNR}), the number
of event for this run (\texttt{NSHOW}), the energy range
(\texttt{ERANGE}), the simulated slope of the spectrum in this energy
range (\texttt{ESLOPE}), the range of $(\theta,\phi)$ angles
(\texttt{THETAP}, \texttt{PHIP}), the observation levels
(\texttt{OBSLEV}), the cuts in energy (\texttt{ECUTS}) which allow us
to reduce the computation time by rejecting those particles below a
certain threshold, and the geometry of the detector (\texttt{CERARY}
in the original version, \texttt{ERANGE} and \texttt{ERANGE} in our
modified version).

CORSIKA is able to treat the particles that are listed in Table
\ref{table:CORSIKA_GEANTcodes}

\CORSIKAGeantcodes_table

The output of CORSIKA comes in two different ways. First, we have an
ASCII output, with statistics and sumaries of the simulation of the
showers and even with debugging information if needed.  The real
output of CORSIKA is given in binary format. For each particle hitting
the detector area in our pre-defined observation levels we get the
type of the particle, the position in the plane of that observation
level, the direction, the energy, the height of production and the
arrival time. If we selected the CERENKOV version of CORSIKA, we'll be
able to get information about the \Cherenkov photons as well. Due to
the huge number of \Cherenkov photons produces at high energies, we
can generate and follow them in groups (\emph{bunches}).  However, for
our purposes we traced each single \Cherenkov photon. For these
photons we stored into disk the wavelength, the position, the
direction, height of production and arrival time.


\MORE%%%%%%%%%%%%%%%%%%%%%%%%%%%%%%%%%%%%%%%%%%%%%%%%%%%%%%%%%%%%

\subsection{Input and output from CORSIKA}
%

%%==========================================================
\section{Simulation of showers at High Zenith Angle}

%%==========================================================
\section{Information available from the showers}

%%==========================================================
\section{Time Structure of the \Cherenkov pulse from an atmospheric shower}

%%%%%%%%%%%%%%%%%%%%%%%%%%%%%%%%%%%%%%%%%%%%%%%%%%%%%%%%%%
\section{Other simulation methods}
\label{sec:simothers}
\endinput
%
%% Local Variables:
%% mode:latex
%% End:
%%EOF
