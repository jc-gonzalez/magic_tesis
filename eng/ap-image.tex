%%%%%%%%%%%%%%%%%%%%%%%%%%%%%%%%%%%%%%%%%%%%%%%%%%%%%%%%%%%%%%%%%%%%%%%%%%%
%%
%%  ap-image.tex
%%
%%  Created: Fri Oct 10 14:24:37 1997
%%  Author.: Jose Carlos Gonzales
%%  Notes..:
%%          
%%-------------------------------------------------------------------------
%% Filename: $RCSfile$
%% Revision: $Revision$
%% Date:     $Date$
%%%%%%%%%%%%%%%%%%%%%%%%%%%%%%%%%%%%%%%%%%%%%%%%%%%%%%%%%%%%%%%%%%%%%%%%%%%

\chapter{Calculation of the image parameters}
\label{appendix:image}

%%------------------------------------------------------------
\section{Statistical definition of the image parameters}

The data from every single event comes mainly represented as a set in
the form $\{(x_i, y_i, q_i)\}$, where $i=1,2,\ldots,n$, being $n$ the
number of pixels in the camera, $(x_i, y_i)$ the position of the
center of the $i$-th pixel, and $q_i$ its charge. We will assume that
all the pixels are equal, or, equivalent, that inside the value of
$q_i$ is already included any possible difference. If we
define\footnote{A runing index like $i$ will be assumed to run from
  its lowest to its highest value (previously defined) in a sum,
  unless explicitly written.} $Q\equiv\sum_i q_i$, then the first and
second moments of the distribution of values given by $\{(x_i,
y_i, q_i)\}$ are:
%
\begin{gather}
  \label{eq:defmeans}
  \mean{x} = \frac{\sum_i q_i x_i}{Q}, \qquad
  \mean{y} = \frac{\sum_i q_i y_i}{Q}, \qquad
  \\
  \label{eq:defmeans2}
  \mean{x^2} = \frac{\sum_i q_i x_i^2}{Q}, \qquad
  \mean{xy} = \frac{\sum_i q_i x_i y_i}{Q}, \qquad
  \mean{y^2} = \frac{\sum_i q_i y_i^2}{Q}
\end{gather}
%
We will call the point $(\mean{x},\mean{y})$ the \emph{centroid of the
  distribution}. The first moment around this centroid is zero. The
second moments are called \emph{variances}:
%
\begin{equation}
  \label{eq:defvar}
  \sigma_x^2  = \frac{\sum_i q_i (x_i - \mean{x})^2}{Q}, \qquad
  \sigma_{xy} = \frac{\sum_i q_i 
    (x_i - \mean{x})(y_i - \mean{y})}{Q}, \qquad
  \sigma_y^2  = \frac{\sum_i q_i (y_i - \mean{y})^2}{Q}
\end{equation}
%
being $\sigma_{xy}$ the so called \emph{covariance}. It can be shown
that these expressions can also be written as:
%
\begin{equation}
  \label{eq:defvarbis}
  \sigma_x^2  = \mean{x^2} - \mean{x}^2, \qquad 
  \sigma_{xy}  = \mean{xy} - \mean{x}\mean{y}, \qquad 
  \sigma_y^2  = \mean{y^2} - \mean{y}^2, \qquad 
\end{equation}
%
Finally, we define the \emph{correlation coefficient}:
\begin{equation}
  \label{eq:corr}
  \rho\equiv\frac{\sigma_{xy}}{\sigma_x\, \sigma_y}
\end{equation}
%
The range of values for the correlation coefficient is the interval
$[-1,1]$.

We define now the following auxiliary variables:
%
\begin{equation}
\label{eq:auxiliary}
\begin{split}
  d = \sigma_y^2 - \sigma_x^2 &\qquad\qquad
  z = \left(d^2+4\sigma_{xy}^2\right)^{1/2} \\
  u = \left(1+(d/x)\right) &\qquad\qquad
  v = 2 - u
\end{split}
\end{equation}
                                %
Using these expressions, the Standard Image Parameters are defined as:
%
\begin{subequations}
  \label{eq:imageparams}
  \begin{align}
%
    \label{eq:lengthdef}
    \text{\scshape length} &\equiv 
    \left[\frac{1}{2}\left(\sigma_x^2+\sigma_y^2+z\right)\right]^{1/2}\\
%
    \label{eq:widthdef}
    \text{\scshape width} &\equiv 
    \left[\frac{1}{2}\left(\sigma_x^2+\sigma_y^2-z\right)\right]^{1/2}\\
%
    \label{eq:distdef}
    \text{\scshape distance} &\equiv 
    \left(\mean{x}^2 + \mean{y}^2\right)^{1/2}\\
%
    \label{eq:azwidthdef}
    \text{\scshape azwidth} &\equiv 
    \left[\frac{\mean{x}^2\mean{y^2} 
        - 2\mean{x}\mean{y}\mean{xy} 
        + \mean{x^2}\mean{y}^2}{\text{\scshape distance}^2}\right]^{1/2}\\
%
    \label{eq:missdef}
    \text{\scshape miss} &\equiv 
    \left[\frac{1}{2}\left(u\mean{x}^2 + v\mean{y}^2\right)
        - \left(\frac{2\sigma_{xy}\mean{x}\mean{y}}{z}\right)\right]^{1/2}\\
%
    \label{eq:alphadef}
    \text{\scshape alpha} &\equiv
    \arcsin\left(\frac{\text{\scshape miss}}{\text{\scshape distance}}\right)\\
%
    \label{eq:concdef}
    \text{\scshape conc} &\equiv
    \frac{q_{\mathrm{max.1}} + q_{\mathrm{max.2}}}{Q}
%
  \end{align}
\end{subequations}

\imageparaxesfig

%\clearpage

%%------------------------------------------------------------
\section{Inference of the image parameters as Eigenvalues of the 
  Covariance matrix}

We will use the \emph{covariance matrix} to get expressions for the
standard image parameters {\scshape distance}, {\scshape length},
{\scshape width} and {\scshape alpha}.

The distribution of values $\{q_i\}$ along the image can be viewed as
a two-dimensional gaussian. The expression of such function (we change
the notation to $\mu_x\equiv\mean{x}$, in order to make things more
clear) is:
%
\begin{gather}
  \label{eq:gauss2d}
  f(x,y) = A 
  \exp\left\{ -\frac{1}{2(1-\rho^2)} 
    \left[
      \left(\frac{x-\mu_x}{\sigma_x}\right)^2
      - 2 \rho \left(\frac{x-\mu_x}{\sigma_x}\right)
      \left(\frac{y-\mu_y}{\sigma_y}\right)
      + \left(\frac{y-\mu_y}{\sigma_y}\right)^2
      \right]\right\}
    \intertext{with}
    A = \frac{1}{2\pi\sigma_x\sigma_y\sqrt{1-\rho^2}}
\end{gather}
%
The \emph{covariance matrix} is defined as:
%
\begin{equation}
  \label{eq:covmat}
  \Sigma \equiv
  \begin{pmatrix}
    \sigma_x^2 & \sigma_{xy} \\ \sigma_{xy} & \sigma_y^2
  \end{pmatrix}
\end{equation}
% 
The determinant and inverse of this matrix are:
%
\begin{align}  
  \label{eq:covmatdet}
  |\Sigma| &= \sigma_x^2 \sigma_y^2 - \sigma_{xy}^2 \\
  \label{eq:covmatinv}
  \Sigma^{-1} &= \frac{1}{|\Sigma|}
  \begin{pmatrix}
    \sigma_y^2 & -\sigma_{xy} \\ -\sigma_{xy} & \sigma_x^2
  \end{pmatrix}
\end{align}
%
With these definitions, we can put \eqref{eq:gauss2d} in vectorial form:
%
\begin{align}
  \label{eq:gauss2dbis}
  f(x,y) &= \frac{1}{2\pi|\Sigma|^{1/2}} 
  \exp\left[ -\frac{1}{2 |\Sigma|}
      \left( x - \mu_x, y - \mu_y \right)
      \begin{pmatrix}
        \sigma_y^2 & -\sigma_{xy} \\ -\sigma_{xy} & \sigma_x^2
      \end{pmatrix}
      \begin{pmatrix}
        x -\mu_x \\ y - \mu_y
      \end{pmatrix}
    \right]
    \\
  \intertext{or better}
  \label{eq:gauss2dvec}
  f(x,y)  &=  \frac{1}{2\pi|\Sigma|^{1/2}} 
  \exp\left[ -\frac{1}{2}
    (\mathbf{x} - \boldsymbol{\mu}) \Sigma^{-1} 
    (\mathbf{x} - \boldsymbol{\mu})^{\mathrm{T}}
  \right]
\end{align}
%
where the superscript \texttt{T} means transposition.  
  
First of all, we have our {\scshape distance} parameter, as always:
%
\begin{equation}
  \label{eq:distance}
  \text{\scshape distance} = \sqrt{\mean{x}^2 + \mean{y}^2}
\end{equation}
%
Let's make a translation of the axes from the system $S\equiv\{X,Y\}$
to $\hat{S}\equiv\{\hat{X},\hat{Y}\}$ (see Fig.
\ref{fig:imageparaxes}). Now the center of our new system is the
original $(\mean{x}, \mean{y})\equiv(\mu_x, \mu_y)$.  Since
$\mu_{\hat{x}} = \mu_{\hat{y}} = 0 $ in this new system, after this
translation the expression for our two-dimensional function becomes
(note that still $\hat{\Sigma} = \Sigma$):
%
\begin{align}
  \label{eq:gauss2dvechat}
  f(x,y)  &=  \frac{1}{2\pi|\Sigma|^{1/2}} 
  \exp\left[ -\frac{1}{2} \;
    \hat{\mathbf{x}} \; \Sigma^{-1} \; \hat{\mathbf{x}}^{\mathrm{T}}
  \right]
\end{align}
%
The intrinsic properties of this distribution are not changed by this
change of variable. In particular, the determinant of the covariance
matrix is invariant under translations and rotations (since it
expresses intrisic, local properties of the distribution itself). We
will use the following theorems (the reader can look for proofs
anywhere in the literature, in particular in \cite{}):

\begin{Theo}
  If $X$ and $Y$ are bivariate normal random variables, then it is
  always possible to find a change of variables to U and V, the new
  variables being a linear combination of the old, such that the new
  variables are uncorrelated.
\end{Theo}

\begin{Theo}
  In particular, we can always find a linear transformation
  $\mathbf{U}=\mathcal{C}\,\mathbf{X}+\mathbf{D}$, $\mathcal{C}$ a
  matrix and $\mathbf{U}$, $\mathbf{X}$ and $\mathbf{D}$ vectors, such
  that the components of $\mathbf{U}$ are standarized independent
  normals ($\mu=0,\sigma=1,\rho=0$).
\end{Theo}

\begin{Theo}
  If $\mathbf{X}$ is bivariate normal with diagonal variance matrix,
  then the components of $\mathbf{X}$ are independent. That is, that
  the correlation is zero is necessary and sufficient for the
  components of $\mathbf{X}$ to be independent. This is \emph{not}
  true in general for other distributions.
\end{Theo}

Therefore, there will be a system of axes, $S'\equiv\{X',Y'\}$, in
which the covariance matrix will be \emph{diagonal}. Let's try to
diagonalize $\Sigma$.  In order to do that, we have to calculate its
\emph{eigenvalues}, $\lambda_k$, $k=1,2$, by solving the equation:
%
\begin{equation}
  \label{eq:eigen}
  |\Sigma - \lambda \mathbf{I}| = 0
\end{equation}
% 
where $\mathbf{I}$ represents the identity matrix in two dimensions.
This becomes:
%
\begin{equation}
  \label{eq:eigen2}
  \begin{split}
    \begin{vmatrix}
      \sigma_x^2-\lambda & \sigma_{xy} \\ \sigma_{xy} & \sigma_y^2-\lambda
    \end{vmatrix}
    & = (\sigma_x^2-\lambda)(\sigma_y^2-\lambda) - \sigma_{xy}^2 \\
    & = \lambda^2 - \lambda(\sigma_x^2+\sigma_y^2) + 
    (\sigma_x^2\sigma_y^2 - \sigma_{xy}^2)  = 0
  \end{split}
\end{equation}
%
The solutions to this equation are:
%
\begin{equation}
  \label{eq:eigensol}
  \lambda_{1,2} = \frac{(\sigma_x^2+\sigma_y^2) \pm 
    \sqrt{(\sigma_x^2-\sigma_y^2) + 4\sigma_{xy}^2}}{2}
\end{equation}
%
and therefore the diagonalized matrix in that new system $S'$ will be:
%
\begin{equation}
  \begin{split}
    \label{eq:Snew}
    \begin{pmatrix}
      \lambda_1 & 0\\0 & \lambda_2
    \end{pmatrix}
    = &
    \begin{pmatrix}
      \displaystyle \frac{(\sigma_x^2+\sigma_y^2) +
        \sqrt{(\sigma_x^2-\sigma_y^2) + 4\sigma_{xy}^2}}{2} & 0 \\
      0 & \displaystyle \frac{(\sigma_x^2+\sigma_y^2) - 
        \sqrt{(\sigma_x^2-\sigma_y^2) + 4\sigma_{xy}^2}}{2}
    \end{pmatrix}\\
    \equiv &
    \begin{pmatrix}
      {\sigma'}_x^2 & 0\\0 & {\sigma'}_y^2
    \end{pmatrix}
    \equiv \Sigma'
  \end{split}
\end{equation}
%
where we assume that ${\sigma'}_x^2 > {\sigma'}_y^2$ and take into
account that in this new system must be ${\sigma'}_{xy}=0$. These
expresions represent our first two shape parameters:
%
\begin{align}
  \label{eq:lengthwidth}
  \text{\scshape length}^2 &\equiv {\sigma'}_x^2 = 
  \frac{1}{2}\left[(\sigma_x^2+\sigma_y^2) +
    \sqrt{(\sigma_x^2-\sigma_y^2) + 4\sigma_{xy}^2}\,\right] \\
  \text{\scshape width}^2 &\equiv {\sigma'}_y^2 = 
  \frac{1}{2}\left[(\sigma_x^2+\sigma_y^2) -
    \sqrt{(\sigma_x^2-\sigma_y^2) + 4\sigma_{xy}^2}\,\right]
\end{align}

The \emph{eigenvectors} of the matrix $\hat{S}$ are the vector
solutions to the equation:
%
\begin{equation}
  \label{eq:eigenveceq}
  \hat{S}\,\mathbf{v}_k = \lambda_k\,\mathbf{v}_k
\end{equation}
%
where $\lambda_k$ are the \emph{eigenvalues} calculated above. These
$\mathbf{v}_k$ can be written as:
%
\begin{equation}
  \label{eq:eigenvec}
  \mathbf{v}_{1,2} = \left(
    \frac{(\sigma_x^2-\sigma_y^2) \mp
      \sqrt{(\sigma_x^2-\sigma_y^2) + 4\sigma_{xy}^2}}{2\,\sigma_{xy}},
    \quad 1 \right)
\end{equation}
%
where the minus and the plus correspond to the indexes 1 and 2,
respectively. In general, these vectors are not normalized.

We could have obtained the same result applying the following
strategy. The difference between the systems $\hat{S}$ and $S'$ is a
rotation of angle $\theta$ (see Fig.\ref{fig:imageparaxes}):
%
\begin{equation}
  \label{eq:rot}
  \begin{pmatrix}
    x' \\ y'
  \end{pmatrix} =
  \begin{pmatrix}
    \cos\theta & \sin\theta \\ -\sin\theta & \cos\theta
  \end{pmatrix}
  \begin{pmatrix}
    x \\ y
  \end{pmatrix}
\end{equation}
%
The values of our two-dimensional gaussian however remain the same,
that is:
%
\begin{equation}
  \label{eq:equalz}
  \frac{1}{2\pi|\Sigma|^{1/2}} 
  \exp\left[ -\frac{1}{2} \;
    \hat{\mathbf{x}} \; \Sigma^{-1} \; \hat{\mathbf{x}}^{\mathrm{T}}
  \right]
  =
  \frac{1}{2\pi|\Sigma'|^{1/2}} 
  \exp\left[ -\frac{1}{2} \;
    {\mathbf{x}'} \; {\Sigma'}^{-1} \; {\mathbf{x}'}^{\mathrm{T}}
  \right]  
\end{equation}
%
but we know that $|\Sigma|$ remains invariant. Therefore:
%
\begin{equation}
  \label{eq:equalzvec}
  \hat{\mathbf{x}} \; \Sigma^{-1} \; \hat{\mathbf{x}}^{\mathrm{T}} =
  {\mathbf{x}'} \; {\Sigma'}^{-1} \; {\mathbf{x}'}^{\mathrm{T}} =
  \hat{\mathbf{x}} \; R^{\mathrm{T}} \; 
  {\Sigma'}^{-1} \; R \; \hat{\mathbf{x}}^{\mathrm{T}}
\end{equation}
%
that is:
%
\begin{equation}
  \label{eq:equalmat}
  \Sigma^{-1} = R^{\mathrm{T}} \; {\Sigma'}^{-1} \; R 
  \qquad \text{or better} \qquad
  {\Sigma'}^{-1} = R \; \Sigma^{-1} \; R^{\mathrm{T}} 
\end{equation}
%
Explicitly expressed in matricial form:
%
\begin{multline}
  \label{eq:rotexplicit}
  \begin{pmatrix}
    {\sigma'}_y^2 & -{\sigma'}_{xy} \\ -{\sigma'}_{xy} & {\sigma'}_x^2
  \end{pmatrix}
  =
  \begin{pmatrix}
    \cos\theta & \sin\theta \\ -\sin\theta & \cos\theta
  \end{pmatrix}
  \begin{pmatrix}
    \sigma_y^2 & -\sigma_{xy} \\ -\sigma_{xy} & \sigma_x^2
  \end{pmatrix}
  \begin{pmatrix}
    \cos\theta & -\sin\theta \\ \sin\theta & \cos\theta
  \end{pmatrix} \\
  =
  \begin{pmatrix}
    \sigma_x^2 \sin^2\theta + \sigma_y^2 \cos^2\theta 
    - \sigma_{xy} \sin2\theta&
    \frac{1}{2}(\sigma_x^2 - \sigma_y^2)\sin2\theta  
    - \sigma_{xy}\cos2\theta \\
    \frac{1}{2}(\sigma_x^2 - \sigma_y^2)\sin2\theta 
    - \sigma_{xy}\cos2\theta &
    \sigma_x^2 \cos^2\theta + \sigma_y^2 \sin^2\theta 
    + \sigma_{xy} \sin2\theta
  \end{pmatrix}
\end{multline}
%
We already know that ${\sigma'}_{xy} = 0$, and therefore the following
equation must hold:
%
\begin{equation}
  \label{eq:sigmapxy}
  \frac{1}{2}(\sigma_x^2 - \sigma_y^2)\sin2\theta  
    - \sigma_{xy}\cos2\theta = 0
\end{equation}
%
This gives us the angle $\theta$:
%
\begin{equation}
  \label{eq:theta}
  \theta = \frac{1}{2} \arctan 
  \left(\frac{2\sigma_{xy}}{\sigma_x^2 - \sigma_y^2}\right)
\end{equation}
%
On the other hand, the angle $\chi$ in Fig. \ref{fig:imageparaxes} is
given by:
%
\begin{equation}
  \label{eq:chi}
  \chi = \arctan\left(\frac{\mean{y}}{\mean{x}}\right)
\end{equation}
%
And therefore, our last standard image parameter is:
%
\begin{equation}
  \label{eq:alpha}
  \text{\scshape alpha} \equiv \alpha = \chi - \theta
\end{equation}


%%------------------------------------------------------------
\section{New parameters}

\subsubsection{Asymmetry}
%
We have assumed that the distribution of light from a gamma event in
the camera plane can be approximated by a two-dimensional gaussian.
But in reality our distribution of light will be asymmetric. In
particular, there will be a clear asymmetry along the main axis of the
elliptical image, as it is shown in Fig.\ref{fig:asym}. In order to
quantify this effect we will use, in addition to the definitions given
in Eqs.\eqref{eq:defmeans} and \eqref{eq:defmeans2}, the third moments
around the origin:
%
\begin{equation}
  \label{eq:defthirdmom}
  \mean{x^3} = \frac{\sum_i q_i x_i^3}{Q}, \qquad
  \mean{x^2y} = \frac{\sum_i q_i x_i^2 y_i}{Q}, \qquad
  \mean{xy^2} = \frac{\sum_i q_i x_i y_i^2}{Q}, \qquad
  \mean{y^3} = \frac{\sum_i q_i y_i^3}{Q}  
\end{equation}
%
and the third moments around the mean\footnote{Although now we use the
  notation $\sigma_{x^2y}$, for example, without any exponent for the
  $\sigma$, a $3$ must appear. We just removed it for clarity}:
%
\begin{equation}
  \label{eq:defthirmommean}
  \begin{split}
    \sigma_{x^3} & = \mean{x^3} - 3\mean{x^2}\mean{x} + 2\mean{x}^3\\
    \sigma_{x^2 y} & = \mean{x^2 y} - 2\mean{xy}\mean{x} 
    + 2\mean{x}^2\mean{y} - 2\mean{x^2}\mean{y} \\
    \sigma_{x y^2} & = \mean{x y^2} - 2\mean{xy}\mean{y} 
    + 2\mean{x}\mean{y}^2 - 2\mean{x}\mean{y^2} \\
    \sigma_{y^3} & = \mean{y^3} - 3\mean{y^2}\mean{y} + 2\mean{y}^3
  \end{split}
\end{equation}
%
Using again the definitions given in Eq.\eqref{eq:auxiliary}, we calculate the vector $\mathbf{u}$:
%
\begin{equation}
  \label{eq:vectoru}
  \mathbf{u} = \left(
    \left(\frac{z-d}{2z}\right)^{1/2},\quad
    \sign(\sigma_{xy})\left(\frac{z-d}{2z}\right)^{1/2}
    \right)
\end{equation}
%
With this, we define the \emph{Asymmetry vector} $\mathbf{A}$ (see
Fig.\ref{fig:asym}):
%
\begin{equation}
  \label{eq:asymmetryvec}
  \mathbf{A} \equiv (x_{\mathrm{A}},\, y_{\mathrm{A}}) 
  \equiv -\sigma_{\mathrm{A}}\,\mathbf{u}
\end{equation}
%
with
%
\begin{equation}
  \label{eq:sigmaA}
  \sigma_{\mathrm{A}} = \left[
    \sigma_{x^3}\cos^3\phi +
    3\, \sigma_{x^2 y}\cos^2\phi\sin\phi +
    3\, \sigma_{x y^2}\cos\phi\sin^2\phi +
    \sigma_{y^3}\sin^3\phi\right]^{1/3}
\end{equation}
%
being $\phi$ the angle that forms the vector $\mathbf{u}$ with the $X$
axis. Finally, we define the new image parameter {\scshape asymmetry}:
%
\begin{equation}
  \label{eq:asymmetry}
  \text{\scshape asymmetry} \equiv
  \sign(\mathbf{A}\cdot\mathbf{D})
  \frac{|\sigma_{\mathrm{A}}|}{\text{\scshape length}} =
  \frac{\mathbf{A}\cdot\mathbf{D}}{\text{\scshape distance}\mspace{8mu}
    \text{\scshape length}\,\cos(\text{\scshape length})}
\end{equation}

\subsubsection*{Concentration}
%
We can extend the definition of the {\scshape concentration}
parameter, as follows:
%
\begin{equation}
  \label{eq:conck}
  \text{\scshape conc}_k \equiv
  \frac{q_{\mathrm{max.1}} + q_{\mathrm{max.2}} + \cdots 
    + q_{\mathrm{max.}k}}{Q}
\end{equation}
%
We then have that the concentration parameter defined in
Eq.\eqref{eq:concdef} is $\text{\scshape conc} \equiv \text{\scshape
  conc}_2$.

\clearpage

\asymfig

\mbox{}

\endinput
%
%% Local Variables:
%% mode:latex
%% TeX-master: t
%% End:

%%EOF
