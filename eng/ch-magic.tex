%%%%%%%%%%%%%%%%%%%%%%%%%%%%%%%%%%%%%%%%%%%%%%%%%%%%%%%%%%%%%%%%%%%%%%%%%%%
%%
%%  ch-magic.tex
%%
%%  Created: Fri Oct 10 14:24:37 1997
%%  Author.: Jose Carlos Gonzalez
%%  Notes..:
%%          
%%-------------------------------------------------------------------------
%% Filename: $RCSfile$
%% Revision: $Revision$
%% Date:     $Date$
%%%%%%%%%%%%%%%%%%%%%%%%%%%%%%%%%%%%%%%%%%%%%%%%%%%%%%%%%%%%%%%%%%%%%%%%%%%

\chapter{The \MAGIC Telescope}
\label{chapter:magic}

The \I{MAGIC} Telescope (\emph{Mayor Advanced Gamma-rays Imaging
  Cherenkov Telescope}) is a project devoted to the construction of a
17\u{m} diameter \Cherenkov telescope, with the most advanced
technology available in this field. Many of the new elements used and
developed especially for this project, among other technical
innovations, are: low weight, high reflectivity, coated aluminum
mirrors; \I{active mirror control}, \I{Flash ADC}s, \I{optical fiber}
links, fast slew for \I{GRB}s observation, multilevel trigger,
atmospheric monitoring, etc.

Nowadays $\gamma$-rays experiments on board of satellites can detect
photons with energies as high as 10\u{GeV}. Higher energies are
impossible to detect for such devices, due to their small collection
area. On the other hand, the most advanced Cherenkov detectors
currently operatives have a detection threshold above 200--300\u{GeV}.
Below this values, their sensibility drops down drastically. The main
goal of the \MAGIC Telescope (\MAGIC hereafter) is to cover this
energy range between 10\u{GeV} and 200\u{GeV}.
%
\energygapfig

The detector \I{EGRET} (\emph{Energetic Gamma Ray Experiment
  Telescope}, installed on board of the satellite \I{CGRO}
(\emph{Compton Gamma-Ray Observatory}, and with an upper detection
threshold of 10\u{GeV}, registered data about a big amount of
$\gamma$-ray sources. However, from the set of observed extra-galactic
sources, most of sources with high %
%
$z$\footnote{The magnitude denoted by $z$ is the so called
  \emph{redshift}.  It gives the observed deviation in the wavelength
  from a wave coming from a source moving away from us. Its
  calculation can be done using the expression
  $\Delta\lambda/\lambda=v/c=z$, where $v$ is the velocity of the
  source, and $c$ is the speed of
  the light in vacuum.}%
%
cannot be observed at energies above some hundreds of \u{GeV} using
other detectors. Also, there are only three detected
\emph{\Isee{supernov\ae\ remnant}{SNR}} (\I{SNR} hereafter): the Crab
Nebula, Vela and \I{SN\,1006}, at energies of some \u{TeV}, even when
they are supposed to be one of the main sources of cosmic
radiation. In both cases (extra-galactic sources and supernov\ae) the
detection could be possible in the range 10--200\u{GeV}.  Therefore,
it is clear that this energy window is of extreme importance for the
$\gamma$-ray astronomy.

\CGROenergiesfig

\section{Physical goals}
%
We can say, there is a major physical goal within the \MAGIC Telescope
Projects, namely, \emph{covering the unexplored energy range from 10\u{GeV}
to 100\u{GeV}}. This global aim is expressed, however, in a variety of
more specific targets, which now I will try to enumerate:

\subsubsection*{Active Galactic Nuclei (AGN)} 
%
About 1\% of the galaxies are found to be active. Active means, in
this context, that with their core is associated a highly luminous and
variable non-thermal emission. In this big class of objects we can
find the radio-galaxies, the Seyfert galaxies, quasars (quasi-stellar
radio objects, both radio-loud and radio-quiet), and the BL-Lacert\ae\
(BL-Lac) objects. A detailed classification of the AGN is shown in
figure \ref{fig:AGNclassification} (taken from
\cite{Petry:tesis}).

\AGNclassificationfig

Since the detection of several AGN by EGRET, the attention has been
focussed on the \emph{blazar} subclass. This subclass consists of the
\emph{BL Lac objects} and the \emph{Flat Spectrum Radio Quasars}
(FSRQs). Both types of objects are characterized by flat
radio-spectrum, strong variability and polarization of the optical
emission. The blazars are the main candidates for the high and very
high energy gamma-ray emission.

The main questions related to the AGN observations are:
%
\begin{itemize}
\item How do AGN form and evolve?
\item What do the jets (relativistic plasma outflows) from AGN consist
of and how is this plasma accelerated?
\item Is there a \emph{cut-off} for the photon energy spectrum? How
and where are these cutoffs caused?
\item What's the reason for such strong variability?
\end{itemize}

Blazar emission for X-ray selected BL Lac objects at lower energies
(up to about 1-100 keV) is almost certainly due to synchrotron
emission from a beam of highly relativistic electrons. Above
300\u{GeV}, only three AGN have been detected (and one of them only by
the Whipple Telescope, in a short flare). But more important than
this: from these AGN, only one (Mkn421) has been detected by
EGRET. Since EGRET detected several tens of AGN at energies below
30\u{GeV}, most of these AGN detected by EGRET must have their cut-off
at energies between 30 and 300\u{GeV}. However, there are several
scenarios with different explanations. Under the assumption that the
\emph{diffuse extra-galactic $\gamma$-background} might be due to the
sum of the emission of unresolved blazars, the spectrum of these AGN
should extend almost unchanged in slope up to 50--100\u{GeV}.

Moreover, there could be a whole sub-class of AGN with its second peak
in the spectral energy distribution above 10\u{GeV} (the first peak in
the spectrum is between UV to X-rays). These AGN would be difficult to
detect by satellite experiment. This makes the study of AGN in this
unexplored energy range especially interesting for modeling blazars.

\subsubsection*{Supernov\ae\ remnants (SNRs)}
% 
As we already mentioned, only three SNRs are related to $\gamma$-ray
emission: the Crab Nebula, Vela and SN1006. The production mechanism
for $\gamma$-rays is believed to be Inverse Compton Scattering (ICS)
of synchrotron and background field photons by VHE electrons, and
$\pi^0$ decay into $\gamma$s after the interaction of accelerated
hadrons with target matter external to the remnant. The main question
that link MAGIC with the study of SNRs are:
%
\begin{itemize}
\item Which kind of particles are responsible of the production of
high and very high energy $\gamma$s in SNRs?
\item Is the mechanism of production of $\gamma$s dependent on the
type on SNR?
\item Are SNRs really responsible for the bulk of cosmic rays?
\end{itemize}

\subsubsection*{Pulsars} 
%
Radio pulsars are the expected energy sources in Plerion type
supernova remnants. Apart from the Crab pulsar, another two pulsars
have been detected at very high energies: the Vela pulsar (plerion)
and PSR\,1706-44. All the pulsars detected by EGRET seem to have quite
different properties. The most recent theoretical considerations
suggest that pulsed emission from isolated pulsars cannot be expected
beyond 6--20\u{GeV} in the \emph{polar-cap model}, or around
100\u{GeV} in the \emph{outer-cap model}. This is already more than
enough to study the radio-pulsars in the MAGIC energy range.

Concerning to radio-quiet pulsars, only one has been so far detected
at high energies: the Geminga pulsar. If we assume this is the closest
pulsar of its type, one can estimate the number of such objects in our
galaxy to be around 1600. MAGIC will look for objects fainter than
Geminga. In this set are also the faintest unidentified EGRET sources.

\subsubsection*{Gamma Ray Bursts} 
%
A clear milestone in the current status of the Gamma Ray Astrophysics
is the nature of the GRBs. With MAGIC we will be able to measure the
power spectrum of GRBs. MAGIC will be able of fast positioning (from
one position to any other in less than 30 seconds). This, together
with the help of fast timing and positioning information from future
detectors on board of satellites will allow us to record the bulk of
the flux emitted by GRBs (around 100\u{sec}, although it cam be longer
--- for the GRB\,971208 the single peaked burst lasted around
800\u{sec}).

\subsubsection*{Unidentified EGRET sources} 
%
We have already mentioned in many places the importance of look for
the unidentified EGRET sources in the MAGIC energy range
10--200\u{GeV}. About 40\% of these sources have hard power spectra
which seem to extend beyond 10\u{GeV}. The observation of these
objects with MAGIC will enhance the accuracy of the location from
0.1\deg to 0.02\deg, and therefore the possibility of identification
and study of these sources.

\MAGICframefig

\section{Optical characteristics of the reflector}
%
The \MAGIC telescope is provided with a 17\u{m} diameter octagonal
reflector. The mount is alt-azimuthal. The reflector is supported by a
3 layers space frame, and is fixed at two sides onto the azimuth drive
by stub-axes. This space frame is assembled from very strong
low-weight carbon fiber-epoxy tubes and aluminum knots. The dish is
mounted inside a ring (the $\theta$-ring) which hold the camera at its
upper apex, while the lower section is equipped with a drive chain for
altitude movement (see Fig.\ref{fig:MAGICframe}). The telescope has
been designed to have a f/D:1 optics, i.e. the global focal length of
the system is 17\u{m}. The $\theta$-ring has an elliptical shape
(\emph{gothic arc}), in order to avoid the 30--32\u{m} diameter
circular ring needed to follow the f/D:1 optics.

The global shape of the dish is parabolic, and is tessellated: it is
composed of 920 small squared mirrors, $50\times 50 \u{cm}^2$ each.
The total mirrored surface is then $230\u{m}^2$. The local shape of
these small mirrors is spherical. This design allows a small
\emph{spherical aberration} together with a small time spread of the
signal in the camera. Due to the relatively large increasing in
distance of the mirrors to the optical axis of the telescope, their
curvature radii are taken to vary between 34\u{m} and 36\u{m},
approximately.  The small square mirrors ($50\times 50 \u{cm}^2$) are
distributed following a square array. They are grouped in sets of
four, forming an individual element of $1\times 1 \u{m}^2$. Each of
these sets will be independently adjusted and focussed (\emph{active
  optics}).

The deformation of the frame will be monitorized by a laser beam
pointed to several semi-transparent silicon strip sensors, with an
overall resolution in the measurement of the deformations on the order
of few micrometers. The mirror elements, in groups of four, will be
focussed and aligned used an \emph{active mirror control
system}. Briefly, a laser pointer located in each panel of four
mirrors will be pointed towards the center of the camera. The spot in
the camera plane will be checked using a video camera, and two
stepping motors, driven via a multiplexer by a computer-controlled
drive circuit, will adjust the position and orientation of the panel
(see Fig.\ref{fig:MAGICmirrorcontrol}).

\MAGICmirrorcontrol_wfig
%
The single mirror elements have different focal distances, and
therefore the surfaces have to be machined on a numerically controlled
machine. \MAGIC will use all-aluminum sandwich mirrors, similar to
those already in use in the telescope CT1 of HEGRA (see
\ccite{MAGIC:Barrio_Kruger}). In Fig. \ref{fig:MAGICmirrorsandwich} we can
see an exploded view of a mirror element. The reflectivity of these
mirrors, made from an Aluminum alloy, is of the order of $R(\lambda)
> 80\%$ for the wavelength range $320-650\u{nm}$. In addition, the
mirrors are protected by a hard transparent overcoating.

\section{The camera of \MAGIC}
%
\subsection{General overview}
%
The camera is the decisive element in the construction of every
Cherenkov telescope. As a detection device, its characteristics will
influence the global performance of the telescope in terms of
efficiency, sensitivity and dynamic range (in the sense of energy
threshold and saturation energy, for instance). Nowadays, every camera
of a Cherenkov telescope is composed of several small detection units
(\emph{pixels}), normally \emph{photo-multipliers} (PMTs). Therefore,
the overall characteristics of the Cherenkov telescope can be
attributed to:
%
\begin{enumerate}[a.]
\item The distribution of pixels in the camera (geometry)
\item The pixel size
\item The camera size (field of view --- f.o.v.)
\item The different density of pixels across the camera (related to
the pixel size)
\item The sensitivity of the pixels
\end{enumerate}

\paragraph{The \emph{distribution of pixels in the camera}} affects only
slightly the performance of the telescope. While a square matrix
distribution can be easier to build and even more convenient for data
analysis, the most widely distribution used is the \emph{hexagonal
pattern}. Using hexagonal pixels one can minimize the spaces between
pixels.

\paragraph{The \emph{pixel size}} is getting smaller and smaller. For
instance, the first HEGRA camera had a pixel size of 0.4\deg, while in
the next generation of Cherenkov telescopes the pixels size will be
around 0.10\deg--0.15\deg. Theoretical considerations give us an
optimal value of $\leq 0.1\deg$. For \MAGIC we plan to use two
different pixel sizes for the camera. In the central part we will use,
in a first step, a pixel size of 0.1\deg. Then we will have an outer
ring of 0.2\deg PMTs. The reason for this is the following: the
low-energy $\gamma$-rays produce showers which image is closer to the
camera center (since they are produce higher in the atmosphere) and
more compact (the distance is bigger, the amount of light smaller, and
the Cherenkov angle in the maximum development of the shower is
smaller than for higher-energy showers). Therefore, in the center of
the camera we need to use a small pixel size. Since the aberration
effects are increasing towards the edge of the camera, and only the
high-energy showers will give trigger even when the image is far away
from the center (for they give much more light), in the outer part of
the image we don't need a very fine pixelization. By using bigger
pixels in the edge, we can lower the already high cost of the whole
camera.

\mirrorsandwichfig

\paragraph{The \emph{camera size} (field of view --- f.o.v.)} is, on
the other hand, getting larger. The benefits of a f.o.v are mainly:
%
\begin{enumerate}[i)]
\item We can study \emph{extended sources}.
%
\item The reconstruction of high energy showers is more efficient
(for a small camera we have the so called \emph{edge-effects}: a large
shower image is truncated at the edge, and therefore the parameters of
the image and the amount of light is badly estimated, leading to a
wrong energy reconstruction or, eventually, a rejection of genuine
$\gamma$-ray events).
%
\item At \emph{high zenith angles} (HZA hereafter) the showers develop
at much larger distances from the telescope. Therefore, due to the
scattering and the transverse momentum of the charged particles, the
\emph{Cherenkov pool}\footnote{The \emph{Cherenkov pool} is the region
from the axis of the shower till the position of the
\emph{hump}. Depending on the energy, in this region the density of
light can increase, decrease or be constant with the distance to the
shower axis, but can be assumed to be very uniform for studies with
Cherenkov telescopes. Beyond the \emph{hump}, the density of light
drops exponentially.} is much bigger. The consequence of this is that
the showers can trigger the system at larger core distances. But this
means that the shower image will be further from the center than in
the \emph{low zenith angle} (LZA) case. Therefore, a large camera will
be necessary to make use of HZA observations. 
\end{enumerate}

\paragraph{The \emph{sensitivity of the pixels}} is one of the most
crucial elements in the development of the camera. The performance of
the whole system is directly proportional to the \emph{Quantum
efficiency} (\QE hereafter) of the pixels. We will talk about the
different kinds of pixels further below.

\mirrorreflecfig

For \MAGIC we will use a camera size of $\sim$ 4\deg--5\deg, with
around 900 pixels. (In the simulations we only used a camera of
3.5\deg diameter, and all the pixels are the same size, 0.1\deg).  In
Fig.\ref{fig:MAGICcamera} we can see the geometrical pixel pattern of
a possible design of the MAGIC Telescope's camera (design of the
author, adapted from \ccite{MAGIC:DR}).

\QuantumEfffig

\subsection{Light sensors}
%
In the design of \MAGIC we took into account the possibility of an
exchangeable camera, in order to allow modularity and simplicity in
the upgrades. Several kinds of detection devices have been designed,
improved and tested, and in principle three different camera concepts
are under study:
%
\begin{enumerate}[a.]
\item The \emph{Classical camera}, using standard photo-multipliers
%
\item The \emph{Standard camera}, which uses hybrid photo-detectors:
high \QE GaAsP, intensified photocells with avalanche diode readout.
%
\item The \emph{Advanced camera}, a future solution which employs all
silicon avalanche photo-diodes.
\end{enumerate}
%
We will locate hollow funnels (\emph{Winston cones}) on top of each
pixel to concentrate the light in the detector device and restrict the
acceptance. In addition, wavelength-shifter coatings (WLS) will be
used in order to enhance the detection of UV Cherenkov light by
shifting part of this light into the region where the detection
devices have a greater \QE (see Fig.\ref{fig:QE}). We will talk a
little bit about these different detection devices in the following
paragraphs.

\MAGICcamerafig

\pixelreadoutfig

\subsubsection{Classical devices: photo-multipliers (PMTs)}
%
In the first stage of the development of the project a classical PMTs
camera will be used. This will affect not only the energy threshold of
the system but also the potential for HZA observations. However, still
the energy threshold is estimated to be around 30\u{GeV}, which is a
huge improvement compared to other existing telescopes or projects. In
the simulations the PMT used for the classical camera is the EMI
9083-A (see Fig.\ref{fig:QE}) with a peak \QE of about 26\% and a
\emph{mean effective} \QE, \QEeff, of around 17\%, being the \QEeff
defined as the mean value of the \QE in the wavelength range
290--300\u{nm}, folded with the Cherenkov light spectrum, i.e.:
%
\meanQEdefeq
%
where $I_\mathrm{Ch}(\lambda)$ is the intensity of the \Cherenkov
light emitted at the wavelength $\lambda$.

\subsubsection{Hybrid intensified photo-cells (IPCs)}
%
The aim of the telescope project is to lower the energy threshold down
to 10\u{GeV}. This goal is easily reachable by using IPCs (intensified
photo-cells, as they are called by the company Intevac, one of the
developers). These devices reach more than 40\% \QE at some points in
the region of interest 290--300\u{nm}, with room for more
improvements. This, together with the use of WLS will increase
significantly the \QEeff\ for this type of camera. By using an
avalanche photo-diode in the anode (instead of a classical anode
diode), a single electron response is obtained. In addition, the
problem of ion-feedback (ions released in the vicinity of the anode
which accelerated back to the cathode) can be prevented with an ion
deflector.

\subsubsection{Avalanche Photo-Diodes (APDs)}
%
The future version of the camera of \MAGIC plans to use all-silicon
Avalanche Photo-Diodes (APDs). These detection devices have a \QEeff\ 
of around $\geq$80\%. This value is twice the best value achieved with
an IPC using an avalanche photo-diode in the anode. This devices will
allow not only a dramatic increase in the yield of \Cherenkov photons,
but also smaller pixel sizes (0.05\deg--0.07\deg), for they have sizes
of 3--4\u{mm}. However, the technology is not yet ready for
application. The main problem of APDs is the large noise levels, still
unacceptable for our purposes. But the development continues and in
few years it will be possible the use of APDs as detection devices in
a future \MAGIC camera.

All of the camera designs will share a common readout chain, shown in
Fig.\ref{fig:pixelreadout}. 

\MORE%%%%%%%%%%%%%%%%%%%%%%%%%%%%%%%%%%%%%%%%%%%%%%%%%%%%%%%%%%%%

\section{Other components of the system}

\paragraph{Frame deformation monitor}

\paragraph{Tracking monitor}

\paragraph{Starfield monitor}

\paragraph{Mirror quality monitor}

\paragraph{Active mirror control}

\paragraph{Weather conditions monitor}

\section{Data Acquisition}

\section{Observational Programs}

\section{The Telescope CT1 of the HEGRA Collaboration}

\endinput
%
%% Local Variables:
%% mode:latex
%% End:

%%EOF
