%%%%%%%%%%%%%%%%%%%%%%%%%%%%%%%%%%%%%%%%%%%%%%%%%%%%%%%%%%%%%%%%%%%%%%%%%%%
%%
%%  ap-corsika.tex
%%
%%  Created: Fri Oct 10 14:24:37 1997
%%  Author.: Jose Carlos Gonzales
%%  Notes..:
%%          
%%-------------------------------------------------------------------------
%% Filename: $RCSfile$
%% Revision: $Revision$
%% Date:     $Date$
%%%%%%%%%%%%%%%%%%%%%%%%%%%%%%%%%%%%%%%%%%%%%%%%%%%%%%%%%%%%%%%%%%%%%%%%%%%
%

\chapter{Description of the \CORSIKA code}
\label{appendix:corsika}

For the generation of Monte Carlo atmospheric showers we used a
modified version of \CORSIKA 5.20, suitable for our purposes.  We
modified the detector geometry in order to allow the simulation of
collection of light by \Cherenkov telescopes. Several minor
modifications were also introduced in order to make easier the
handling of the output data.  

As we said in Chapter \ref{chapter:simshowers}, the input for a
\CORSIKA run is a plain ASCII parameters file, with several entries in
the form ``\texttt{KEYWORD <par1> <par2>} \ldots'', where \texttt{<parN>}
means the N-th parameter for that keyword \texttt{KEYWORD}. An example
of a parameters file was shown in Fig. \fullref{fig:corinput}.

The binary output of \CORSIKA is divided in different files, blocks
and subblocks. The nature of these blocks was already explained in
Chapter \ref{chapter:simshowers}. Here, we show the structure of each
of these blocks.

\CORSIKAtableRH

\CORSIKAtableRH

\CORSIKAtableEHone

\CORSIKAtableEHtwo

\CORSIKAtableEE

\CORSIKAtableRE

\CORSIKAtablePART

\CORSIKAtableCHER

\CORSIKAtableSTA


\endinput
%
%% Local Variables:
%% mode:latex
%% TeX-master: t
%% End:

%%EOF
