%%%%%%%%%%%%%%%%%%%%%%%%%%%%%%%%%%%%%%%%%%%%%%%%%%%%%%%%%%%%%%%%%%%%%%%%%%%
%%
%%  intro.tex
%%
%%  Created: Fri Oct 10 14:24:37 1997
%%  Author.: Jose Carlos Gonzales
%%  Notes..:
%%          
%%-------------------------------------------------------------------------
%% Tesis  :: RCS controlled system
%% Filename: $RCSfile$
%% Revision: $Revision$
%% Date:     $Date$
%% $Id$
%%%%%%%%%%%%%%%%%%%%%%%%%%%%%%%%%%%%%%%%%%%%%%%%%%%%%%%%%%%%%%%%%%%%%%%%%%%
%

\chapter*{Introduction}
\label{chapter:intro}
\markboth{Introduction}{Introduction}
\renewcommand{\headname}{Introduction}
\addcontentsline{toc}{chapter}{\numberline{}Introduction}
%
In the last years the {\bfseries High Energy Astrophysics} (HEA from
now on), discipline which studies the high and very high energy cosmic
radiation, has evolved considerably.  Since the beginning of this
century, when the first measurements of cosmic radiation were done by
instruments in baloons, this field of Physics has provided more and
more data, useful to achieve a more deep knowledge of the cosmic
radiation. In the last ten years, with the direct observation of gamma
ray sources from detectors on ground and from satellites, the HEA has
experienced a big impulse, and the experiments are built
everywhere. Under these conditions, more and more sources are
detected. The HEA now tries not only to detect new gammas ray sources,
but also to understand the properties of the observed sources, to make
new astrophysical models of these objects, and to provide of
constraints to accept or reject the existent models.

Still, there are a lot of new things to be discovered. Most of them
are related with the different kinds of astrophysical objects, namely
the $gamma$-ray sources, the
\Isee{active galactic nucleii}{AGNs} (\I{AGNs}), the
\I{BL-Lacert\ae} objects, \I{nov\ae}, \I{supernov\ae},
\I{neutron stars}, \I{pulsars}, \I{black holes}, \Iw{Gamma
Ray Bursts}{GRBqs} (\I{GRBs}), or \Iw{quasars}{QSOs} (\I{QSOs}).

The observations in HEA are splitted, traditionally, in two minor
fields. First, the study of {\itshape cosmic rays}, high energy
particles generated at high energy sources, that fly in the space and,
at some point, reach the surface of the Earth. Second, we can study
the $\gamma$-ray sources.

In the study of the {\itshape cosmic rays}, the spectrum in the very
high and ultra-high energy regions has been measured. There are direct
observations till around 1 PeV
(1\u{PeV}=10\pow{15}\u{eV})\footnote{About units and orders of
  magnitude, see appendix \fullref{appendix:units}.}. There are also
indirect measurements from detectors on ground from approx. 500 GeV
till around 10 EeV (1\u{EeV}=10\pow{18}\u{eV}). Finally, there are
studies concerned with the chemical composition of the cosmic rays as
well.

The last advances in Cosmic Rays Physics are very important for the
HEA. By studying the characteristics, quantitatively and
qualitatively, of the cosmic rays, we can discriminate what models are
suitable for the creation, emission, transport and interaction of
these particles. We can also fine-tune the current models available.
Finally, we can discover and study new properties, new objects or new
physical scenarios as well.

In the first chapter we study briefly the different cosmic rays
sources. We also make a review of the physical processes related with
the generation, composition and interaction of the cosmic radiation.
Then we describe the physics of the atmospheric showers and the more
relevant observables measured or calculated in any cosmic or gamma
rays experiment (chapter \ref{chapter:atmshowers}). The next chapter
is devoted to the methods of atmospheric showers' simulation, (chapter
\ref{chapter:simshowers}).  We talk about the \CORSIKA code (section
\ref{sec:CORSIKA}), and all the modifications introduced. This chapter
ends up with a brief description of another simulation methods used
(section \ref{sec:simothers}).

In the next chapter (\ref{chapter:magic}) we describe the \MAGIC
telescope, now still a project, which is in few words an octogonal
tesellated mirror of about 17\u{m} diameter. The different
characteristics of the detectors, reflector and the planned
operational mode are shortly exposed. We talk also concisely, as a
comparison, about the telescope CT1 of the \HEGRA ({\itshape High
  Energy Gamma Ray Astronomy}) Collaboration.

In the following chapter we describe the way how our simulations were
done, both for \MAGIC and CT1 (chapters \ref{chapter:simmagic} and
\ref{chapter:simct1}). We show the results obtained for different
physical parameters such as \emph{detection} and \emph{trigger
  efficiencies}, \emph{effective collection areas}, or
\emph{differential and integral detection rates}. By simulating the
detection of atmospheric showers by \MAGIC y CT1, we can obtain
theoretical values for several parameters, like the \emph{angular} and
\emph{energy resolution}. These calculations are described in the
chapter \ref{chapter:resol}.

In the next chapter (chapter \ref{chapter:imageana}) we start the
analysis of the images obtained in the cameras of \MAGIC and CT1.
After applying different algorithms for the cleanning of the images,
we calculate different parameters, which will allow us to
differentiate the primary that originated the showers, and therefore
will provide of methods to separate the signal observed from the
isotropic background.

With all these studies we will be able to apply old and new criteria
of \emph{gamma-hadron distrimination}, in order to find signals from
new sources or to get more information about the sources of gamma
radiation. To this topic is devoted the chapter \ref{chapter:ghsep}.

Finally, I tried to apply some of this knowledge to the study of the
possible detected signal from Mkn 501 (chapter \ref{chapter:mkn501}),
to end up with some conclusions.

With this work we try to make a realistic analysis of simulated data,
in order to show up new methods of \emph{gamma-hadron separation}.
Moreover, we try to give a clear impresion of how important is the
study of cosmic-rays and gamma-rays sources. With this investigation
we try to extend the horizon of our knowledge in this new field.

\endinput
%
%% Local Variables:
%% mode:latex
%% TeX-master: t
%% End:

%%EOF
