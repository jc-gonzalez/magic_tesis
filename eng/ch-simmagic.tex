%%%%%%%%%%%%%%%%%%%%%%%%%%%%%%%%%%%%%%%%%%%%%%%%%%%%%%%%%%%%%%%%%%%%%%%%%%
%%
%%  ch-simmagic.tex
%%
%%  Created: Fri Oct 10 14:24:37 1997
%%  Author.: Jose Carlos Gonzalez
%%  Notes..:
%%          
%%-------------------------------------------------------------------------
%% Filename: $RCSfile$
%% Revision: $Revision$
%% Date:     $Date$
%%%%%%%%%%%%%%%%%%%%%%%%%%%%%%%%%%%%%%%%%%%%%%%%%%%%%%%%%%%%%%%%%%%%%%%%%%%
%

\chapter{Simulation of the detector: \MAGIC}
\label{chapter:simmagic}

A program for the simulation of the detector is necessary in order to
study the response of the system (the \Cerenkov telescope) to the
input data (\Cerenkov light from the incoming electromagnetic and
hadronic showers). Several small simulation codes have been written.
In this chapter we present the main programs used for this topic:
namely, the simulation codes \reflector and \camera.

The general flow of the simulation process is outlined in Fig.
\ref{fig:simprocess}. As it was already shown, the simulation of the
detector is splitted in two steps:
%
\begin{enumerate}
\item Passage of the generated \Cherenkov light through a simulation
  of the atmosphere, and reflection of this light by the multi-mirror
  reflector of \MAGIC
  
\item Pixelization of the light-sensitive camera, simulation of the
  trigger logic and the image analysis
\end{enumerate}
%
%In this chapter I will explain in some detail these two steps, and the 
%results obtained.

%------------------------------------------------------------
\section{Reflection of \Cherenkov light and collection in 
the focal plane}
\label{sec:reflcoll}

If one looks at the output of CORSIKA, one can find some data files
with information about particles and \Cherenkov photons in each of the
user selected observation levels. We concentrate on one observation
level, 2\,200\u{m} a.s.l., corresponding to the Roque de los
Muchachos, at La Palma, Canary Islands. In our investigations we are
interested in \Cherenkov light. Therefore, the programs developed for
the simulation concentrate in the analysis of this light produced in
the atmosphere by air showers.

The information saved in the output files for each \Cherenkov photon
was shown in Table \fullref{tab:cphstruct}, and consists, for each
\Cherenkov photon, of the 7-tuple $\{\lambda,x,y,u,v,t,h\}$, where:
%
\begin{center}
\begin{tabular}{cl}
$\lambda$ & Wavelength of the emitted \Cherenkov photon \\
$x,y$     & Position of the \Cherenkov photon in the horizontal\\
          & plane at the observation level \\
$u,v$     & Director cosines in X and Y directions of the \\
          & incoming photon\\
$t$       & Time elapsed since the first interaction of the primary \\
          & until the photon was emitted\\
$h$       & Height of production of the photon\\
\end{tabular}
\end{center}

The quantities $x,y,u$ and $v$ are measured in the coordinate system
of the observation level (I will call this system the \emph{observer's
  coordinate system}). However, we use two more systems in our
simulation. In total we use these three coordinate systems (see
Fig. \ref{fig:coordsys}):
%
\begin{Ventry}{$S" \equiv O"X"Y"Z"$:}
\item[$S \equiv OXYZ$] The observer's coordinate system, where all
  the variables from the output of CORSIKA are measured
  
\item[$S' \equiv O'X'Y'Z'$] The system of the global frame of the
  telescope (we have that $O\equiv O'$)
  
\item[$S" \equiv O"X"Y"Z"$] The local system of the single mirror
  element to which each incoming photon hits
\end{Ventry}

\simprocessfig
%
In the simulation of the reflection, we obey the following algorithm:
%
\begin{enumerate}[1.]
  
\item Read parameters of the simulation, allocate memory, initialize
  different variables, and open the files.

\item For each \Cherenkov photon in every shower to be analyzed,
  perform the following steps:
  
  \begin{enumerate}[\theenumi.1.]
  \item Calculate the transmittance of the atmosphere, and test
    if the photon passed the absorption.
    
  \item Apply the reflectivity of the mirrors, possible dead zones
    around single mirrors and simulation of mirror imperfections.
    
  \item If the photon survives all these steps, its way to the camera
    plane is traced and the photon parameters are saved in the output
    file.
  \end{enumerate}

\item Go back to 2.1 while there are photons/showers left.

\end{enumerate}

Let's explain briefly the most important steps.

%%------------------------------------------------------------
\subsection{Simulation of the Atmospheric Absorption}

In order to simulate the atmospheric attenuation suffered by the
Cherenkov light emitted in the development of an atmospheric shower,
an appropriate estimation of the quantities involved must be done.
These quantities are the \emph{true vertical height} (defined for a
point in the atmosphere as the vertical distance between this point
and the sea level) and the \emph{air mass} (defined as the density of
air traversed by the photons). Since CORSIKA simply gives the height
of emission of a photon in the vertical of the observer, we have to
calculate the values of these variables in our simulation programs.
They are of special importance in the case of High Zenith Angle (HZA)
studies.

%\coordsysfig
\Systemsfig %% in landscape

\afterpage{\clearpage}

%%------------------------------------------------------------
\subsubsection{Calculation of the true vertical height}

For the simple simulation of HZA using our plane-parallel-atmosphere
version of CORSIKA, as well as for the calculation of the true
vertical height of a photon (at its emission point) and the airmass
(AM) it traverses, a very simple estimation can be done.

We will assume that our observer is located at point A (see Fig.
\ref{fig:geomview}), at a heigh \ho above sea level (a.s.l.).  Our
observer is looking at a source (which eventually emits gamma rays),
under a Zenith Angle $\theta$. Assume one of these gamma rays just
arrived at the top of Earth's atmosphere, and developed an gamma ray
induced atmospheric shower. (For simplicity, we will assume that the
trajectory of the photon will cross the observation level at point A.)

I will use the following notation (from Fig. \ref{fig:geomview}):

\begin{tabular}{ll}
A & Point where the observer is located \\
O & Point at sea level, in observer's vertical \\
P & Point of emission of the photon \\
\ho & Height of the observation level \\
\hc & Height given by CORSIKA, in the vertical of the observer\\
\hv & Height in the vertical of the photon\\
$R \equiv R_\oplus$ & Earth's radius\\
$\theta$ & Zenith Angle in observer's level\\
$\omega$ & Zenith Angle at sea level\\
$\alpha$ & Angle from Earth's center between A and P\\
\end{tabular}

%\geomviewfig

If $\ho \neq 0$ nor negligible with respect to \hc, and $\theta \simeq
\omega$, the expression that gives us the value of the \emph{true
  vertical height} at P in the vertical, \hv, is:
%
\hveq
%
In the case of $\ho \simeq 0$ or negligible with respect to \hc, this
becomes:
%
\hvapproxeq

%------------------------------------------------------------
\subsubsection{Pure geometrical estimation of the air mass}

In the simplest approximation, the air mass factor (AM, the relation
between the optical travel path at a given zenith angle and the
optical path at the vertical) can be estimated from geometrical
consideratins. Using this simple approach, the air mass factor will
be:
%
\mgeomeq

%------------------------------------------------------------
\subsubsection{Calculation of the air mass using an 
exponential density model}

The above expression in Eq.\eqref{eq:mgeom} uses just geometrical
arguments to estimate the ratio represented by $m$. We would like to
go one step beyond, by introducing an exponential density model. In
our simulations we are using an atmospheric model represented by 4
exponentials and a linear function at the top of the atmosphere.
Therefore, although the expression that we want to get is not perfect,
nevertheless it will be closer to the reality.

We will use then the following density model:
%
\denseq
%
where $h$ is the height a.s.l., and \Hs is the \emph{scale height} of
the atmosphere. In the Earth, and for not very big heights, this
results to be approximately $\Hs = 7.4\u{km}$. With this model, the
optical path (ignoring refraction) can be written (with the
approximation that $h/R \simeq 0$) as:
%
\optpathredeq
%
Note that we already wrote explicitly that we want to calculate the
travel path for a Zenith Angle $\theta$ for vertical heights between
\ho and \hv. Our \emph{air mass} $\mathcal{AM}$ is defined
as\footnote{The \emph{Airmass} in Optical Astronomy is usually defined
  as the relation $\mathit{Airmass} \equiv
  {I(\theta;0,\infty)}/{I(0\deg;0,\infty)} $}:
%
\AMdefeq
%
Therefore, integrating we have:
%
\AMfulleq
%
where $\mathrm{erfc}$ is the \emph{complementary error function}. An
approximation for low zenith angles, or more technically, for
moderately large $X\equiv\sqrt{{R\,\cos^2\theta}/{2\,\Hs}}$ is
%
\AMapproxeq

\paragraph{Correction for refraction.} We have defined $R \equiv
R_\oplus$, the radius of the Earth. A simple correction for refraction
can be obtained by taking instead
%
\refracapproxeq

%------------------------------------------------------------
\subsubsection{Comparison between different approximations}

First, we would like to compare the true expression for the
calculation of the \hv with the one obtained by assuming that \ho is
small enough, $h_{\mathrm{v,approx}}$. For this purpose we calculate
the expression
%
\diffhveq
%
for $\ho=2.2\u{km}$ (height of the future location of MAGIC), for
different Zenith Angles and different original height \hc (see Fig.
\ref{fig:hvdiffs}). We can see that the difference increases with the
zenith angle, as expected, although it still remains well below 5\%
for angles up to 85\deg.

\hvdiffsfig
%
The next thing we want to compare is the geometrical expressions for
the air mass, namely $m$ and $m_{\mathrm{simple}}$ (the simple
approximation using $h_{\mathrm{approx}}$), with the more elaborated
calculation using the exponential density model, $\mathcal{AM}$. We
will include in the last expression the simple correction for
refraction given in Eq.\eqref{eq:refrac}. This comparison, for three
value of \ho (0\u{km}, 4\u{km} and 2.2\u{km}), is shown in
Fig. \ref{fig:AMcomp}. We have used in this case two values for the
vertical height, $\hv=5\u{km}$ and $\hv=100\u{km}$. (The function
$\mathrm{sec}(x)$ is shown for illustration.)

\AMcompfig

\afterpage{\clearpage}

%------------------------------------------------------------
\subsubsection{Calculation of the atmospheric transmission}

Once we determined the \emph{true vertical height} of production of a
photon, \hv, and its corresponding \emph{airmass} $\mathcal{AM}$, we
can calculate the atmospheric transmission. Three effects have been
included in the final calculation of this quantity:

\paragraph{Rayleigh scattering.} The molecules in the atmosphere
produce scattering of photons. This effect is strong a strong function
of the wavelength $\lambda$ of light, being more important for shorter
wavelengths. The transmission coefficient due to Rayleigh scattering
is:
%
\Rayleigheq
%
where $\tau_{i=1,2} = \tau_0 \, \exp(-h_i/H_{\mathrm{S}}) \sec\theta$
is the \emph{slanted thickness} above a height $h_i$, in the case of
an exponential density profile (actually in our case we calculated
these values with the CORSIKA density model),
$\tau_{\mathrm{R}}(\lambda\!=\!400\u{nm}) = 2970\u{g/cm^2}$ is the
mean free path of the Rayleigh scattering in terms of thickness,
$\tau_0 = 0.00129\u{g/cm^2}$, and $H_{\mathrm{S}}$ is the mentioned
scale-height of the atmosphere. We pre-calculated the exponents for
different wavelengths.
  
\paragraph{Mie scattering.} Due to the aerosol (dust, water, vapour, 
smoke, \ldots) particles suspended in the air, its effect is bigger at
lower heights.  It depends strongly on the size of these particles.
For an exponential density profile, the transmission coefficient is
%
\Mieeq
%
where $l_{\mathrm{M}}(\lambda\!=\!400\u{nm}) \simeq 14\u{km}$ is the
mean free path for the Mie scattering, and $h_{\mathrm{M}} \simeq
1.2\u{km}$ is the scale-height for the aerosol distribution. We have
pre-calculated the exponents for different wavelengths, although the
dependence on the wavelength is very weak.
  
\paragraph{Ozone absorption.} This effect is very important in the
range 280--340\u{nm} and for low energy showers, in which most of the
\Cherenkov light emission is located where the density of ozone is
high (between 20--30\u{km}). For this we have first estimated the
extinction in magnitudes per unit of air mass:
%
\amagneq
%
where $k(\lambda)$ is the absorption coefficient in \u{cm^{-1}}, and
$T(\hv)$ is the total ozone concentration for a typical Tropical
region (we coose this because was very similar to the actual profile
in La Palma), above the vertical in each position, in \u{atm cm}. The
coefficient 1.11 was chosen to correspond to Elterman's optical
thickness for ozone at 320\u{nm}\cite{Elterman:book}.

With this value, we can estimate the extinction in magnitudes per air
mass between two height, and using the Pogson law we can determine the
transmission coefficient, which results in:
%
\Ozoneeq

In consequence, the total transmission coefficient will be:
%
\TotalTransmissioneq

%%------------------------------------------------------------
\subsection{Reflection in the mirrors}

If a photon survived to the atmospheric attenuation, we apply the
reflectivity of the mirrors and some other minor effects, like
possible dead zones, and calculate the reflected trajectory of the
photons, and its position in the camera plane. The reflectivity has
been assumed constant in the interval of wavelengths
$[290\u{nm},600\u{nm}]$, our range of interest.

In general, the telescope is looking in any direction, and a photon is
coming from another different direction (of course, if it comes
outside the limits of the field of view of the telescope, it will not
be reflected back to the camera). We associate a coordinate system to the
telescope (see Fig. \ref{fig:coordsys}), $S'$. The system in the
observer's level is $S$, and it is in this system where we have the
parameters of the trajectory of the photon, the direction vector,
$\mathbf{r}$, and the incident point in the observation level,
$\mathbf{x}$. We must therefore make a change of coordinate system:
%
\rotateAeq
%
where \thetaCT and \phiCT are the angles that define the telescope
direction, and $\Omega(\theta,\phi)$ is defined as:
%
\omegaAeq
%
Here $R_x(\alpha)$ it's a positive rotation of the space around the
axis $X$ of $\alpha$ degrees (the same for $Y$ and $Z$). Therefore,
the expression for the matrix of change of coordinate system is:
%
\transformAeq
%
with $\theta=\thetaCT$ and $\phi=\phiCT$. Once we are in the system
$S'$, we can calculate in which mirror hits the photon. In our
simulation we have 920 square mirrors of $50\times 50\u{cm^2}$,
located in realistic positions with the proper inclination for
focussing in the camera. There are small gaps (about 5\u{mm}) between
to adjacent mirrors, and the photon could be lost through one of them.

If the photon indeed hit a mirror, we calculate the reflected
trajectory from the point were the photon arrives to the mirror,
$\mathbf{x}_{\mathrm{cut}}$. For this purpose we must once more change
our coordinate system, to that of the mirror. This means now not only
a rotation, but an additional traslation to the center of the mirror.
The transformation is:
%
\transformBeq
%
where $\mathbf{x}_{\mathrm{m}}$ and $(\thetam,\phim)$ give the
position of the center of the mirror in the system of the telescope,
$S'$, and the direction of the normal vector in the center, in the
same system.

With these $\mathbf{x}"$ and $\mathbf{r}"$, we can calculate the
reflected trajectory, given by a new vector $\hat{\mathbf{r}}"$, and
therefore the final position of the reflected photon in the camera
plane.

\subsection{Focussing the mirrors}

Every single mirror in the global frame is pointing towards the
optical axis of the whole telescope, with an angle and a focal length
determined by two factors:

\begin{enumerate}[i.]
\item The position of the mirror element in the dish, and
\item The focal length of the whole system
\end{enumerate}

This later parameter, the \emph{overall focal length}, is fixed to
$F=1700\u{cm}$. 

Let's assume we focus our telescope to \emph{infinity}.  The image
plane of a parabolic mirror coincides with the focal plane only for
objects at infinity. When the distance of the light source is finite,
to get the best image one has to move away from the focus. In our
case, the bulk of our data will be showers which have their maximum
development at around $12\u{km}$ a.s.l., this means, a height of
$\hv\simeq 10\u{km}$ above our observation level.  Therefore, we must
afterwards put the telescope out of focus, i.e. move the camera a
little bit. This effect is called \emph{collimation}. In Fig.
\ref{fig:collimation} we can see a geometrical construction which help
us to understand this behaviour.

For simplicity, let us assume that the telescope is pointing to the
Zenith. A single mirror is located at a distance $r$ of the optical
axis of the telescope, $\overline{\mathrm{OP}}$. This mirror is fixed
to the global, parabolic shape of the reflector, and therefore at a
height (with respect to the center of the telescope $\mathrm{O}$) of
%
\zparabeq
%
with $F\equiv\overline{\mathrm{OF}}$ the already mentioned
\emph{overall focal length}. The straight line
$\mathbf{n}\equiv\overline{\mathrm{PQ}}$ is the normal to the mirror
surface in the center, being $\mathrm{P}$ the intersection point
between this normal and the optical axis of the telescope. The line
$\mathbf{r}$ shows the trajectory of a photon coming from the
infinity. The reflected photon will cross the optical axis of the
telescope at $\mathrm{F}$, by definition. But, as it has been
mentioned, the bulk of the data will come from heights around
$\hv\simeq 10\u{km}$ above the observation level, i.e., the trajectory
of a photon hitting in the center of the mirror will be $\mathbf{s}$.
The angle between this and the ``original'' trajectory is
$\widehat{\mathbf{rs}}\equiv\zeta$. The reflected photon will cross
the optical axis instead in $\mathrm{C}$.

\collimationfig
%
Therefore, in order to have well focussed our system at 10\u{km} above
the observation level, we shift the camera from $\mathrm{F}$ to
$\mathrm{C}$. From Fig. \ref{fig:collimation} we get the relations:
%
\relationsAeq
%
and
%
\relationsBeq
%
Therefore, the displacement to apply is given by the expression:
%
\collimationeq
%
In our case, this displacement is in the range
2.9--3.3\u{cm}$\approx$3\u{cm} (see Fig. \ref{fig:colldelta}),
depending of the mirror position. We would say then that this new
position of the camera is the \emph{optimal-focus position}.

However, we could have focussed our telescope from the beginning to
this height of 10\u{km}. By doing so, we avoid any shifting of the
camera. Note that, in any case, we are interested on having our
telescope focussed at 10\u{km}, and therefore \emph{the stars in the
  field of view will be slightly defocused}.
% Note that, in any case, we are interested on having our telescope
% focussed at 10\u{km}, and therefore \emph{the stars in our field of
%   view will be out of focus}, and hence will form in our image larger
% spots than ni the case of a system focussed at infinity.

\colldeltafig

In Figs.\ref{fig:hlfivekm}.a--\ref{fig:hlinf}.a we show the image of a
point-like source of light, located at different heights \hv in the
axis of the telescope (pointing to the Zenit) when the telescope is
focussed \emph{at infinity}, and for different positions of the camera
plane. We can see that when the source is at infinity and the camera
plane is the focal plane, we get the smallest spot. However, in the
case of a source of light at $\hv=10\u{km}$, we see that the best
focus position is at 1703\u{cm}, as calculated above. The green
hexagon represents the central pixel. We also observe that 100\u{km}
is virtually infinity for our purposes. The reflectivity taken here is
100\%, a perfect, mathematical reflection has been done, and any
atmospheric effect has been neglected.

\hlfivekmfig
 
\hltenkmfig
 
\hltwelvekmfig

\hlhundredkmfig
 
\hlinffig

\Mspotsfig

\afterpage{\clearpage}

In Figs.\ref{fig:hlfivekm}.b-\ref{fig:hlinf}.b we see something
similar, but now the system is directly focussed at 10\u{km}.
Therefore, for a light located at this height, we get the best focus
position at the nominal focal plane position, 1700\u{cm}.  In these
figures, the scale is normalized to the maximum concentration of light
(light at infinity and projection plane at 1703\u{cm} for the (a)
cases, and light at 10\u{km} and projection plane at 1700\u{cm} for
the (b) cases).

In Fig. \ref{fig:Mspots} we see the reflected image of a letter
\texttt{M}, enclosed in a square, hypothetically projected by a lamp
located at our reference position 10\u{km}, and with a size on the
ground comparable to the dish of \MAGIC. Again, the minimal spot
position is at 1700\u{cm}. We see how the image gets inverted when we
move the projection plane from the front to the back of the focal
plane.

Another effect, already taken into account in the calculation of the
positions, orientations and focal distances of the mirrors, is the
following: the best \emph{focal surface} is no longer a plane. In
general, it will be a surface of revolution, similar to a paraboloid.
For a spherical mirror of radius of curvature $R_{\mathrm{c}}$ the
focus lies at a distance $R_{\mathrm{c}}/2$ from the center of the
mirror, for an incident parallel beam of light coming parallel to the
optical mirror axis (incident angle of 0\deg).  If the incident angle
is different from 0\deg, the smallest blur size is at somewhat shorter
distances than $R_{\mathrm{c}}/2$. This is the so called
\emph{shortening of the focal length}, which depends of the incident
angle of the light.

In the case of the tessellated mirror of \MAGIC and a parallel beam of
light (incident angle of 0\deg with respect of the optical axis of the
reflector), all the single mirror elements will receive the light at
angles different from 0\deg. In particular, the distance from the
mirror where we get the smallest blur size will decrease when we take
mirrors with increasing distances to the optical axis of the system.
Therefore, we must \emph{increase} the focal length of the mirrors.

Having this in mind, all the mirrors have been grouped in a total of
eight sets or \emph{zones}, each of them characterized by a constant
focal length. This zones divide the dish in eigth concentric
pseudo-rings (see Fig. \ref{fig:zones}). As it has been mentioned,
each of the focal lengths for these zones $f_i$ must be larger than
$F$, in order to comply with the fixed focal length of the system. The
values of these focal lengths $f_i$ are summarized in Table
\ref{tbl:zonefocals}.

\zonefocalstbl

\thetaspotsfig

\afterpage{\clearpage}

We used a \emph{perfect} system in figures Fig. \ref{fig:hlfivekm} to
\ref{fig:Mspots}: reflectivity 100\%, no imperfections in the mirror
surface, orientations of the mirrors or focal distances, and no
atmospheric absorption. If we apply the effect of a non-perfect
reflection, possible axis-deviations, imperfections in the mirrors and
the effect of the atmosphere, we will get the results shown in Fig.
\ref{fig:spotsh}.

%%------------------------------------------------------------
\subsection{Plate-scale in the camera of \MAGIC}

We are now interested on the \emph{plate-scale}, that is, the relation
between linear distances in the camera plane and angular distances in
the sky. For this, we change the Zenith Angle of our point-like source
(initially 0\deg) to several values. In Fig. \ref{fig:thetaspots} we
can see the results. For completeness, we show the effect of the
height of the source, although we are only interested in the height
$\hv=10\u{km}$ above the detector --- the one used for the
optimal-focus position.

The value taken for the plate-scale in the camera plane of \MAGIC is:
%
\platescaleeq

%%------------------------------------------------------------
\section{Detection in the camera}

At the end of the first stage in the detector simulation process, we
will have, for each shower, a bunch of photons in the camera plane.
The next stage will be the so called \emph{pixelization}.  This
process, among others, is performed by the program \texttt{camera}. In
few words, the steps taken in this program are:

\begin{enumerate}
\item Pixelization. Each photon is assigned to a light sensor
  
\item Inclusion of Light Of the Night Sky (LONS)
  
\item Simulation of Quantum Efficiency and other effects (light
  guides, plexiglas protective window --- if any --- and the first
  dynode collection efficiency)

\item Simple simulation of the electronic chain

\item The trigger logic simulation
  
\item Parametrization of images in the camera
\end{enumerate}

\zonesfig

%\zonesfocalsfig

%%------------------------------------------------------------
\subsection{Pixelization}

In Fig. \ref{fig:triaxis} we can see the definition of the tri-axial
frame used for the numbering\footnote{We are using the so called
  \emph{spiral-numbering}, where the first pixel is the one in the
  center, and then for each ring, starting from the innermost one, we
  give sequential numbers in counterclockwise order.} of the pixels.
The pixels will have a hexagonal shape. If the edge-to-edge width of a
pixel is $w$ and the corner-to-corner width is $\hat{w}=(2/\sqrt{3})
w$, then the unit in each of the three axes is $1.5\times (\hat{w}/2)$
(in the implementation we take a scaling factor such that $w=1.0$).
With this definition the coordinates of the pixels are, for example
for the first seven pixels shown in the Fig. \ref{fig:triaxis}:
%
\begin{center}
  \begin{tabular}{rc}
    1 :& ( 0, 0, 0) \\
    2 :& ( 1,-1, 0) \\
    3 :& ( 1, 0,-1) \\
    4 :& ( 0, 1,-1) \\
    5 :& (-1, 1, 0) \\
    6 :& (-1, 0, 1) \\
    7 :& ( 0,-1, 1) \\
  \end{tabular}
\end{center}

The primary question can be expressed in the following terms:
\emph{What is the pixel number $k$ (in spiral-numbering) of the PMT
  where a photon falls in?}.  Following \cite{Fu:hexgrid}, we get an
elegant solution to the problem. Indeed, what we think of as an
uniform hexagonal grid, can be isometric projection\footnote{An
  isometric projection is an orthographic, i.e., non-perspective,
  projection onto the $x+y+z=0$ plane.} of an infinite grid of unit
cubes whose centers satisfy the equation
%
\planeeq
%
In this particular case, the problem of determining which hexagon
contains a given point becomes the problem of which cube contains a
point.  Additionaly, only the transformation from the plane to the
cube grid and vice versa are needed.

With this approach, the the formula that gives the answer to our
question is:
%
\hexgrideq
%
where $\mathbf{r}$ is the position of the photon in the camera plane,
$r_i$ is the $i$-th component of the vector $\mathbf{r}$ and
$\mathcal{R}[\mathord{.}]$ is a rounding function.

\axesfigs

%\triaxisfig

With this definition of the axes, we pre-calculate the coordinates
$(x',y',z')$, in order to use them afterwards in the simulation code.
Of course, only two of these coordinates are independent, since we
have the constraint \eqref{eq:plane0}\footnote{Note that the algorithm
  is symmetric in $X'$, $Y'$ and $Z'$, since the axes are symmetric.}
  
Now, two of the coordinates (only two are independent), say
$\{x',y'\}$, are used to identify each pixel in an array $A$, which
has in its cells the number of pixel to which those coordinates
belong.

This algorithm, however, does not have {\itshape a priori} information
about the position of the center of the pixels, information that is
indeed needed in the simulation code. So, we have to calculate in
addition the coordinates of the centers of the pixels. For this
purpose, yet another different coordinate system is used, the one
shown in Fig. \ref{fig:biaxis}. The units of the axes X and Y are
millimeters, and the units of the axes I and J are both two times the
apotheme of one pixel ($2 \times a$).

%\biaxisfig

With this definition of the axis, we use the intermediate
$(i,j)$-coordinates, in order to use them to calculate the
$(x,y)$-coordinates of the centers. To get these $(x,y)$-coordinates,
we just apply the following change of system:
%
\bitoeucleq

\afterpage{\clearpage}

%%------------------------------------------------------------
\subsection{Contribution of the Light Of Night Sky}

The amount of diffuse Light Of Night Sky (LONS) in La Palma, at
the site where \MAGIC will be locate, has been measured
\cite{Razmick:nsb}, leading to a flux 
%
\LONSeq
%
in the wavelength sensitivity region 300--600\u{nm} of a bialkali PMT.
Let's estimate the number of photons that the LONS contribute to each
pixel in the camera of \MAGIC. For this calculation, the following
parameters have to be taken into account:
%
\begin{center}
\begin{tabular}{lrl}
Mirror surface & $S_{\text{mirror}}$ &$= 230 \u{m}^2 $ \\
Reflectivity & $R$ &$= 85\% $ \\
Light guide efficiency & $\epsilon_{\text{l.guide}}$ &$= 90\% $ \\
Transmittance of PMT window & $\epsilon_{\text{window}}$ &$= 95\% $ \\
First dynode collection efficiency &
             $\epsilon_{1^{\mathrm{st}}\text{dyn.coll.}}$ &$= 90\%$ \\
Pixel angular size & $\theta_{\text{1pixel}}$ &$= 0.1^\circ$ \\
Mean \QE folded with LONS & \QElons &$\sim 13\% $ \\
\end{tabular}
\end{center}
%
With this, we have that the half angular size of a pixel is
$\theta_{\frac{1}{2}\text{pixel}} = 0.05^\circ$, and the solid angle
for each pixel $\Delta\Omega =
2\pi(1-\cos\theta_{\frac{1}{2}\text{pixel}}) = 2.4\cdot 10^{-6}
\u{sr}$. Then, the \emph{mean number of photons} arriving at the
entrance of the pixel in $1\u{ns}$ is:
%
\Nineq
%
and using the mean \QE for the LONS, \QElons:
%
\Ninbiseq
%
We have used for this calculation PMTs as detection devices. If we use
IPCs, with an assumed mean $\QE_{\text{\,IPC}} = 45\%$, then this number
becomes:
%
\NinbisIPCeq
%
As an example, if we use then an integration time window of $\Delta
T=5\u{ns}$, we arrive at a \emph{mean contribution of LONS per pixel}
of (back to the case of PMTs):
%
\LONStimeeq

%%------------------------------------------------------------
\subsection{Simulation of the response of a PMT}

The simulation of the response of a PMT, despite of its complexity,
depends mainly on a well known set of effects, namely:

\begin{enumerate}[a.]
\item the Quantum Efficiency (\QE) of the PMT, which is a function of
  the wavelength of the incident photon,
  
\item the natural fluctuations of this QE for any given fixed
  wavelength,

\item the first dynode response, 

\item the afterpulsing, 
  
\item the single photoelectron response
\end{enumerate}

Let's study the process of conversion of photons (\Cherenkov photons,
photons comming from the LONS, and starlight) into number of
photoelectrons after the photocathode of the PMT.

%------------------------------------------------------------
\subsubsection{Fluctuations}

Before we try to understand the process of conversion of \Cherenkov
photons into photoelectrons, let's try to identify the possible
sources of fluctuations.

The number of incoming \Cherenkov photons fluctuates from shower to
shower. Even for a fixed energy of the primary particle which
generated the shower, the probabilistic generation of secondary
particles, the fluctuations in the height of the first interaction
(and hence in the height where the maximum particle generation is
achieved) and the random generation of \Cherenkov photons by charged
particles will lead to a fluctuating number of Cherenkov photons.

Let's call \Nphot the \emph{input number of photons in the
  photomultiplier tube}. Now we want to simulate a measurement of
\Nphot leading to a certain current $I$, or a given amount of
electrons in the anode, $N_{\mathrm{e}}$. The \Cherenkov photons
hitting the photocathode will produce a certain number of
photoelectrons. This process is probabilistic, and depends on the
Quantum Efficiency, \QE, of the photocathode.  The \QE depends not
only on the wavelength $\lambda$ of the incident photon, but also on
the point on the photocathode where the photon hits.  Additionally,
the measured \QE (for a given $\lambda$) of a PMT is just an average
value over many photons (of this $\lambda$). This means that the \QE
itself must be seen as a fluctuating term.

After the photoelectrons are emitted from the photocathode, a certain
number of them will arrive at the first dynode. Of course, this
depends on the place where the photoelectron comes from, and on the
design of the PMT itself. This dependency is given by the so called
\emph{first dynode collection efficiency}. Different measurements lead
to a value of about 90\%, for our chosen PMT. After hitting the first
dynode, each photoelectron can liberate a typical number of 6
electrons, but this number also fluctuates.

We just started the cascading process. After the first dynode comes
the second, where we again have a multiplicating term (and hence an
efficiency term). This multiplicating process continues until we reach
the anode, where we finally have a big number of electrons. This
number depends on the gain and the voltage of our PMT.

This cascading process can be simulated by using the so called
%\emph{single photoelectron response} or 
\emph{single electron spectrum} (SES) of the photomultiplier, which is
nothing but the distribution of the output that we get from the
photomultiplier for a single photoelectron release by the
photocathode. This depends also on the high voltage applied to the
PMT. Therefore, by using this distribution, we can simulate the output
of each single photoelectron, superpose all these output signals, and
we will get at the end a realistic response of the PMT to the incident
light.

%------------------------------------------------------------
\subsubsection{Simulation of the Quantum Efficiency}

We have assumed that \emph{a single photon cannot produce, for
  whatever processes inside the photocathode, more than one
  photoelectron}. Therefore, for each single photon, the generation of
a photoelectron is what is called a \emph{Bernoulli process}. For such
processes, we can have only two outcomes: \emph{success} (with
probability $p$), or \emph{fail} (with probability $1-p$). For a
given, fixed number of incoming photons, the generation of
photoelectrons is a good example of a \emph{binomial} process. Let's
forget first about the dependency of \QE with the wavelength.  Let's
assume, therefore, that we have a monochromatic bunch of photons, and
call call $\QEo=\QE(\lambda_0)$, the Quantum Efficiency (the
``average'' value, obtained by measurements) of the photocathode at a
fixed wavelength $\lambda_0$.

\noutphotfig

We can see that each incoming photon have a certain probability \QEo
of generating a photoelectron. This can be simulated by using a
\emph{uniformly distributed random variable} $X$ in the interval
$[0,1]$ for each photon: whenever the value of $X$ is smaller than
\QEo, we assume the photon did generate a photoelectron; otherwise, it
didn't:
%
\begin{enumerate}
\item Take a photon.
\item Generate uniform random number $r$ in $(0,1)$.
\item If $r < \QEo$, a \phe is generated, else go to 1.
\item Are there more photons left? If yes, go to 1.; else Stop.
\end{enumerate}
%
After following this algorithm \Nphot times, we will have, \emph{on
  average}, a number of photoelectrons given by
%
\nmeaneq
%
The words \emph{on average} mean simply that we will not get, from a
single realization of our experiment, exactly a number \Nmean of
photoelectrons (at least, not always). If we repeat this experiment
many times, we will get instead a distribution of numbers \Ntrial,
corresponding to the number of photoelectrons produced. This
distribution, given the \emph{binomial} nature of the process, will
result in a \emph{binomial distribution}. The mathematical expresion
for this is:
%
\binomeq
%
where $\mathcal{P}_{\mathrm{Binom}}(r)$ is the probability to obtain
$r$ successes out of $N$ independent trials, each of them with only
two possible outcomes: success (with probability $p$) or failure (with
probability $q=1-p$). It can be shown that the expectated value of the
number of successes is $\bar{r}=\sum r\mathcal{P}(r) = N p$, which is
not surprising. For this distribution we have also that $\sigma^{2}
\leq \bar{r}$, where the equality holds only for $p=0$. In general the
variance is smaller than the mean. This is so because the upper limit
imposed on $r$ (which cannot be larger than $N$) reduces the spread of
the $r$-distribution.

\varquotfig

\noutqesmallfig

The main use of the binomial distribution is in the limits:
%
\begin{itemize}
\item if $p\rightarrow 0$, $N\rightarrow\infty$, but $Np=\mu$ (constant),
  Binomial $\longrightarrow$ Poisson, and
\item if $p=const.$, $N\rightarrow\infty$, Binomial $\longrightarrow$
  Gaussian
\end{itemize}
%
In our case, $N=\Nphot$, $p=\QEo$ and $Np=\Nmean$.  This means that,
\emph{for \Nphot large and in the case of $\QEo\rightarrow 0$} (more
generally speaking, for \QEo small) the distribution of the number of
outcoming photoelectrons \Ntrial will be very similar to the
distribution obtained by using the mean number of photoelectrons
\Nmean as the mean value of a Poisson distribution. More clearly, the
requierements of \Nphot large and \QEo small are needed if we want to
use this procedure, what I will call the \emph{Poisson approach}.

The use of the \emph{Poisson approach} is very attractive because you
don't need to work with separate photons. Moreover, some times it is
the most direct approach: you can get your photons at the entrance of
the PMT in bunches of several tens of photons. In this case it's more
comfortable to use mean values and the corresponding Poisson
distribution: we are sure that $N$ is going to be large enough. But we
still have to be sure that our \QEo is small enough to allow us to use
this approach. Fortunately, this is normally the case.

In order to show the possible deviations that we can get by using the
\emph{Poisson approach}, a simple simulation has been done: a
pre-fixed number of photons \Nphot was sent to a hypothetic
photocathode, where we simulated the Bernoulli process of the
production of a photoelectron by using the simple algorithm shown
above.  The \QEo of the photocathode was also fixed, and we counted
the number of photoelectrons emerging from it \Ntrial.  In addition,
using the number of incoming photons and \QEo, we estimated the
average number of photoelectrons \Nmean we should get, and then
followed the \emph{Poisson approach} to get the final number of
photoelectrons \Nrand.

In order to show how one can get significant deviations from the true
behavior by using the \emph{Poisson approach}, I performed simulations
by varying the number of incoming photons and the value of \QEo. The
results are shown in Figs. \ref{fig:distrib1}, \ref{fig:varwithp},
\ref{fig:varwithqe}. In Fig.  \ref{fig:distrib1} we see the effect of
using a \QEo different from $0$. For small values of \QEo, both
distributions of photoelectrons \Ntrial (solid line) and \Nrand
(dashed line) appear reasonably similar. However, as \QEo is diverging
from the hypothesis of \QEo being small, both distributions get more
and more different. In Fig. \ref{fig:varwithqe} we can see how
different these distributions are for different \QEo, by using the
quantity
%
\varquoteq
%
Although the mean value remains practically the same for both
distributions, indenpendently of the value of \QEo, the variance
$\sigma^2(\Nrand)$ increases much more than the variance
$\sigma^2(\Ntrial)$ with increasing\footnote{In the literature we can
  find that the Poisson approach gives reliable results only when
  $p\lesssim 0.05$, i.e. $\QEo\lesssim 5\%$.} \QEo. As it has been
mentioned earlier, this is just the result of the upper limit imposed
in \Ntrial (it cannot be larger than \Nphot). Finally, in
\ref{fig:varwithqe} one can see how the Binomial distribution of
\Ntrial becomes Gaussian when, for a fixed value of \QEo, we increase
the number of photons \Nphot.

%------------------------------------------------------------
\subsubsection{Simulation of Single Electron Spectrum}

This is just an application of the general procedure of obtaining a
series of random numbers following a user-defined distribution, called
the \emph{Acceptance-Rejection Method} (Von Neumann). Let's assume we
know the \emph{single electron spectrum}, $S(x)$, defined in an
interval $(a,b)$. This will be in principle a not normalized function
proportional to the normalized probability density function (p.d.f.)
of distribution we want to simulate. Then, choose a p.d.f. uniform on
the interval $(a,b)$. Find a constant $C$ such that $C$ times this
uniform p.d.f.  is everywhere greater than or equal to $S(x)$. This
scenario is shown in Fig. \ref{fig:vonneumann}.

\vonneumannfig

First, simulate a random value $x$ uniformly on $(a,b)$. Then generate
a $y$ on $(0,C/(b-a))$. The point $(x,y)$ will uniformly populate the
box shown in Fig. \ref{fig:vonneumann}. If $y\le S(x)$, we accept $x$
as then next value f the random number. If $y\geq S(x)$, reject $x$
and try again.  This method is very simple and has an efficiency
(fraction of values $x$ accepted) of
%
\vonneumanneffeq
%
(Note that if the function $S(x)$ has sharp peaks, the efficiency can
be very low; one can then use different constants for different
regions in the interval $(a,b)$.)

\timeresponsefig

If \Nphot photons hit the photocathode and produce \Ntrial then,
providing we know the fraction of then that hit the first dynode and
the single \phe amplitude response of the PMT, we can sample randomly
the distribution, add-up single amplitudes and obtain the resulting
amplitude of the signal. In order to simulate a realistic signal at
the output of our PMT, we should use the arrival time of every photon.
In addition, the PMT introduces a delay by itself. This delay follows
a distribution similar to a gaussian. The time between the arrival of
a delta-function light pulse and the time where the output signal
reaches its maximum is called \emph{electron transit time} (ETT). The
delay between the input light and the output signal is then governed
by the ETT and the \emph{transit time spread} (TTS, also called
\emph{transit time jitter}, the FWHM of the distribution of delays).
The output signal is characterized by its \emph{rise time} (time where
the output signal rises from 10\% to 90\% of the maximum amplitude).
Sometimes, instead of having the \emph{rise time},
$t_{\mathrm{rise}}$, we have the FWHM of the output characteristic,
$\approx 2.36\sigma$ (under the assumption of gaussian behavior). The
situation is schematized in Fig.  \ref{fig:timeresponse}. With all
this we can reproduce the response of our PMT to the bunch of incident
photons.

After all this chain of events, we can then introduce either the
trigger logic of our system, or any electronic circuit which modifies
this signal.

%%------------------------------------------------------------
\subsection{Simulation of the signal processing chain}

In Fig. \ref{fig:pixelreadout} it was shown the basic electronic chain
for every single pixel in the camera of \MAGIC. For an incident photon
the realistic signal produced is shown in Fig. \ref{fig:pulse}. After
the cascading process in the PMT, we will have an electronic current
$I=I_{\mathrm{A}}$, equal to the area A shown in this figure.  We use
a capacitive coupling (\emph{AC coupling}\footnote{The \emph{AC
    coupling} is the use of a special circuit, normally just a
  capacitor, to remove the static (DC) components from the input
  signal to the amplifier of an instrument, leaving only the
  components of the signal that vary with time.})
%%  [http://www.measurementsgroup.com/Guide/indexes/g\_index.htm].} 
of the light sensor to the measuring electronics. Because one can not
create charge in an electronic chain after a capacitor, the net
created charge must be zero. Therefore a fast pulse in one polarity is
followed by a slowly decaying, small amplitude signal in the opposite
polarity. The areas under curves of both polarities are equal
% Due to
% the \emph{AC coupling}\footnote{The \emph{AC coupling} is the use of a
%   special circuit to remove the static (DC) components from the input
%   signal to the amplifier in an instrument, leaving only the
%   components of the signal that vary with time [??].}
% %%  [http://www.measurementsgroup.com/Guide/indexes/g\_index.htm].} 
% used in the design of our electronic chain, after a single pulse there
% is a charge equal to the one coming from the pulse, which developes in
% an inverse current. The result is a positive pulse (positive voltage)
% after the incoming pulse dies 
(this is represented in Fig. \ref{fig:pixelreadout} by the area B). The
fading out of this ``remmanent pulse'' is characterized by the
constant $\tRC=RC$ of the circuit. This value is chosen, in the case
of \MAGIC, to be about $\tRC=5\,\mu\mathrm{s}$.

Since the rate of incoming pulses is typically higher than this \tRC,
and the pulses pile-up the \emph{zero-voltage level} will become a
baseline with a voltage \emph{greater that zero}. The baseline will
fluctuate randomly around a given level: this level depends on the
value of \tRC and the rate of incoming pulses.

\pulsefig

In the simulation programs, we have the option of using or not the
simulation of this fluctuating baseline, as well as the discretization
of the pulses in time bins. In case of using this later option, the
signal is integrated in a defined number of bins that depends on the
trigger gate. In case of not using this discretization, the whole
charge developed in the pulse is integrated. I will explain this in
detail more in section \ref{sec:triggerlogic}.

%%------------------------------------------------------------
\subsection{Trigger Logic}
\label{sec:triggerlogic}
%
First, we shall define a \emph{trigger area} in the camera, which is a
disc centered in the center of the camera, with a fixed radius. In our
investigations we used different sizes for this area. The estimation
of the optimal radius is as sketched in Fig. \ref{fig:trigangle}. We
are interested mainly in the low energy gamma-ray showers, i.e., with
primary energies in the range 10--100\u{GeV}. The maximum development
of these showers is at around 10--12\u{km} a.s.l., this means a height
of $\hv \simeq 8--10\u{km}$, when looking at the Zenith. On the other
hand, the \Cherenkov light pool extends up to about
$\rhump \simeq 100--120\u{m}$ from the shower axis: let's take
\rhump=110\u{m}.  This distance fixes an upper bound for the area
where we have the maximum detection efficiency for these showers.

\triganglefig
%
Looking from the telescope, and in the case of pointing to the Zenith
(see Fig. \ref{fig:trigangle}), this distance corresponds to an
angular distance calculated as
%
\trigradeq
%
The value of $\varphi$ is in the range $\simeq 0.7\deg-0.8\deg$ for
primaries in the energy range indicated. We chose the larger, more
conservative value of
%
\phitriggereq

We should not make this area too big, since the increase of the
$\gamma$-ray rate with an increased camera radius above some optimal
value would be marginal, while for the background rate we would get an
increment proportional to the \emph{square} of the increment in radius
for the trigger area. In Fig. \ref{fig:trigarea} one can see how big
in the camera is the area corresponding to $0.8\deg$ radius.

\trigareafig

Inside this \emph{trigger area} is where we define the different
trigger conditions. The most elementary trigger conditions are the so
called \emph{multiplicity triggers}:

\begin{description}
\item[\emph{Multiplicity trigger} \trigM{n}{q_0}:] An image is said
  to give trigger if there are at least $n$ pixels in the
  \emph{trigger area} with a charge of at least $q_0$ photoelectrons
  each.
\end{description}

\noindent
Another kind of trigger conditions are the \emph{next-neighbour
  triggers}:

\begin{description}
\item[\emph{Next-neighbour trigger} \trigNN{n}{q_0}:] An image is said
  to give trigger if there is a simply connected area of at least
  $n$ pixels in the \emph{trigger area} with a charge of at least
  $q_0$ each.
\end{description}

\noindent 
We should note that for fixed values of $n$ and $q_0$ we can define a
subset of \emph{next neighbour trigger} conditions, namely, those
which require that those pixels in the simply connected area, form
what is called a \emph{close pack}. We will denote this sub-class
of conditions with the symbol \trigNNc{n}{q_0}.

Yet another type of trigger conditions, a superset of the
\emph{next-neighbour} conditions, are the so called \emph{topological
  triggers}. For this type of conditions we must define a
\emph{topology} of pixels $S$, consisting of a set of pixels following
a given geometrical pattern.

\begin{description}
\item[\emph{Topological trigger} \trigTop{S}{q_0}:] An image is said
  to give trigger if there is a pattern of pixels coincident with the
  pattern given by $S$, with a charge of at least $q_0$ in each pixel
  .  (Of course, a different minimum number of photoelectrons can be
  defined for each pixel in $S$.)
\end{description}

The studies presented in this work have been done using the following
trigger conditions (the notation ``$a\mathord{:}b$'' denotes the range of
integer values from $a$ to $b$):
%%
\begin{center}
  \begin{tabular}[t]{rl}
    \trigNN{k}{q_0} & Next neighbour trigger with multiplicity \\
    & $k=3\mathord{:}5$, and charge per pixel $q_0$\\
    
    \trigNNc{k}{q_0} & Next neighbour in a closed packet trigger with \\
    &  multiplicity $k=3\mathord{:}5$, and charge per pixel $q_0$\\
  \end{tabular}
\end{center}
%%
In the simulation of the trigger logic, even for \trigNN{k}{q_0} and
\trigNNc{k}{q_0}, a pattern approach has been taken. For a given value
of $k$, and both for \trigNN{k}{q_0} and \trigNNc{k}{q_0} conditions,
we define the set of possible geometric arrangements of triggering
pixels (following the selected trigger condition) in a fictitious
group of seven pixels (a central pixel plus its seven neighbours),
where the charges of the pixels are supposed to be either $0$ or $\geq
q_0$. We give then a name to each of these sets of arrangements.
Finally, in the simulation process, only those patterns in the sets
selected by the user will be searched for in the camera. In
Fig. \ref{fig:trigpatt} we can see a graphical illustration of these
sets of patterns.

\trigpattfig

\MORE%%%%%%%%%%%%%%%%%%%%%%%%%%%%%%%%%%%%%%%%%%%%%%%%%%%%%%%%%%%%

%%------------------------------------------------------------
\section{Results}

After analyzing the raw output data, with the programs \reflector and
\camera, described above, we obtained the following results.

%%------------------------------------------------------------
\subsection{Trigger Efficiencies}

The trigger efficiencies for gammas and hadrons are calculated from
the ratio of triggered showers (for a given trigger) to the total
number of incident showers, for a given set of primary particle, range
in Zenith Angle and shower core distance.

For gamma-rays we have the following quantities:
%
\begin{center}
  \begin{tabular}{cl}
    $n_\gamma(r;E,\Theta)$ & number of \emph{incoming} 
    gamma showers arrived with primary energy $E$, \\
    & at an impact parameter $r$, from a zenith angle $\Theta$.\\
    $n_\gamma^T(r;E,\Theta)$ & number of \emph{triggered} 
    gamma showers arrived with primary energy $E$, \\
    & at an impact parameter $r$, from a zenith angle $\Theta$.\\
  \end{tabular}
\end{center}
%
In these calculations it is assumed that the telescope is tracking the
source of $\gamma$-rays, i.e. that the $\gamma$'s are coming parallel
to the telescope axis.  The \emph{trigger efficiency} (for a given
trigger set-up) for gamma-rays of energy $E$, a zenith angle $\Theta$
and an impact parameter $r$ is:
%
\effgammaeq

For the background of charged cosmic rays the situation is more
complex. Since this background comes isotropically, we have a new
degree of freedom, the \emph{angle off-axis}, $\delta$, of the
incoming shower with respect to the optical axis of the telescope. The
quantities involved are now:
%
\begin{center}
  \begin{tabular}{cl}
    $n_{\mathrm{cr}}(r,\delta;E,\Theta)$ & number of \emph{incoming} 
    cosmic ray showers arrived with primary energy $E$, \\
    & at an impact parameter $r$, from a zenith angle $\Theta$,
    and an off-axis angle $\delta$\\
    $n_{\mathrm{cr}}^T(r,\delta;E,\Theta)$ & number of \emph{triggered} 
    cosmic ray showers arrived with primary energy $E$, \\
    & at an impact parameter $r$, from a zenith angle $\Theta$,
    and an off-axis angle $\delta$\\
  \end{tabular}
\end{center}
%
Thus, \emph{trigger efficiency} (for a given trigger set-up) for
cosmic rays is given by:
%
\effcreq

Using different trigger patterns and different values of the Zenith
Angle $\Theta$, we studied the dependence of the \emph{trigger
  efficiency} with the angle $\delta$ and the impact parameters $r$.
The results are shown in Figs. \ref{fig:effg1}--\ref{fig:effh2}

%%------------------------------------------------------------
\subsection{Effective Collection Areas}

The definition of \emph{effective collection area} is the following:
%
\begin{description}
\item[{\bfseries Effective Collection Area}:] Hypothetic circular area
  in the plane perpendicular to the optical axis of the telescope
  where an incoming shower can be detected.
\end{description}
%
(see Fig. \ref{fig:collareaIdea}). In reality, the probability of
detection, for a given energy of the primary (and hence for a given
height of development of the shower), is a more or less smooth
function of the distance to the optical axis of the telescope, and not
only a matter of a yes/no decision. This concept is very useful to
compare different detectors and to calculate fluxes of sources.

\collareaIdeafig

For gamma-rays, the calculation of the effective collection area is
simple. Once we have the trigger efficiency, we calculate, for a fixed
energy $E$ of the primary, and a fixed Zenith Angle $\Theta$:
%
\Sgammaeq
%
(again, this is true for a flux of $\gamma$-rays parallel to the
telescope axis). The first integral comes from the symmetry around the
telescope axis, and will give just a factor $2\pi$.

For the background of cosmic rays, the situation is more complicated.
Since the cosmic rays (and hence the atmospheric showers developed)
arrive to the telescope isotropically distributed, we have to take
into account the acceptance of our system. Therefore, we have to
include two additional degrees of freedom in the calculations. We have
%
\Screq
%
The third integral comes now from symmetry around the telescope axis
of the arrival directions of the incoming showers (giving again a
factor $2\pi$). The fourth integral sums up all the contributions for
diferent off-axis angles $\delta$, up to an angle where the trigger
efficiency is zero (here expressed as $\delta_{\mathrm{max}}$).

Note that this $\hat{S}_{\mathrm{cr}}(E,\Theta)$ has now dimensions
[area$\times$solid angle]. This means that for any comparison of the
hadron effective collection area with the one for gammas we must
normalize the former to the solid angle.  The first impression is that
one has to divide the whole curve $\hat{S}_{\mathrm{cr}}(E,\Theta)$
vs. $E$ by the solid angle used when generating the incoming cosmic
rays, that is by
$\Omega_{\mathrm{ALL}}=2\pi(1-\cos\delta_{\mathrm{ALL}})$. This is not
wrong, as we will see, but it is useless for any comparison between
gammas and cosmic rays.  Indeed, by using a new arbitrarily large
$\delta'_{\mathrm{ALL}} \gg \delta_{\mathrm{ALL}}$, and hence
$\Omega'_{\mathrm{ALL}} \gg \Omega_{\mathrm{ALL}}$, we will at most
barely increase the value of $\hat{S}_{\mathrm{cr}}(E,\Theta)$, but
will make the quantity
$\hat{S}_{\mathrm{cr}}(E,\Theta)/\Omega'_{\mathrm{ALL}}$ arbitrarily
small.

Note nevertheless that this strategy (dividing the quantity
$\hat{S}_{\mathrm{cr}}(E,\Theta)$ by an arbitrary solid angle,
expressing explicitly this solid angle) is of course valid. The
actual, physical magnitude measured in reality will be the
\emph{rate}. In particular, for cosmic rays we will measure a rate
(for a given Zenith Angle $\Theta$) of incoming particles with
energies greater or equal to $E_0$ of:
%
\begin{equation}
  \label{eq:Rcr}
  R(E\ge E_0,\Theta) = 
  \int_{E_0}^{\infty}\,\d E\,\frac{\d F(E)}{\d E}\,
  \chronopair{
  \int_0^{2\pi}\,\d \varphi 
  \int_0^{\infty}\,r\,\d r 
  \int_{\Omega}\,\varepsilon_{\mathrm{cr}}(r,\delta;E,\Theta)\,\d \Omega}
\end{equation}
%
Grouping the expression marked with the sign
{\raisebox{6pt}{$\chronopair{\quad}$}\thickspace, we can re-write this
  as:
%
\begin{equation}
  \label{eq:Rcrbis}
  R(E\ge E_0,\Theta) = 
  \int_{E_0}^{\infty}\,\d E\,\frac{\d F(E)}{\d E}\,\hat{\mathcal{S}}(E,\Theta)
\end{equation}
%
with $\hat{\mathcal{S}}(E,\Theta)$ being our
$\hat{S}_{\mathrm{cr}}(E,\Theta)$, with units [area$\times$solid
angle]. If one wants to obtain results in units [area], one has simply
to multiply and to divide, inside the integral, by \emph{any} solid
angle
%
\begin{equation}
  \label{eq:Rcrtris}
  R(E\ge E_0,\Theta) = 
  \int_{E_0}^{\infty}\,\d E\,\frac{\d F(E)}{\d E}\,
  \frac{1}{\Omega}\,\Omega\,\hat{\mathcal{S}}(E,\Theta) =
  \int_{E_0}^{\infty}\,\d E\,\frac{\d F(E)}{\d E}\,
  \frac{1}{\Omega}\,\mathcal{S}(E,\Theta)
\end{equation}
%
and use the new $\mathcal{S}(E,\Theta)$ to compare with the effective
collection area calculated for gammas. But, as it was explained above,
this can be always as small as one wants, and therefore any comparison
would be simply useless.

One could think then that the solid angle used in the normalization
should be the one where the efficiency is non-zero. But since the
efficiency at a given off-axis angle depends on the energy of the
primary, the Zenith Angle and the core distance, so will this solid
angle do. In other words, we can not use a single, fixed solid angle
for all energies, Zenith Angles and core distances.

Going just a bit further in this approach, we will then calculate a
new variable, $S_{\mathrm{cr}}(E,\Theta)$, by means of a
\emph{weighted average of the detection efficiency}. We will use as
weights the solid angle, integrating until the detection efficiency
becomes zero. Note that this uses the fact that
$\delta_{\mathrm{max}}$ depends on $E$, $r$ and $\Theta$. In symbols,
if we calculate this \emph{weighted average of the detection
  efficiency},
$\mean{\varepsilon_{\mathrm{cr}}}_{\mbox{}_{\mathrm{w}}}(r;E,\Theta)$,
as follows:
%
\begin{equation}
  \label{eq:effw}
  \tilde{\varepsilon}_{\mathrm{cr}}(r;E,\Theta) \equiv 
  \mean{\varepsilon_{\mathrm{cr}}}_{\mbox{}_{\mathrm{w}}}(r;E,\Theta) =
  \frac{\displaystyle\int_0^{\delta_{\mathrm{max}}(r;E,\Theta)}\,
    \varepsilon_{\mathrm{cr}}(r,\delta;E,\Theta)\d\Omega}{
    \displaystyle\int_0^{\delta_{\mathrm{max}}(r;E,\Theta)}\,\d\Omega}
\end{equation}
%
with $\d\Omega = \sin\delta\,\d\delta\,\d\phi$, then:
%
\begin{equation}
  \label{eq:Scrnorm}
  S_{\mathrm{cr}}(E,\Theta) = 
  \int_0^{2\pi}\,\d\varphi 
  \int_0^{\infty}\,\tilde{\varepsilon}_{\mathrm{cr}}(r;E,\Theta)\,r\,\d r 
\end{equation}

In this way we calculated the effective collection areas for gammas
and cosmic rays (properly normalized, as we indicated). The results
are shown in Figs. \ref{fig:areagamma}, \ref{fig:areahadr} and
\ref{fig:areacompare}.

\MORE%%%%%%%%%%%%%%%%%%%%%%%%%%%%%%%%%%%%%%%%%%%%%%%%%%%%%%%%%%%%

%%------------------------------------------------------------
\subsection{Detection Rates}

One of our goals is the estimation of the detection rates for gamma
and cosmic rays. Let's see how to calculate these quantities.

Let's assume we are observing a point like source of gamma rays, with
a flux similar to that of our standard candle, the Crab Nebula.
Indeed, we will use as the spectrum for our assumed source of gamma
rays a flux
%
\gFluxeq
%
normalized at $10\u{GeV}$, and extrapolated to higher energies.
This means, a differential flux of
%
\gdFluxeq

For protons, we used the measured (by other experiments) flux:
%
\crFluxeq

We will call \emph{spectral index}, $\alpha$, the (module of the)
exponent of the differential flux. Therefore, we have in our
simulations
%
\specindexeq

%%------------------------------------------------------------
\subsubsection{Differential rates of detection}

With these differential fluxes, we can calculate the differential
detection rates as follows: for gamma rays
%
\gdRateeq
%
where we omitted the dependency in $\Theta$; for cosmic rays
%
\crdRateeq

%%------------------------------------------------------------
\subsubsection{Integral rates of detection}

From the Eq. \ref{eq:gdRate}, the expression for the integral
detection rate for gammas is (omitting again the dependency on
$\Theta$; i.e., for a given value of the Zenith Angle)
%
\gRateeq
%
and similarly for cosmic rays
%
\crRateeq


%%------------------------------------------------------------
\subsubsection{Hadronic, electronic and muonic backgrounds}


%\randpoifig



\endinput
%

%%TODO
%
% Efficiencias de trigger para gammas y hadrones 
%
%
%%TODO


%% Local Variables:
%% mode:latex
%% TeX-master: t
%% End:
%%EOF
