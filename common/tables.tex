%%%%%%%%%%%%%%%%%%%%%%%%%%%%%%%%%%%%%%%%%%%%%%%%%%%%%%%%%%%%%%%%%%%%%%%%%%%
%%
%%  tables.tex
%%
%%  Created: Mon Oct 20 14:58:56 1997
%%  Author.: Jose Carlos Gonzalez
%%  Notes..:
%%          
%%-------------------------------------------------------------------------
%% Filename: $RCSfile$
%% Revision: $Revision$
%% Date:     $Date$
%%%%%%%%%%%%%%%%%%%%%%%%%%%%%%%%%%%%%%%%%%%%%%%%%%%%%%%%%%%%%%%%%%%%%%%%%%%

%%%%%%%%%%%%%%%%%%%%%%%%%%%%%%%%%%%%%%%%%%%%%%%%%%%%%%%%%%%%
%% THE COSMIC RADIATION %%%%%%%%%%%%%%%%%%%%%%%%%%%%%%%%%%%%
%%%%%%%%%%%%%%%%%%%%%%%%%%%%%%%%%%%%%%%%%%%%%%%%%%%%%%%%%%%%

\def\CRfluxindextable{
\begin{table}[htbp]
\centering
\footnotesize
    \begin{tabular}{|lrcc|}
\hline
%
\ifenglish
Element & Z & $\Phi_0$ & $\alpha$ \\ 
 &   & {\footnotesize $[\u{m^2}\u{s}\u{sr}\u{TeV/nucleus}]^{-1}$} & \\
\else
Elemento & Z & $\Phi_0$ & $\alpha$ \\ 
 &   & {\footnotesize $[\u{m^2}\u{s}\u{sr}\u{TeV/\text{n\'ucleo}}]^{-1}$} & \\
\fi
%
\hline\hline
H  & 1  & $(10.57 \pm 0.30) \cdot 10^{-2}$ & $2.76 \pm 0.02$ \\
He & 2  & $(6.73 \pm 0.20)  \cdot 10^{-2}$ & $2.63 \pm 0.02$ \\
Li & 3  & $(2.08 \pm 0.51)  \cdot 10^{-3}$ & $2.54 \pm 0.09$ \\
Be & 4  & $(4.74 \pm 0.48)  \cdot 10^{-4}$ & $2.75 \pm 0.04$ \\
B  & 5  & $(8.95 \pm 0.79)  \cdot 10^{-4}$ & $2.95 \pm 0.05$ \\
C  & 6  & $(1.06 \pm 0.01)  \cdot 10^{-2}$ & $2.66 \pm 0.02$ \\
N  & 7  & $(2.35 \pm 0.08)  \cdot 10^{-3}$ & $2.72 \pm 0.05$ \\
0  & 8  & $(1.57 \pm 0.04)  \cdot 10^{-2}$ & $2.68 \pm 0.03$ \\
F  & 9  & $(3.28 \pm 0.48)  \cdot 10^{-4}$ & $2.69 \pm 0.08$ \\
Ne & 10 & $(4.60 \pm 0.10)  \cdot 10^{-3}$ & $2.64 \pm 0.03$ \\
Na & 11 & $(7.54 \pm 0.33)  \cdot 10^{-4}$ & $2.66 \pm 0.04$ \\
Mg & 12 & $(8.01 \pm 0.26)  \cdot 10^{-3}$ & $2.64 \pm 0.04$ \\
Al & 13 & $(1.15 \pm 0.15)  \cdot 10^{-3}$ & $2.66 \pm 0.04$ \\
Si & 14 & $(7.96 \pm 0.15)  \cdot 10^{-3}$ & $2.75 \pm 0.04$ \\
P  & 15 & $(2.70 \pm 0.20)  \cdot 10^{-4}$ & $2.69 \pm 0.06$ \\
S  & 16 & $(2.29 \pm 0.24)  \cdot 10^{-3}$ & $2.55 \pm 0.09$ \\
Cl & 17 & $(2.94 \pm 0.19)  \cdot 10^{-4}$ & $2.68 \pm 0.05$ \\
Ar & 18 & $(8.36 \pm 0.38)  \cdot 10^{-4}$ & $2.64 \pm 0.06$ \\
K  & 19 & $(5.36 \pm 0.15)  \cdot 10^{-4}$ & $2.65 \pm 0.04$ \\
Ca & 20 & $(1.47 \pm 0.12)  \cdot 10^{-3}$ & $2.70 \pm 0.06$ \\
Sc & 21 & $(3.04 \pm 0.19)  \cdot 10^{-4}$ & $2.64 \pm 0.06$ \\
Ti & 22 & $(1.13 \pm 0.14)  \cdot 10^{-3}$ & $2.61 \pm 0.06$ \\
V  & 23 & $(6.31 \pm 0.28)  \cdot 10^{-4}$ & $2.63 \pm 0.05$ \\
Cr & 24 & $(1.36 \pm 0.12)  \cdot 10^{-3}$ & $2.67 \pm 0.06$ \\
Mn & 25 & $(1.35 \pm 0.14)  \cdot 10^{-3}$ & $2.46 \pm 0.22$ \\
Fe & 26 & $(1.78 \pm 0.18)  \cdot 10^{-2}$ & $2.60 \pm 0.09$ \\
Co & 27 & $(7.51 \pm 0.37)  \cdot 10^{-5}$ & $2.72 \pm 0.09$ \\
Ni & 28 & $(9.96 \pm 0.43)  \cdot 10^{-4}$ & $2.51 \pm 0.18$ \\
\hline
\ifenglish All-particle \else Multi-part'icula \fi %
          & 1--28 &  $(25.70 \pm 1.63) \cdot 10^{-2}$ & $2.68 \pm 0.03$ \\
\hline
e$^-$    & ~ & $(0.95\pm 0.19) \cdot 10^{-4}$ & $3.26\pm 0.06$ \\
\hline
\end{tabular}
\ifenglish
\caption[Absolute flux normalizations and spectral 
  indices for cosmic rays]{Absolute flux normalizations $\Phi_0$ and 
  spectral indices $\alpha$ for the various elements with nuclear charge
  number $Z$, and electrons (only statistical errors are shown)}
\else
\caption[Flujos absolutos e 'indices espectrales para rayos 
  c'osmicos]{Flujos absolutos  $\Phi_0$ e 'indices espectrales $\alpha$ para 
  los diferentes elementos, con carga nuclear $Z$, que componen los 
  rayos c'osmicos (s{\'o}lo se muestran errores estad{\'\i}sticos)}
\fi
\label{table:CRfluxindex}
\end{table}
}

%%%%%%%%%%%%%%%%%%%%%%%%%%%%%%%%%%%%%%%%%%%%%%%%%%%%%%%%%%%%
%% ATMOPHERIC GAMMA AND COSMIC RAY SHOWERS %%%%%%%%%%%%%%%%%
%%%%%%%%%%%%%%%%%%%%%%%%%%%%%%%%%%%%%%%%%%%%%%%%%%%%%%%%%%%%

%%%%%%%%%%%%%%%%%%%%%%%%%%%%%%%%%%%%%%%%%%%%%%%%%%%%%%%%%%%%
%% SIMULATION OF ATMOSPHERIC SHOWERS %%%%%%%%%%%%%%%%%%%%%%%
%%%%%%%%%%%%%%%%%%%%%%%%%%%%%%%%%%%%%%%%%%%%%%%%%%%%%%%%%%%%

%%%%%%%%%%%%%%%%%%%%%%%%%%%%%%%%%%%%%%%%%%%%%%%%%%%%%%%%%%%%
\def\CORSIKAGeanttable{
\begin{table}[htb]
\begin{center}
\normalsize
Naming convention for particles in \CORSIKA according to GEANT \\
with extensions for resonances ($\rho$, $K^*$, and $\Delta$) and neutrinos\\[10pt]
\scriptsize
\begin{tabular}{rc|rc}
\hline
Code & \hspace{5em} Particle \hspace{5em} & Code & \hspace{5em} Particle \hspace{5em} \\
\hline %%----------------------------------------
\hline %%----------------------------------------
        1 & $\gamma$ &                               39 & ($F^+$) \\
        2 & $e^+$ &                                  40 & ($F^-$) \\
        3 & $e^-$ &                                  41 & ($\Lambda_{\text{C}}^+$) \\
        4 & $\mu$ (see codes 66--69) &               42 & ($W^+$) \\
        5 & $\mu^+$ &                                43 & ($W^-$) \\
        6 & $\mu^-$ &                                44 & ($Z^0$) \\
        7 & $\pi^0$ &                                45 & (Deuteron) \\
        8 & $\pi^+$ &                                46 & (Tritium) \\
        9 & $\pi^-$ &                                47 & (Alpha) \\
       10 & $K^0_{\text{L}}$ &                       48 & --- \\
       11 & $K^+$ &                                  49 & --- \\
       12 & $K^-$ &                                  50 & --- \\
       13 & $n$ &                                    51 & $\rho^0$ \\
       14 & $p$ &                                    52 & $\rho^+$ \\
       15 & $\bar{p}$ &                              53 & $\rho^-$ \\
       16 & $K^0_{\text{S}}$ &                       54 & $\Delta^{++}$ \\
       17 & $\eta$ (see codes 71--74) &              55 & $\Delta^+$ \\
       18 & $\Lambda$ &                              56 & $\Delta^0$ \\
       19 & $\Sigma^+$ &                             57 & $\Delta^-$ \\
       20 & $\Sigma^0$ &                             58 & $\bar{\Delta}^{--}$ \\
       21 & $\Sigma^-$ &                             59 & $\bar{\Delta}^-$ \\
       22 & $\Xi^0$ &                                60 & $\bar{\Delta}^0$ \\
       23 & $\Xi^-$ &                                61 & $\bar{\Delta}^+$ \\
       24 & $\Omega$ &                               62 & $K^{*0}$ \\
       25 & $\bar{n}$ &                              63 & $K^{*+}$ \\
       26 & $\bar{\Lambda}$ &                        64 & $K^{*-}$ \\
       27 & $\bar{\Sigma}^-$ &                       65 & $\bar{K}^{*0}$ \\
       28 & $\bar{\Sigma}^0$ &                       66 & $\nu_e$   \\
       29 & $\bar{\Sigma}^+$ &                       67 & $\bar{\nu}_e$ \\
       30 & $\bar{\Xi}^0$ &                          68 & $\nu_\mu$ \\
       31 & $\bar{\Xi}^+$ &                          69 & $\bar{\nu}_\mu$ \\
       32 & $\bar{\Omega}$ &                         70 & --- \\
       33 & ($\tau^+$) &                             71 & $\eta \longrightarrow 2 \gamma$ \\
       34 & ($\tau^-$) &                             72 & $\eta \longrightarrow 3 \pi^0$ \\
       35 & ($D^+$) &                                73 & $\eta \longrightarrow \pi^+ + \pi^- + \pi^0$ \\
       36 & ($D^-$) &                                74 & $\eta \longrightarrow \pi^+ + \pi^- + \gamma$ \\
       37 & ($D^0$) &                                75 & $\muon^+$  additional information of origin \\
       38 & ($\bar{D}^0$) &                          76 & $\muon^-$  additional information of origin \\
\hline
\end{tabular}\\
\vspace{20pt}
\normalsize
Naming convention (extension) for nuclei\\[10pt]
\scriptsize
\begin{tabular}{rc}
\hline
Code & Particle\\
\hline %%----------------------------------------
\hline %%----------------------------------------
     $AAZZ$ & Nucleus of $ZZ$ protons and $(AA-ZZ)$ neutrons \\
          & restrictions:  $AA < 59$   and   $ZZ < AA+1$ \\
     9900 & \Cherenkov photons on the particle output file \\
\hline
\end{tabular}
\end{center}
\ifenglish
\caption{List of particles handled by \CORSIKA}
\else
\caption{Lista de part'iculas utilizadas en \CORSIKA}
\fi
\label{table:CORSIKAGeantcodes}
\end{table}
}

%%%%%%%%%%%%%%%%%%%%%%%%%%%%%%%%%%%%%%%%%%%%%%%%%%%%%%%%%%%%
\def\CORSIKAtableRH{
\begin{table}[t]
  \begin{center}
    \footnotesize
    \begin{tabular}{|r|l|}
\multicolumn{2}{c}{\bfseries Run header sub-block: (once per run)}\\
\hline
No. of word&Contents of word\\
\hline %%----------------------------------------
\hline %%----------------------------------------
1& `RUNH' \\
2& run number \\
3& date of begin run ( yymmdd ) \\
4&version of program \\
5& number of observation levels (maximum 10) \\
5+ $i$& height of level $i$ in cm\\
16 &slope of energy spectrum \\
17 &lower limit of energy range \\
18 &upper limit of energy range \\
19 &flag for EGS4 treatment of em. component \\
20 &flag for NKG treatment of em. component \\
21 &kin. energy cuto, for hadrons in GeV \\
22 &kin. energy cuto, for muons in GeV \\
23 &kin. energy cuto, for electrons in GeV \\
24 &energy cuto, for photons in GeV\\
\hline
\multicolumn{2}{|c|}{physical constants and interaction flags}\\
\hline
24+$i$ &C($i$), $i$ = 1,50 \\
74+$i$ &CC($i$), $i$ = 1,20 \\
94+$i$ &CKA($i$), $i$ = 1,40 \\
134+$i$& CETA($i$), $i$ = 1,5 \\
139+$i$& CSTRBA($i$),i = 1,11 \\
150+$i$& 0, $i$ = 1,4 (no longer used) \\
154+$i$& CAN($i$), $i$ = 1,50 \\
204+$i$& CANN($i$), $i$ = 1,50 \\
254+$i$& AATM($i$), $i$ = 1,5 \\
259+$i$& BATM($i$), $i$ = 1,5 \\
264+$i$& CATM($i$), $i$ = 1,5\\
270& NFLAIN (as real) \\
271& NFLDIF (as real) \\
272& NFLPI0+100$\times$NFLPIF (as real) \\
273& NFLCHE+100$\times$NFRAGM (as real)\\
\hline
    \end{tabular}
  \end{center}
  \caption{Structure of the run header sub-block}
  \label{tab:rh}
\end{table}
}

%%%%%%%%%%%%%%%%%%%%%%%%%%%%%%%%%%%%%%%%%%%%%%%%%%%%%%%%%%%%
\def\CORSIKAtableEHone{
\begin{table}[t]
  \begin{center}
    \footnotesize
    \begin{tabular}{|r|l|}
\multicolumn{2}{c}{\bfseries Event header sub-block: (once per event)}\\
\hline
No. of word&Contents of word\\
\hline %%----------------------------------------
\hline %%----------------------------------------
1& `EVTH'  \\
2& event number  \\
3& particle id (particle code or A$\times$ 100+Z for nuclei)  \\
4& total energy in GeV  \\
5& starting altitude in g=cm2  \\
6& number of first target if fixed  \\
7& z coordinate (height) of first interaction in cm  \\
8& px momentum in x direction in GeV  \\
9& py momentum in y direction in GeV  \\
10& pz momentum in -z direction in GeV \\
&(pz is positive for downward going particles)  \\
11& zenith angle $\theta$ in radian  \\
12& azimuth angle $\phi$ in radian  \\
13& number of different random number sequences (max. 10)  \\
11+3$\times$ $i$& integer seed of sequence $i$  \\
12+3$\times$ $i$& number of offset random calls ($\mathrm{mod} 10^6$) of sequence $i$  \\
13+3$\times$ $i$& number of offset random calls ($/10^6$) of sequence i \\
44& run number  \\
45& date of begin run (yymmdd)  \\
46& version of program  \\
47& number of observation levels  \\
47+$i$& height of level $i$ in cm \\
58& slope of energy spectrum  \\
59& lower limit of energy range in GeV  \\
60& upper limit of energy range in GeV  \\
61& cutoff for hadrons kinetic energy in GeV  \\
62& cutoff for muons kinetic energy in GeV  \\
63& cutoff for electrons kinetic energy in GeV  \\
64& cutoff for photons energy in GeV  \\
65& NFLAIN as a real number  \\
66& NFLDIF as a real number  \\
67& NFLPI0 as a real number  \\
68& NFLPIF as a real number  \\
69& NFLCHE as a real number  \\
70& NFRAGM as a real number  \\
71& x component of Earth's magnetic field in $\mu$T  \\
72& z component of Earth's magnetic field in $\mu$T  \\
73& flag for activating EGS4 as real number  \\
74& flag for activating NKG as real number \\
\hline
    \end{tabular}
  \end{center}
  \caption{Structure of event header sub-block.}
  \label{tab:eh1}
\end{table}
}

%%%%%%%%%%%%%%%%%%%%%%%%%%%%%%%%%%%%%%%%%%%%%%%%%%%%%%%%%%%%
\def\CORSIKAtableEHtwo{
\begin{table}[t]
  \begin{center}
    \footnotesize
    \begin{tabular}{|r|l|}
\multicolumn{2}{c}{\bfseries Event header sub-block: (continued)}\\
\hline
No. of word&Contents of word\\
\hline %%----------------------------------------
\hline %%----------------------------------------
75 &GHEISHA flag as real number\\
76 &VENUS flag as real number   \\
77 &CERENKOV flag as real number\\
78 &NEUTRINO flag as real number\\
79 &HORIZONT flag as real number\\
80 &computer flag (1=IBM, 2=Transputer, 3=DEC/UNIX,   \\
&4=Macintosh, 5=VAX/VMS, 6=LINUX) as real number\\
81 &lower edge of $\theta$ interval (in $^\circ$)\\
82 &upper edge of $\theta$ interval (in $^\circ$)\\
83 &lower edge of OE interval (in $^\circ$)\\
84 &upper edge of OE interval (in $^\circ$)\\
85 &Cherenkov bunch size in the case of Cherenkov calculations\\
86 &number of Cherenkov detectors in x-direction\\
87 &number of Cherenkov detectors in y-direction\\
88 &grid spacing of Cherenkov detectors in x-direction in cm\\
89 &grid spacing of Cherenkov detectors in y-direction in cm\\
90 &length of each Cherenkov detector in x-direction in cm\\
91 &length of each Cherenkov detector in y-direction in cm\\
92 &Cherenkov output directed to particle output file (= 0.)   \\
&or Cherenkov output file (= 1.)      \\
93 &angle (in rad) between array x-direction and magnetic north\\
94 &flag for additional muon information on particle output file\\
95 &step length factor for multiple scattering step length in EGS\\
96 &Cherenkov bandwidth lower end in nm \\
97 &Cherenkov bandwidth upper end in nm \\
98 &number $i$ of uses of each event \\
98+$i$& x coordinate of $i^{\mathrm{th}}$ core location for scattered events in cm \\
118+$i$& y coordinate of $i^{\mathrm{th}}$ core location for scattered events in cm\\
139 &SIBYLL interaction flag as real number \\
140 &SIBYLL cross section flag as real number \\
141 &QGSJET interaction flag as real number \\
142 &QGSJET cross section flag as real number \\
143 &DPMJET interaction flag as real number \\
144 &DPMJET cross section flag as real number \\
145 &VENUS cross section flag as real number \\
146 &muon multiple scattering flag (1.=Moli{\`e}re, 0.=Gauss) \\
147 &EFRCTHN energy fraction of thinning level \\
148 &NKG radial distribution range in cm \\
149...273 &not used\\
\hline
    \end{tabular}
  \end{center}
  \caption{Structure of event header sub-block (continued).}
  \label{tab:eh2}
\end{table}
}

%%%%%%%%%%%%%%%%%%%%%%%%%%%%%%%%%%%%%%%%%%%%%%%%%%%%%%%%%%%%
\def\CORSIKAtableEE{
\begin{table}[t]
  \begin{center}
    \footnotesize
    \begin{tabular}{|r|l|}
\multicolumn{2}{c}{\bfseries Event end sub-block}\\
\hline
No. of word&Contents of word\\
\hline %%----------------------------------------
\hline %%----------------------------------------
1 &`EVTE'  \\
2 &event number \\
&statistics for one shower :  \\
3 &weighted number of photons written to particle output file  \\
4 &weighted number of electrons written to particle output file  \\
5 &weighted number of hadrons written to particle output file  \\
6 &weighted number of muons written to particle output file  \\
7 &number of weighted particles written to particle output file PATAPE \\
&(This number includes also Cherenkov bunches, if Cherenkov output  \\
&is directed to PATAPE, but excludes additional muon information)  \\
&NKG output (if selected) :  \\
7+$i$&$i$= 1..21 lateral distribution in x direction for 1. level in \u{cm^2}\\
28+$i$&$i$= 1..21 lateral distribution in y direction for 1. level in \u{cm^2}\\
49+$i$&$i$= 1..21 lateral distribution in xy direction for 1. level in \u{cm^2}\\
70+$i$&$i$= 1..21 lateral distribution in yx direction for 1. level in \u{cm^2}\\
91+$i$&$i$= 1..21 lateral distribution in x direction for 2. level in \u{cm^2}\\
112+$i$&$i$= 1..21 lateral distribution in y direction for 2. level in \u{cm^2}\\
133+$i$&$i$= 1..21 lateral distribution in xy direction for 2. level in \u{cm^2}\\
154+$i$&$i$= 1..21 lateral distribution in yx direction for 2. level in \u{cm^2}\\
175+$i$&$i$= 1..10 electron number in steps of 100 \u{g/cm^2} \\
185+$i$&$i$= 1..10 age in steps of 100 \u{g/cm^2} \\
195+$i$&$i$= 1..10 distances for electron distribution in cm \\
205+$i$&$i$= 1..10 local age 1. level\\
215+$i$&$i$= 1..10 height of levels for electron numbers in \u{g/cm^2}\\
225+$i$&$i$= 1..10 height of levels for electron numbers in cm \\
235+$i$&$i$= 1..10 distance bins for local age in cm \\
245+$i$&$i$= 1..10 local age 2. level \\
255+$i$&$i$= 1..6 parameters of longitudinal distribution of charged particles\\
262&$\chi^2$ per degree of freedom of fit to longitudinal distribution  \\
263..273&not used \\
\hline
    \end{tabular}
  \end{center}
  \caption{Structure of event end sub-block.}
  \label{tab:ee}
\end{table}
}

%%%%%%%%%%%%%%%%%%%%%%%%%%%%%%%%%%%%%%%%%%%%%%%%%%%%%%%%%%%%
\def\CORSIKAtableRE{
\begin{table}[t]
  \begin{center}
    \footnotesize
    \begin{tabular}{|r|l|}
\multicolumn{2}{c}{\bfseries Run end sub-block}\\
\hline
No. of word&Contents of word\\
\hline %%----------------------------------------
\hline %%----------------------------------------
1&`RUNE' \\
2&run number \\
3&number of events processed \\
4..273&not used yet\\
\hline
    \end{tabular}
  \end{center}
  \caption{Structure of run end sub-block.}
  \label{tab:re}
\end{table}
}

%%%%%%%%%%%%%%%%%%%%%%%%%%%%%%%%%%%%%%%%%%%%%%%%%%%%%%%%%%%%
\def\CORSIKAtablePART{
\begin{table}[t]
  \begin{center}
    \footnotesize
    \begin{tabular}{|r|l|}
\multicolumn{2}{c}{\bfseries Particle data sub-block : (up to 39 particles, 7 words each)}\\
\hline
No. of word&Contents of word \\
\hline %%----------------------------------------
\hline %%----------------------------------------
$7\times (n-1)+1$ &particle description \\
  &(part. id$\times$1000+ hadr. generation$\times$10+ no. of obs. level)  \\
$7\times (n-1)+2$ &px, momentum in x direction in GeV  \\
$7\times (n-1)+3$ &py, momentum in y direction in GeV  \\
$7\times (n-1)+4$ &pz, momentum in -z direction in GeV  \\
$7\times (n-1)+5$ &x coordinate in cm  \\
$7\times (n-1)+6$ &y coordinate in cm  \\
$7\times (n-1)+7$ &t time since first interaction in nsec \\
&(z coordinate in cm for additional muon information)  \\
\hline
\multicolumn{2}{|c|}{for n = 1 \ldots 39}\\
\multicolumn{2}{|c|}{last block may not be completely filled}\\
\hline
    \end{tabular}
  \end{center}
  \caption{Structure of particle data sub-block.}
  \label{tab:part}
\end{table}
}

%%%%%%%%%%%%%%%%%%%%%%%%%%%%%%%%%%%%%%%%%%%%%%%%%%%%%%%%%%%%
\def\CORSIKAtableCHER{
\begin{table}[t]
  \begin{center}
    \footnotesize
    \begin{tabular}{|r|l|}
\multicolumn{2}{c}{\bfseries Cherenkov photon data sub-block : (up to 39 photons, 7 words each)}\\
\hline
No. of word&Contents of word \\
\hline %%----------------------------------------
\hline %%----------------------------------------
$7\times (n-1)+1$ &j$\times$100000.+$\lambda$ \\
$7\times (n-1)+2$ &x coordinate in the obs.level, in cm\\
$7\times (n-1)+3$ &y coordinate in the obs.level, in cm\\
$7\times (n-1)+4$ &u direction cosine to x axis\\
$7\times (n-1)+5$ &v direction cosine to y axis\\
$7\times (n-1)+6$ &arrival time since first interaction in nsec\\
$7\times (n-1)+7$ &height of production of photon in cm\\
\hline
\multicolumn{2}{|c|}{for n = 1 \ldots 39}\\
\multicolumn{2}{|c|}{last block may not be completely filled}\\
\hline
    \end{tabular}
  \end{center}
  \caption[Structure of Cherenkov photon data sub-block]{Structure 
    of Cherenkov photon data sub-block. The value of
    $j$ is defined as $j\equiv$number of the telescope which gave
    trigger (usually 1). We simulate with a bunch size (number of
    photons per sub-block) equal to 1.}
  \label{tab:cher}
\end{table}
}

%%%%%%%%%%%%%%%%%%%%%%%%%%%%%%%%%%%%%%%%%%%%%%%%%%%%%%%%%%%%
\def\CORSIKAtableSTA{
\begin{table}[t]
  \begin{center}
    \footnotesize
    \begin{tabular}{|r|l|}
\multicolumn{2}{c}{\bfseries Statistics data block}\\
\hline
No. of word&Contents of word \\
\hline %%----------------------------------------
\hline %%----------------------------------------
  1 \ldots 273 (273)& Event Header for this shower\\
274 \ldots 547 (273)&Event End for this shower\\
\hline
548&Time of the first photon stored in tape\\
549&Time of the last photon stored in tape\\
\hline
\multicolumn{2}{|c|}{for $i$ = 0 \ldots 9 $\longrightarrow$ 10 obs.levels}\\
\hline
550+$i\times 22$&Number of protons at obs.level $i$ \\
551+$i\times 22$&Number of antiprotons at obs.level $i$ \\
552+$i\times 22$&Number of neutrons at obs.level $i$ \\
553+$i\times 22$&Number of antineutrons at obs.level $i$ \\
554+$i\times 22$&Number of photons at obs.level $i$ \\
555+$i\times 22$&Number of electrons at obs.level $i$ \\
556+$i\times 22$&Number of positrons at obs.level $i$ \\
557+$i\times 22$&Number of neutrinos at obs.level $i$ \\
558+$i\times 22$&Number of $\mu^-$ at obs.level $i$ \\
559+$i\times 22$&Number of $\mu^+$ at obs.level $i$ \\
560+$i\times 22$&Number of $\pi^0$ at obs.level $i$ \\
561+$i\times 22$&Number of $\pi^-$ at obs.level $i$ \\
562+$i\times 22$&Number of $\pi^+$ at obs.level $i$ \\
563+$i\times 22$&Number of $K^0$ Long at obs.level $i$ \\
564+$i\times 22$&Number of $K^0$ Short  at obs.level $i$ \\
565+$i\times 22$&Number of $K^{*-}$  at obs.level $i$ \\
566+$i\times 22$&Number of $K^{*+}$  at obs.level $i$ \\
567+$i\times 22$&Number of strange Baryons at obs.level $i$ \\
568+$i\times 22$&Number of Deuterons at obs.level $i$ \\
579+$i\times 22$&Number of Tritons at obs.level $i$ \\
570+$i\times 22$&Number of $\alpha$ particles at obs.level $i$ \\
571+$i\times 22$&Number of other (strange) particles at obs.level $i$ \\
\hline
770&Number of Nucleons\\
771&Number of Pions   \\
772&Number of Etas\\
773&Number of Kaons   \\
773&Number of strange Baryons\\
774&Number of Cherenkov photons from Electrons\\
775&Number of Cherenkov photons from Electrons\\
\hline
776&LPCT1 = 1 if Long.Distribution must be calculated\\
777&$N_{\mathrm{STEP}}$: Number of steps in the  Long.Distribution\\
778&$W_{\mathrm{STEP}}$: Width of the step in the  Long.Distribution\\
\hline
\multicolumn{2}{|c|}{\underline{\itshape Longitudinal distributions}}\\
\multicolumn{2}{|c|}{for $k$ = 0 \ldots 8 $\longrightarrow$ 9 species}\\
\multicolumn{2}{|c|}{$\gamma$,e$^+$,e$^-$,$\mu^-$,$\mu^+$,hadrons,charged
particles,nuclei and Cherenkov photons}\\
\multicolumn{2}{|c|}{and for $i$ = 0 \ldots $N_{\mathrm{STEP}}$}\\
\multicolumn{2}{|c|}{ $\longrightarrow$ $N_{\mathrm{STEP}}$ steps
of $W_{\mathrm{STEP}}$}\\
\hline
779+$k\times N_{\mathrm{STEP}}+i$&Number of particles of specie
$k$ at depth $i\times W_{\mathrm{STEP}}$ g/cm$^2$ \\
\hline
    \end{tabular}
  \end{center}
  \caption{Structure of Statistics data block as saved in the
    \texttt{staXXXXXX} files.}
  \label{tab:sta}
\end{table}
}


%%%%%%%%%%%%%%%%%%%%%%%%%%%%%%%%%%%%%%%%%%%%%%%%%%%%%%%%%%%%
%% SIMULATION OF THE DETECTOR: MAGIC %%%%%%%%%%%%%%%%%%%%%%%
%%%%%%%%%%%%%%%%%%%%%%%%%%%%%%%%%%%%%%%%%%%%%%%%%%%%%%%%%%%%

%%%%%%%%%%%%%%%%%%%%%%%%%%%%%%%%%%%%%%%%%%%%%%%%%%%%%%%%%%%%
\def\seventupletbl{
\begin{center}
\begin{tabular}{cl}
$\lambda$ & Wavelength of the emitted \Cherenkov photon \\
$x,y$     & Position of the \Cherenkov photon in the horizontal
  plane of the observation level \\
$u,v$     & Director cosines in X and Y directions of the incoming
  photon\\
$t$       & Time elapsed since the primary of the shower entered the
  atmosphere until the photon was emitted\\
$h$       & Height of production of the photon\\
\end{tabular}
\end{center}
}

%%%%%%%%%%%%%%%%%%%%%%%%%%%%%%%%%%%%%%%%%%%%%%%%%%%%%%%%%%%%
\def\zonefocalstbl{
\begin{table}[p]
\begin{center}
\begin{tabular}{|c|D{.}{.}{4}|D{.}{.}{2}|}
\hline
%\ifenglish
%\multirow{2}{2cm}{~~Zone} & 
%\multicolumn{1}{c|}{Focal Length} & 
%\multicolumn{1}{c|}{N.mirrors} \\
%~ & \multicolumn{1}{c|}{(cm)} & ~ \\
%\else 
%\multirow{2}{2cm}{~~Zona} & 
%\multicolumn{1}{c|}{Dist. Focal} & 
%\multicolumn{1}{c|}{N.espejos} \\
%~ & \multicolumn{1}{c|}{(cm)} & ~ \\
%\fi
\ifenglish
Zone & 
\multicolumn{1}{c|}{Focal Length} & 
\multicolumn{1}{c|}{N.mirrors} \\
\phantom{X} & \multicolumn{1}{c|}{(cm)} & \phantom{X} \\
\else 
Zona & 
\multicolumn{1}{c|}{Dist. Focal} & 
\multicolumn{1}{c|}{N.espejos} \\
\phantom{X} & \multicolumn{1}{c|}{(cm)} & \phantom{X} \\
\fi
\hline
I     &  1703.34 &  36 \\
II    &  1709.22 &  60 \\
III   &  1718.14 &  96 \\
IV    &  1730.06 & 108 \\
V     &  1745.07 & 132 \\
VI    &  1763.26 & 168 \\
VII   &  1784.73 & 196 \\
VIII  &  1805.40 & 124 \\
\hline
\end{tabular}
\end{center}
\ifenglish
\caption[Focal length selected for the 8 zones on the reflector of 
  \MAGIC]{Focal length selected for the 8 zones on the reflector of 
  \MAGIC (see Fig. \ref{fig:zones}).}
\else 
\caption[Longitudes focales de las 8 zonas del espejo de 
  \MAGIC]{Longitudes focales para las 8 zonas en que se ha dividido el 
  espejo de \MAGIC (ver fig. \ref{fig:zones}).}  
\fi
\label{tbl:zonefocals}
\end{table}
}


%% \begin{description}
%% \item[$\lambda$:] Wavelength of the emitted \Cherenkov photon.
%%   
%% \item[$x,y$:] Position of the \Cherenkov photon in the horizontal
%%   plane of the observation level.
%%   
%% \item[$u,v$:] Director cosines in X and Y directions of the incoming
%%   photon.
%%   
%% \item[$t$:] Time elapsed since the primary of the shower entered the
%%   atmosphere until the photon was emitted.
%%   
%% \item[$h$:] Height of production of the photon.
%% \end{description}


%%%%%%%%%%%%%%%%%%%%%%%%%%%%%%%%%%%%%%%%%%%%%%%%%%%%%%%%%%%%
%%%%%%%%%%%%%%%%%%%%%%%%%%%%%%%%%%%%%%%%%%%%%%%%%%%%%%%%%%%%
%%%%%%%%%%%%%%%%%%%%%%%%%%%%%%%%%%%%%%%%%%%%%%%%%%%%%%%%%%%%
%%%%%%%%%%%%%%%%%%%%%%%%%%%%%%%%%%%%%%%%%%%%%%%%%%%%%%%%%%%%
%%%%%%%%%%%%%%%%%%%%%%%%%%%%%%%%%%%%%%%%%%%%%%%%%%%%%%%%%%%%
\def\q_tbl{
\begin{table}[hbt]
%\vspace{-6pt}
\DependingOnLanguagE%
{\caption{Quality factor calculated for different MaxDIST bins, for the
 Standard IAPDs Camera.}\label{q:tbl}}%
{\caption{Quality factor calculated for different MaxDIST bins, for the
 Standard IAPDs Camera.}\label{q:tbl}}
\begin{center}
\begin{tabular}{|c|c|c|c|}
\hline
~& Gammas & Protons &~\\
 MaxDIST bin &Aceptance factor &Aceptance factor &Q\\
\hline
0.4\tdeg-- 0.5\tdeg &  0.451 & 0.0077 & 5.14  \\ 
0.5\tdeg-- 0.9\tdeg &  0.452 & 0.0049 & 6.47  \\ 
0.9\tdeg-- 1.1\tdeg &  0.446 & 0.0012 & 12.84 \\ 
\hline
\end{tabular}
\end{center}
\end{table}
}

\endinput
%
%% Local Variables:
%% mode:latex
%% TeX-master: t
%% End:

%%EOF
