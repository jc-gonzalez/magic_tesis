%%%%%%%%%%%%%%%%%%%%%%%%%%%%%%%%%%%%%%%%%%%%%%%%%%%%%%%%%%%%%%%%%%%%%%%%%%%
%%
%%  texis.tex
%%
%%  Created: Fri Oct 10 14:24:37 1997
%%  Author.: Jose Carlos Gonzalez
%%  Notes..:
%%          
%%-------------------------------------------------------------------------
%% Filename: $RCSfile$
%% Revision: $Revision$
%% Date:     $Date$
%%%%%%%%%%%%%%%%%%%%%%%%%%%%%%%%%%%%%%%%%%%%%%%%%%%%%%%%%%%%%%%%%%%%%%%%%%%


%\documentclass[11pt,a4paper,twoside]{book}
\documentclass[11pt,a4paper,twoside]{book}

\newif\ifdraft \drafttrue


%%%%%%%%%%%%%%%%%%%%%%%%%%%%%%%%%%%%%%%%%%%%%%%%%%%%%%%%%%%%%%%%%%%%%%%%%%%
%% Compatibility LaTeX - pdfLaTeX
%%%%%%%%%%%%%%%%%%%%%%%%%%%%%%%%%%%%%%%%%%%%%%%%%%%%%%%%%%%%%%%%%%%%%%%%%%%

%%----------------------------------------------------------------------
%% From pdfLaTeX-FAQ
%%  Q 3.1.6. How can I make a document portable to both latex 
%%           and pdflatex? 
%%    
%%    Contributed by: Christian Kumpf 
%%    
%%    Check for the existence of the variable \pdfoutput:

\newif\ifpdf 
\ifx\pdfoutput\undefined 
  \typeout{### Precompiling for LaTeX . . .}
  \pdffalse           % we are not running pdfLaTeX 
\else 
  \typeout{### Precompiling for pdfLaTeX . . .}
  \pdfoutput=1        % we are running pdfLaTeX 
  \pdftrue 
\fi 

%%    Then use your new variable \ifpdf , like in

\ifpdf 
  \def\special#1{\write16{special: #1}}
  \pdfcompresslevel=9
\fi

%%----------------------------------------------------------------------

%%%%%%%%%%%%%%%%%%%%%%%%%%%%%%%%%%%%%%%%%%%%%%%%%%%%%%%%%%%%%%%%%%%%%%%%%%%
%% packages
%%%%%%%%%%%%%%%%%%%%%%%%%%%%%%%%%%%%%%%%%%%%%%%%%%%%%%%%%%%%%%%%%%%%%%%%%%%

\typeout{----> Reading packages . . .}

\usepackage{a4wide}
%\usepackage{epsfig}
\ifdraft
  \ifpdf 
    \usepackage[pdftex,draft]{graphicx}
  \else
    \usepackage[dvips,draft]{graphicx}
  \fi
\else
  \ifpdf 
    \usepackage[pdftex]{graphicx}
  \else
    \usepackage[dvips]{graphicx}
  \fi
\fi
\usepackage{fancyheadings}
\usepackage{rotating}
\usepackage{subfigure}
\usepackage{amssymb}
\usepackage{amsmath}
\usepackage{xspace}
\usepackage{enumerate}
\usepackage{makeidx}
\usepackage{cite}
\usepackage{calc}
\usepackage{mathpple}
%\usepackage{charter}
\usepackage{floatflt}
\usepackage{afterpage}
\usepackage{multicol}
\usepackage{multirow}
\usepackage{dcolumn}
%\usepackage{floatfig}
\usepackage{moreverb}
\usepackage{wrapfig}
%\usepackage{showkeys} 
%\usepackage{oldgerm}
\usepackage{longtable} 
\usepackage{lscape}
%\usepackage{draft}
%%\usepackage[cam]{crop}   %% option 'mirror' for mirroring every page
\usepackage{theorem}

\theoremstyle{plain}
\theorembodyfont{\itshape}
\newtheorem{Theo}{Theorem}
\theoremheaderfont{\normalfont\bfseries}

%%%%%%%%%%%%%%%%%%%%%%%%%%%%%%%%%%%%%%%%%%%%%%%%%%%%%%%%%%%%%%%%%%%%%%%%%%%
%% define new floating environments
%%%%%%%%%%%%%%%%%%%%%%%%%%%%%%%%%%%%%%%%%%%%%%%%%%%%%%%%%%%%%%%%%%%%%%%%%%%

\usepackage{float}
%\floatstyle{ruled}
\floatstyle{boxed}
\newfloat{listado}{htbp}{lol}[chapter]
\floatname{listado}{Listing}

\usepackage[indent,footnotesize,sc]{caption2}
\setlength{\captionmargin}{5mm}

\setlength{\overfullrule}{10pt}

\newif\ifCOLORversion
\COLORversionfalse

%%%%%%%%%%%%%%%%%%%%%%%%%%%%%%%%%%%%%%%%%%%%%%%%%%%%%%%%%%%%%%%%%%%%%%%%%%%
%% prepares the index and defines some macros for it
%%%%%%%%%%%%%%%%%%%%%%%%%%%%%%%%%%%%%%%%%%%%%%%%%%%%%%%%%%%%%%%%%%%%%%%%%%%

\setcounter{tocdepth}{2}

\newcommand{\I}[1]{#1\index{#1}}             % word
\newcommand{\Iw}[2]{#1\index{#1@#2}}         % word2
\newcommand{\II}[2]{#1\index{#2!#1}}         % word
\newcommand{\III}[3]{#1\index{#2!#3!#1}}     % word
\newcommand{\Isee}[2]{#1\index{#1|see{#2}}}  % word1, see word2
\newcommand{\Ifrom}[1]{#1\index{#1|(}}       % word p-
\newcommand{\Ito}[1]{#1\index{#1|)}}         % word -p
\newcommand{\Istyle}[2]{#1\index{#1|#2}}     % \style{word}


%%%%%%%%%%%%%%%%%%%%%%%%%%%%%%%%%%%%%%%%%%%%%%%%%%%%%%%%%%%%%%%%%%%%%%%%%%%
%% different ``entry'' environments
%%%%%%%%%%%%%%%%%%%%%%%%%%%%%%%%%%%%%%%%%%%%%%%%%%%%%%%%%%%%%%%%%%%%%%%%%%%

\usepackage{calc}
\usepackage{ifthen}

%%\newcommand{\enhance}[1]{{\scshape #1}}
\newcommand{\enhance}[1]{{#1}}

\newcommand{\entrylabel}[1]{\mbox{\enhance{#1}}\hfil}

\newenvironment{entry}%
    {\begin{list}{}%
        {\renewcommand{\makelabel}{\entrylabel}%
        \setlength{\leftmargin}{\labelwidth+\labelsep}}}%
    {\end{list}}

\newenvironment{Ventry}[1]%
    {\begin{list}{}{\renewcommand{\makelabel}[1]{\enhance{##1}\hfil}%
        \settowidth{\labelwidth}{\enhance{#1}}%
        \setlength{\leftmargin}{\labelwidth+\labelsep}}}%
    {\end{list}}

\newenvironment{Wentry}[1]%
    {\begin{list}{}{\renewcommand{\makelabel}[1]{\enhance{##1}\hfil}%
        \setlength{\labelwidth}{#1}%
        \setlength{\leftmargin}{\labelwidth+\labelsep}}}%
    {\end{list}}

\newlength{\Ulength}

\newcommand{\Uentrylabel}[1]{%
    \settowidth{\Ulength}{\enhance{#1}}%
    \ifthenelse{\lengthtest{\Ulength > \labelwidth}}%
        {\parbox[b]{\labelwidth}%  term > labelwidth
            {\makebox[0pt][l]{\enhance{#1}}\\}}%
        {\enhance{#1}}%            term < labelwidth
    \hfil\relax}

\newenvironment{Uentry}%
    {\renewcommand{\entrylabel}{\Uentrylabel}%
    \begin{entry}}%
    {\end{entry}}

%%%%%%%%%%%%%%%%%%%%%%%%%%%%%%%%%%%%%%%%%%%%%%%%%%%%%%%%%%%%%%%%%%%%%%%%%%%
%% definition of macros
%%%%%%%%%%%%%%%%%%%%%%%%%%%%%%%%%%%%%%%%%%%%%%%%%%%%%%%%%%%%%%%%%%%%%%%%%%%

\typeout{----> Processing definitions . . .}

\makeatletter
\newcommand\figcaption{\def\@captype{figure}\caption}
\newcommand\tabcaption{\def\@captype{table}\caption}
\makeatother

\newenvironment{narrow}[2]{%
  \begin{list}{}{%
    \setlength{\topsep}{0pt}%
    \setlength{\leftmargin}{#1}%
    \setlength{\rightmargin}{#2}%
    \setlength{\listparindent}{\parindent}%
    \setlength{\itemindent}{\parindent}%
    \setlength{\parsep}{\parskip}}%
  \item[]}{\end{list}}
\newlength{\narrowlength}

%-------------------- Captions

%\makeatletter

%\long\def\@makecaption#1#2{\vskip \abovecaptionskip 
%\begin{list}{}
%\item{\bf #1.} #2
%\end{list}\vskip\belowcaptionskip}

%\long\def\@makecaption#1#2{%
%\centering%
%\parbox[t]{0.8\linewidth}{\small{\bf #1}.\quad #2}%
%}

\newcolumntype{d}[1]{D{.}{\cdot}{#1}}
\newcolumntype{.}[1]{D{.}{.}{-1}}

% \def\aafigurecaption{figure}

% \long\def\@makecaption#1#2{%
% \makeatletter
% \ifx\@captype\aafigurecaption\vskip10pt\else\vskip10pt\fi
% \makeatother
% \small
% % Make sure that `Table <N>.' or `Fig. <N>.' is typeset in boldface:
%  \setbox\@tempboxa\hbox{{\bf #1}.\quad #2}
%  \ifdim \wd\@tempboxa >\hsize \unhbox\@tempboxa\par \else \hbox
% to\hsize{\box\@tempboxa\hfil}
%  \fi
% \makeatletter
% \ifx\@captype\aafigurecaption\else\vskip6pt\fi
% \makeatother}

%\makeatother

% \newcounter{figure}
% \def\thefigure{\@arabic\c@figure}
% \def\fps@figure{htbp}
% \def\ftype@figure{1}
% \def\ext@figure{lof}
% \def\fnum@figure{Fig.\ \thefigure.}
% \def\figure{\small\rm\@float{figure}}
% \let\endfigure\end@float
% \@namedef{figure*}{\small\rm\@dblfloat{figure}}
% \@namedef{endfigure*}{\end@dblfloat}

% \newcounter{table}
% \def\thetable{\@arabic\c@table}
% \def\fps@table{htbp}
% \def\ftype@table{2}
% \def\ext@table{lot}
% \def\fnum@table{Table \thetable.}
% \def\table{\small\rm\@float{table}}
% \let\endtable\end@float
% \@namedef{table*}{\small\rm\@dblfloat{table}}
% \@namedef{endtable*}{\end@dblfloat}

%-------------------- Abreviations

\def\HEGRA{{HEGRA\xspace}}
\def\MAGIC{{MAGIC}\xspace}
\def\CTuno{{\sffamily CT1}\xspace}
\def\XTAL{{\mdseries\scshape xtal}\xspace}
\def\GMTQ{{\mdseries\scshape gmtq}\xspace}
\def\reflector{\texttt{reflector}\xspace}
\def\camera{\texttt{camera}\xspace}
\def\CORSIKA{{CORSIKA}\xspace}
\def\lastCORSIKA{X.X\xspace}
\def\trigger{\emph{trigger}\xspace}
\def\closepack{\emph{close-pack}\xspace}

%-------------------- Paragraphs and lines

%%\setlength{\parindent}{0pt}
\newcommand{\headname}[1]{#1}
\def\blankline{\vspace{0.4 cm}}
\def\indentfirstpar{\hspace{\parindent}}
%% \def\paragraph{\vspace{2pt}\par}

%-------------------- Math. symbols an so on...

\def\specialcolon{\mathrel{\mathop:}}

\def\defas{\doteq}

\def\mean#1{\ensuremath{\langle #1 \rangle}}
\DeclareMathOperator{\sign}{sign}
\DeclareMathOperator{\card}{card}

%%% Macro from "RockMover" <rmover@golovolomka.net>, The Master of Flame
\def\chronopair#1{\let\saveupbracefill\upbracefill
  \def\upbracefill{$\mathsurround0pt
  \kern2pt\vrule height4pt depth1sp 
  \leaders \hrule depth1sp \hfill
  \vrule height4pt depth1sp \kern2pt
  $}\underbrace{#1}\let\upbracefill\saveupbracefill}

\def\deg{\ensuremath{^\circ}\xspace}        %% degrees
\def\muon{\ensuremath{\mu}}                 %% muon
\def\muons{\ensuremath{\mu}'s}              %% muons
\def\cph{{\v C}-$\gamma$\xspace}              %% C-photon
\def\cphs{{\v C}-$\gamma$'s\xspace}           %% C-photon's
\def\phe{ph.e.\xspace}                      %% photo-electron
\def\phes{ph.e.s\xspace}                    %% photo-electrons
\def\mphe{\mathrm{ph.e}}                    %% photo-electron  (math-mode)
\def\mphes{\mathrm{ph.e.s}}                 %% photo-electrons (math-mode)
\def\wlrange{{290--600 {\it nm\/}}\xspace}  %% wavelength range used
%%\def\cerenkov{{\v C}erenkov\xspace}           %% Cherenkov
\def\cerenkov{Cherenkov\xspace}             %% Cherenkov
\def\Cerenkov{\cerenkov}                    %%    "
\def\cherenkov{\cerenkov}                   %%    "
\def\Cherenkov{\cerenkov}                   %%    "
\def\MonteCarlo{Monte Carlo\xspace}         %% Monte Carlo
\def\MC{\MonteCarlo}                        %%    "
\def\etal{{\sl et al.}\xspace}              %% et al. (slanted)
\def\QE{\ensuremath{QE}\xspace}                    %% Quantum Efficiency
\def\QEeff{\ensuremath{\langle\QE\rangle_\mathrm{eff}}\xspace}
\def\QElons{\ensuremath{\langle\QE\rangle_\mathrm{LONS}}\xspace}
\def\QEo{\ensuremath{\hat{\QE}}\xspace}

\def\ho{\ensuremath{h_0}\xspace}
\def\hc{\ensuremath{h_{\mathrm{C}}}\xspace}
\def\hv{\ensuremath{h_{\mathrm{v}}}\xspace}
\def\Hs{\ensuremath{H_{\mathrm{S}}}\xspace}

\def\Nphot{\ensuremath{%
  \mathcal{N}_{\gamma}}\xspace}
\def\Ntrial{\ensuremath{%
  \mathcal{N}_{\mathrm{ph.e.s}}}\xspace}
\def\Nmean{\ensuremath{%
  \bar{\mathcal{N}}_{\mathrm{ph.e.s}}}\xspace}
\def\Nrand{\ensuremath{%
  \mathcal{N}^{\mathrm{r}}_{\mathrm{ph.e.s}}}\xspace}
\def\Ntrialrand{\ensuremath{%
  \mathcal{N}^{\mathrm{t+r}}_{\mathrm{ph.e.s}}}\xspace}

\def\rhump{\ensuremath{\rho_{\mathrm{hump}}}\xspace}

\def\tRC{\ensuremath{\tau_{\mathrm{RC}}}\xspace}

\def\thetaCT{\ensuremath{\theta_{\mathrm{CT}}}\xspace}
\def\phiCT{\ensuremath{\phi_{\mathrm{CT}}}\xspace}

\def\thetam{\ensuremath{\theta_{\mathrm{m}}}\xspace} 
\def\phim{\ensuremath{\phi_{\mathrm{m}}}\xspace}

\newcommand{\trigM}[2]{\ensuremath{\mathrm{M}[#1;#2]}\xspace}
\newcommand{\trigNN}[2]{\ensuremath{\mathrm{NN}[#1;#2]}\xspace}
\newcommand{\trigNNc}[2]{\ensuremath{\mathrm{NN}^{\mathrm{c}}[#1;#2]}\xspace}
\newcommand{\trigTop}[2]{\ensuremath{\mathrm{Top}[#1;#2]}\xspace}

\newcommand{\eqcomm}[2][2cm]{\mbox{\hspace{#1} #2}} 

\def\d{\mathrm{d}}

\newcommand{\pow}[1]{\ensuremath{^{#1}}}        %% a\pow{x} = a^x
\newcommand{\E}[1]{\ensuremath{\mathord{\cdot}10^{#1}}}  %% a\E{x}   = a.10^x

\def\MORE{\par{\centering \mbox{} \hbox spread 200pt{%
--- T O --- B E --- C O N T I N U E D ---%
}\\}\par}

%-------------------- getting a chapter into the main TeX file

\newcommand{\echapter} {\newpage{\pagestyle{empty}\cleardoublepage}}
\newcommand{\get}[1]{\input{ #1 }\echapter}

%-------------------- bibliographic and Eq./Fig./Table references

\newcommand{\ccite}[1]{\protect\cite{#1}\xspace}     %% cite(s) in captions
\newcommand{\cref}[1]{\protect\ref{#1}\xspace}       %% references in captions
\newcommand{\fullcref}[1]{\protect\ref{#1} %
in page \protect\pageref{#1}\xspace}                 %% id.

\newcommand{\fullref}[1]{\ref{#1} in page \pageref{#1}\xspace} %% id., normal

\renewcommand{\u}[1]{\ensuremath{\mathrm{\,#1}}}    %% units

%\numberwithin{equation}{chapter}   %% don't needed for book class
\renewcommand{\theequation}{%
\oldstylenums{\thechapter}.\oldstylenums
{\arabic{equation}}}

%%\renewcommand{\footnotemark}{\alpha{\thefootnote}}

%%%%%%%%%%%%%%%%%%%%%%%%%%%%%%%%%%%%%%%%%%%%%%%%%%%%%%%%%%%%%%%%%%%%%%%%%%%
%% chapter's look
%%%%%%%%%%%%%%%%%%%%%%%%%%%%%%%%%%%%%%%%%%%%%%%%%%%%%%%%%%%%%%%%%%%%%%%%%%%

%% %\usepackage{oldgerm} 
%\usepackage{utopia} 

\makeatletter

%\DeclareFontFamily{U}{yswab}{}
%\DeclareFontShape{U}{yswab}{m}{n}{
%   <10> <10.95> <12> <14.4> <17.28>  <20.74> <24.88-> yswab   }{}

\def\MIDBONDINGLYHUGE{\fontsize{85}{95}\selectfont}


\renewcommand\chapter{\if@openright\cleardoublepage\else\clearpage\fi
                    \thispagestyle{plain}%
                    \global\@topnum\z@
                    \@afterindentfalse
                    \secdef\mychapter\myschapter}

%%----------------------------------------------------------------------

\def\mychapter[#1]#2{\ifnum \c@secnumdepth >\m@ne
                       \if@mainmatter
                         \refstepcounter{chapter}%
                         \typeout{\@chapapp\space\thechapter.}%
                         \addcontentsline{toc}{chapter}%
                                   {\protect\numberline{\thechapter}#1}%
                       \else
                         \addcontentsline{toc}{chapter}{#1}%
                       \fi
                    \else
                      \addcontentsline{toc}{chapter}{#1}%
                    \fi
                    \chaptermark{#1}%
                    \addtocontents{lof}{\protect\addvspace{10\p@}}%
                    \addtocontents{lot}{\protect\addvspace{10\p@}}%
                    \if@twocolumn
                      \@topnewpage[\mymakechapterhead{#2}]%
                    \else
                      \mymakechapterhead{#2}%
                      \@afterheading
                    \fi}

\def\mymakechapterhead#1{%
  \vspace*{20\p@}%
  {\parindent \z@ \raggedleft \normalfont
    \ifnum \c@secnumdepth >\m@ne
      \if@mainmatter
        \MIDBONDINGLYHUGE \thechapter
        \par\nobreak
        \vskip 15\p@
      \fi
    \fi
    \interlinepenalty\@M
    \Huge \sffamily #1\par\nobreak
    \vskip 20\p@
  }}

%%----------------------------------------------------------------------

\def\myschapter#1{\if@twocolumn
                   \@topnewpage[\mymakeschapterhead{#1}]%
                 \else
                   \mymakeschapterhead{#1}%
                   \@afterheading
                 \fi}

\def\mymakeschapterhead#1{%
  \vspace*{50\p@}%
  {\parindent \z@ \raggedright
    \normalfont
    \interlinepenalty\@M
    \Huge \bfseries  #1\par\nobreak
    \vskip 40\p@
  }}


\makeatother

\endinput

%\usepackage{oldgerm} 
%\usepackage{utopia} 

\usepackage{letterspace}

\makeatletter

%\DeclareFontFamily{U}{yswab}{}
%\DeclareFontShape{U}{yswab}{m}{n}{
%   <10> <10.95> <12> <14.4> <17.28>  <20.74> <24.88-> yswab   }{}

\def\MIDBONDINGLYHUGE{\fontsize{85}{95}\selectfont}


\renewcommand\chapter{\if@openright\cleardoublepage\else\clearpage\fi
                    \thispagestyle{plain}%
                    \global\@topnum\z@
                    \@afterindentfalse
                    \secdef\@chapter\@schapter}

%%----------------------------------------------------------------------

\def\mychapter[#1]#2{\ifnum \c@secnumdepth >\m@ne
                       \if@mainmatter
                         \refstepcounter{chapter}%
                         \typeout{\@chapapp\space\thechapter.}%
                         \addcontentsline{toc}{chapter}%
                                   {\protect\numberline{\thechapter}#1}%
                       \else
                         \addcontentsline{toc}{chapter}{#1}%
                       \fi
                    \else
                      \addcontentsline{toc}{chapter}{#1}%
                    \fi
                    \chaptermark{#1}%
                    \addtocontents{lof}{\protect\addvspace{10\p@}}%
                    \addtocontents{lot}{\protect\addvspace{10\p@}}%
                    \if@twocolumn
                      \@topnewpage[\mymakechapterhead{#2}]%
                    \else
                      \mymakechapterhead{#2}%
                      \@afterheading
                    \fi}

\def\mymakechapterhead#1{%
  \vspace*{20\p@}%
  {\parindent \z@ \raggedleft \normalfont
    \ifnum \c@secnumdepth >\m@ne
      \if@mainmatter
        \MIDBONDINGLYHUGE \thechapter
        \par\nobreak
        \vskip 15\p@
      \fi
    \fi
    \interlinepenalty\@M
    \Huge \sffamily #1\par\nobreak
    \vskip 20\p@
  }}

%\def\mymakechapterhead#1{%
%  \vspace*{20\p@}%
%  {\parindent \z@ \centering \normalfont
%    \ifnum \c@secnumdepth >\m@ne
%      \if@mainmatter
%        \Large \thechapter
%        \par\nobreak
%        \vskip 15\p@
%      \fi
%    \fi
%    \interlinepenalty\@M \Large \scshape
%    \centering  \letterspace to \linewidth{#1} \par\nobreak
%    \vskip 20\p@
%  }}

%%----------------------------------------------------------------------

%%-- These are the previous "schapter" defs ----------------------------
%\def\myschapter#1{\if@twocolumn
%                   \@topnewpage[\mymakeschapterhead{#1}]%
%                 \else
%                   \mymakeschapterhead{#1}%
%                   \@afterheading
%                 \fi}

%\def\mymakeschapterhead#1{%
%  \vspace*{50\p@}%
%  {\parindent \z@ \raggedright
%    \normalfont
%    \interlinepenalty\@M
%    \Huge \bfseries  #1\par\nobreak
%    \vskip 40\p@
%  }}
%%----------------------------------------------------------------------

\def\myschapter#1{\if@twocolumn
                   \@topnewpage[\mymakeschapterhead{#1}]%
                 \else
                   \mymakeschapterhead{#1}%
                   \@afterheading
                 \fi}

\def\mymakeschapterhead#1{%
  \vspace*{50\p@}%
  {\parindent \z@ \raggedleft \normalfont
    \interlinepenalty\@M
    \Huge \sffamily #1\par\nobreak
    \vskip 40\p@
  }}

\let\@chapter\mychapter
\let\@schapter\myschapter

\makeatother

%%%%%%%%%%%%%%%%%%%%%%%%%%%%%%%%%%%%%%%%%%%%%%%%%%%%%%%%%%%%%%%%%%%%%%%%%%%
%% hyphenation
%%%%%%%%%%%%%%%%%%%%%%%%%%%%%%%%%%%%%%%%%%%%%%%%%%%%%%%%%%%%%%%%%%%%%%%%%%%

\typeout{----> Defining hyphenation for several words . . .}

\hyphenation{
prima-rio 
secun-da-rio 
some-ti-dos 
pre-sen-ta-mos
obte-ni-das
MAGIC
}


%%%%%%%%%%%%%%%%%%%%%%%%%%%%%%%%%%%%%%%%%%%%%%%%%%%%%%%%%%%%%%%%%%%%%%%%%%%
%% bibliography
%%%%%%%%%%%%%%%%%%%%%%%%%%%%%%%%%%%%%%%%%%%%%%%%%%%%%%%%%%%%%%%%%%%%%%%%%%%

\typeout{----> Defining bibliographic style . . .}

% Next line is unnecessary in teTeX distribution, because another
% command, \bibname, is already defined.
% \renewcommand\refname{Bibliograf'ia}       %change as needed

\providecommand{\href}[2]{#2}   %generates the href macro if needed


\def\bhline{\noalign{\hrule height 1pt}}
\newcommand{\ms}{\noalign{\vspace{3pt}}}
\newcommand{\br}{\ms\bhline\ms}
\newcommand{\mr}{\ms\hline\ms}


%%%%%%%%%%%%%%%%%%%%%%%%%%%%%%%%%%%%%%%%%%%%%%%%%%%%%%%%%%%%%%%%%%%%%%%%%%%
%% Exceses
%%%%%%%%%%%%%%%%%%%%%%%%%%%%%%%%%%%%%%%%%%%%%%%%%%%%%%%%%%%%%%%%%%%%%%%%%%%

\typeout{----> Defining over/underfull conditions . . .}

%\hbadness5000
%\hfuzz40pt

%%%%%%%%%%%%%%%%%%%%%%%%%%%%%%%%%%%%%%%%%%%%%%%%%%%%%%%%%%%%%%%%%%%%%%%%%%%
%% Hyper-references
%%%%%%%%%%%%%%%%%%%%%%%%%%%%%%%%%%%%%%%%%%%%%%%%%%%%%%%%%%%%%%%%%%%%%%%%%%%

\ifpdf 
  \usepackage[%
    pdftex,
%    backref,
    linktocpage,
    pdfpagemode=UseOutlines, 
    pdfcreator={J.C. Gonzalez}, 
    pdfauthor={J.C. Gonzalez},
    pdfsubject={Ph.D. Thesis},
    pdftitle={Tesis},
    nesting=true,
    pdfview=FitBH,
    pdfstartview=FitV,
    bookmarks,
    bookmarksopen=true]{hyperref}
\fi

\makeatletter
\renewcommand{\@tocrmarg}{2cm}
\makeatother

%%%%%%%%%%%%%%%%%%%%%%%%%%%%%%%%%%%%%%%%%%%%%%%%%%%%%%%%%%%%%%%%%%%%%%%%%%%
%% definitions of style
%%%%%%%%%%%%%%%%%%%%%%%%%%%%%%%%%%%%%%%%%%%%%%%%%%%%%%%%%%%%%%%%%%%%%%%%%%%

\typeout{----> Defining style . . .}

% The commented blocks in this section define different "styles" or
% page/heading layouts

% We use the package 'fancyheadings'

%%------------------------------------------------------------
\makeatletter

\pagestyle{fancyplain}

\addtolength{\headwidth}{40pt}    % this for a4wide 
%\addtolength{\headwidth}{60pt}   % this for normal a4

%\addtolength{\hoffset}{30pt}

\renewcommand{\chaptermark}[1]%
{\typeout{CHAPTER: #1}\markboth{\thechapter. #1}{}\def\izquierda{#1}}
\renewcommand{\sectionmark}[1]%
{\typeout{SECTION: #1}\markright{\thesection. #1}\def\derecha{\thesection\ #1}}

\rhead{}
\lhead{}
\chead[%
    \if@mainmatter \fancyplain{}{\small\scshape\izquierda}% 
    \else \fancyplain{}{\sl\headname}%
    \fi]{%
    \if@mainmatter \fancyplain{}{\small\slshape\derecha}% 
    \else \fancyplain{}{\sl\headname}%
    \fi} 
%\chead{}

\rfoot{}
\lfoot{}
\cfoot{\fancyplain{\if@mainmatter \oldstylenums{\arabic{page}}\ %
       \else \itshape\roman{page}\ %
       \fi}{%
       \if@mainmatter \oldstylenums{\arabic{page}}\ %
       \else \itshape\roman{page}\ %
       \fi}}
%\cfoot{}

\makeatother

%%------------------------------------------------------------

%%------------------------------------------------------------
%\pagestyle{fancyplain}
%%
%\addtolength{\headwidth}{\marginparsep}
%\addtolength{\headwidth}{\marginparwidth}
%%
%\renewcommand{\chaptermark}[1]%
%   {\markboth{#1}{}}
%\renewcommand{\sectionmark}[1]%
%   {\markright{\thesection\ #1}}
%%
%\rhead[]{}
%\lhead[]{}
%\chead[\fancyplain{}{\sl\thechapter.~\leftmark}]
%       {\fancyplain{}{\sl\rightmark}}
%\cfoot{\sl\thepage}
%%------------------------------------------------------------

%%------------------------------------------------------------
%\pagestyle{fancyplain}
%\addtolength{\headwidth}{\marginparsep}
%\addtolength{\headwidth}{\marginparwidth}
%\renewcommand{\chaptermark}[1]%
%   {\markboth{#1}{}}
%\renewcommand{\sectionmark}[1]%
%   {\markright{\thesection\ #1}}
%\lhead[\fancyplain{}{\rm\thepage}]%
%   {\fancyplain{}{\sl\rightmark}}
%\rhead[\fancyplain{}{\sl\leftmark}]%
%   {\fancyplain{}{\rm\thepage}}
%\cfoot{}
%%------------------------------------------------------------

%%------------------------------------------------------------
%\pagestyle{fancyplain}
%%\addtolength{\headwidth}{\marginparsep}
%%\addtolength{\headwidth}{\marginparwidth}
%\addtolength{\headwidth}{2cm}
%\renewcommand{\chaptermark}[1]%
%   {\markboth{#1}{}}
%\renewcommand{\sectionmark}[1]%
%   {\markright{\thesection\ #1}}
%\lhead[\fancyplain{}{\bfseries\upshape \thepage}]%
%   {\fancyplain{}{\bfseries\itshape  \rightmark}}
%\rhead[\fancyplain{}{\bfseries \leftmark}]%
%   {\fancyplain{}{\bfseries\upshape \thepage}}
%\cfoot{}
%%------------------------------------------------------------

%%------------------------------------------------------------
%\pagestyle{plain}
%
%\makeatletter
%
%\let\@mkboth\markboth
%
%%% No foot
%\def\@oddfoot{\hfil{\rm \thepage}\hfil}
%\def\@evenfoot{\hfil{\rm \thepage}\hfil}
%
%%% Left heading
%\def\@evenhead{%
%  ~ \hfil {\sc \leftmark}  \hfil ~\\
%  \rule{\linewidth}{1pt}}
%
%%% Right heading
%\def\@oddhead{%
%  ~ \hfil {\sc \rightmark} \hfil ~\\
%  \rule{\linewidth}{1pt}}
%
%%% Chapter mark
%\def\chaptermark#1{\markboth {\it{\ifnum \c@secnumdepth >\m@ne
%      \@chapapp\ \thechapter. \ \fi #1}}{}}
%
%%% Section mark
%\def\sectionmark#1{\markright {\it{\ifnum \c@secnumdepth >\z@
%   \thesection. \ \fi #1}}}
%
%\makeatother
%%------------------------------------------------------------

%%%%%%%%%%%%%%%%%%%%%%%%%%%%%%%%%%%%%%%%%%%%%%%%%%%%%%%%%%%%%%%%%%%%%%%%%%%
%% graphics conversion rules
%%%%%%%%%%%%%%%%%%%%%%%%%%%%%%%%%%%%%%%%%%%%%%%%%%%%%%%%%%%%%%%%%%%%%%%%%%%

\ifx\pdfoutput\undefined 
\DeclareGraphicsRule{.jpg}{eps}{.jpg.bb}{`convert #1 'eps:-'}
\DeclareGraphicsRule{.png}{eps}{.png.bb}{`convert #1 'eps:-'}
\fi

%
%%% Local Variables: 
%%% mode: latex
%%% TeX-master: t
%%% End: 
%%EOF
