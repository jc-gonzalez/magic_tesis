%%%%%%%%%%%%%%%%%%%%%%%%%%%%%%%%%%%%%%%%%%%%%%%%%%%%%%%%%%%%%%%%%%%%%%%%%%%
%%
%%  thanks.tex
%%
%%  Created: Mon Oct 20 16:12:49 1997
%%  Author.: Jose Carlos Gonzalez
%%  Notes..:
%%          
%%-------------------------------------------------------------------------
%% Filename: $RCSfile$
%% Revision: $Revision$
%% Date:     $Date$
%%%%%%%%%%%%%%%%%%%%%%%%%%%%%%%%%%%%%%%%%%%%%%%%%%%%%%%%%%%%%%%%%%%%%%%%%%%


\typeout{----> Thanks to everybody . . .}

\pagestyle{plain}
\normalfont

\def\thanksname{Agradecimientos}
%\def\thanksname{Acknowledgements}

\chapter*{\thanksname}
\label{chapter:intro}
\markboth{\thanksname}{\thanksname}
\renewcommand{\headname}{\thanksname}
\addcontentsline{toc}{chapter}{\numberline{}\thanksname}

Este trabajo que ahora concluye ha ocupado los \'ultimos
cinco a\~nos de mi vida. Y en cinco a\~nos han sido muchas
las personas que me han ayudado y apoyado y que han confiado
en mi. A todos ellos les estoy agradecido, y se que ellos lo
saben.

En primer lugar me gustar\'{\i}a dar las gracias a mi directora de
tesis, Prof.\! D$^{\mathrm{a}}$ Victoria Fonseca Gonz\'alez,
a la cual debo la confianza que en mi deposit\'o cinco a\~nos atr\'as
cuando por primera vez puse pie en el grupo del departamento que ella
dirige. Ella me ha apoyado siempre y ofrecido oportunidades que no
creo hubiese tenido de otra manera.

Sin duda alguna uno de mis padres cient\'{\i}ficos es Razmick
Mirzoyan. El vio en mi de alguna manera una herramienta sin forma a la
cual habia que forjar y dotar de cierta utilidad. Para mi \'el ha sido
un profesor, un consejero y un amigo. Espero no haberle defraudado con
mi trabajo. Como maestro ha sido tambi\'en desde siempre Eckart
Lorenz, jefe del grupo HEGRA y ahora MAGIC del MPI de Munich, sin cuyo
est\'{\i}mulo m\'as de una vez estuve a punto de darme por vencido. Su
criterio cient\'{\i}fico y humano, firme pero comprensivo al mismo
tiempo, han servido muchas veces de ejemplo. Gracias a ellos este
trabajo ve la luz por fin.

Por supuesto, la amistad con los compa\~neros, de la que he gozado
tanto en el Grupo de Altas Energ\'{\i}as de la Universidad Complutense
como en el MPI de Munich, tambi\'en ha resultado fundamental para mi.
No solo he disfrutado con su compa\~n\'{\i}a, sino que he aprendido
much\'{\i}simo. He de citar a Juan Jos\'e Garc\'{\i}a Beteta y Juan
Alberto S\'anchez, Jes\'us Salgado, Juan Fern\'andez y Eduardo
Faleiro, compa\~neros con los que el contacto ha sido en los \'ultimos
a\~nos espor\'adico pero constante, a Ignacio y Jose Mar\'{\i}a
Fontanillo, nuestro flamante administrador del sistema, y a Luis
Padilla y Abelardo Moralejo, mis primeros compa\~neros en este
departamento y que intentaron ense\~narme todo lo que ellos
sab\'{\i}an y ayudarme en todo momento. Las gracias tambien a Aitor
Ibarra, por su constancia al ayudarme a dilucidar tantisimas dudas
relacionadas con mi trabajo, y por apoyarme personalmente en los
momentos dif\'{\i}ciles. A Jos\'e Luis Contreras por tanto como he
aprendido a trav\'es de innumerables charlas y discusiones, en el
departamento, durante las comidas o a la hora del caf\'e, y por sus
terribles desvelos a la hora de proporcionarnos un sistema
computacional estable y eficiente. Y a Juan Cortina, con el que he
podido compartir inquietudes en el campo de la simulaci\'on de
cascadas atmosf\'ericas, y sin cuya infinita ayuda y paciencia en
tantos momentos no hubiese sido capaz en absoluto de terminar este
trabajo.

En Munich tambi\'en he conocido a personas que me han prestado su
apoyo y su amistad. He de citar a Martin Kestel, siempre dispuesto a
ayudar a los dem\'as, y a Mar\'{\i}a D\'{\i}az Trigo, colega y amiga
personal, que siempre han sabido confiar en m\'{\i}. Como no, Sven
Denninhoff, que a parte de su trabajo como doctorando ha sido
encargado de la administraci\'on del sistema de ordenadores en Munich,
y cuyas numerables ayudas han resultado fundamentales. Y a Daniel
Kranich, cuyo apoyo y ayuda a nivel cient\'{\i}fico y amistad personal
han resultado tambi\'en clave en este trabajo.  Y quiero recordar
tambi\'en aqu\'\i\ la amistad y la ayuda que me ofrecieron en su
momento Christian Prosch, Dirk Petry y Stella Bradbury.

A muchas personas debo tambi\'en buenos ratos y discusiones
fructiferas. Entre otros quiero destacar a Norbert Magnussen, Manel
Mart\'{\i}nez, y sobre todo a Sasha Ostankov y Mireia Dosil. Y no
puedo olvidarme de Okkie de Jager, persona cuyo entusiasmo contagioso
tambien ha sido una incre\'{\i}ble fuente de est\'{\i}mulo. Gracias a
todos ellos.

Un abrazo de agradecimiento muy especial va para mi gran amiga Gemma
Romero, por el apoyo personal que ha resultado siempre y que sigue
resultando para m\'{\i}. Con ella he compartido siete a\~nos de mi
vida, y sin ella nunca hubiese realizado toda esta labor. Ella ha
sufrido m\'as que nadie mis malos momentos, pero siempre ha sabido
ofrecerme su cari\~no. Siempre se ha preocupado por mi y ha deseado
tanto como yo que este trabajo viese la luz. Espero que de alguna
manera pueda al fin sentirse orgullosa de m\'{\i}. Es a ella, y a mi
familia y mis amigos a quienes va mi m\'as cari\~noso agradecimiento.
Mis amigos me han regalado su maravillosa amistad y me han ofrecido su
apoyo en los momentos que he podido flaquear. Y mi familia,
especialmente mis padres, ha sido la referencia obligada en todo
momento. Por su infinita paciencia y su probada confianza, a todos
ellos les dedico este trabajo.

\echapter

\endinput
%
%%EOF
